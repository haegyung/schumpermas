Today, at around 8, my friend and colleague Matthias and I are welcoming our students to our class in "E-Democracy - Opportunities for Participatory Democracy?".
We look forward to learning (and so much more ...) with them for the next two and a half months, here at Deutsche SchülerAkademie in Brunswick. For me, it's the mother of all summer schools.
Here's the teaser with which we're beginning our teaching tonight.
[youtube=http://www.youtube.com/watch?v=inKqQjZU980&w=500&h=300&fmt=18&rel=0]
Learn more about our course here.
Heute abend um acht Uhr heissen mein Freund and Kollege Matthias Bröcheler und ich die Teilnehmer unseres Kurses "E-Democracy - Chance für partizipatorische Demokratie" zum ersten Mal willkommen.
Wir freuen uns auf die nächsten zweieinhalb Wochen Lernen (und so viel mehr ...) mit unseren Teilnehmern, hier auf der Deutschen SchülerAkademie in Braunschweig. Für mich: Die Mutter aller Sommerakademien.
It's a living utopia
http://en.wikipedia.org/wiki/Dunbar%27s_number
Mit diesem Teaser beginnen wir heute abend den Unterricht:
[youtube=http://www.youtube.com/watch?v=inKqQjZU980&w=500&h=300&fmt=18&rel=0]
Hier gibt's mehr über unseren Kurs.
It's a commons, a common pool resource problem made to work:
- identity: in/outgroup!
- this is a group of people, who, on average, have huge emotional and cognitive resources, and who are likely to be a selection
this isn't about them being essentially better, just with more opportunities.
wildavski



three ideas that make this a great human institutions.

my idea is not so much to be either socialist or capitalist.
The idea is to recognize that we are a very fallible species, we need a habitat that serves us well. That should guide us, nothing else. And I think we err on the side of too much inequality.

We don't produce for incentives, we produce and cooperate because we are social beings.

Key case: DSA:

features three ideas (that all have a dark side, too).

1) Flow experience
2) Intensely interconnected / the social mind
3) Equality and a sense of belonging (resolving the prisoner's dilemma)

and then relate all of this to SM.



van den Berghe: Gruppe
Nepotism vs. Arendt / kommunistische Fiktion
Abgrenzung
FMRI
Freedom from need is arendt, and also solves prisoners dilemma
Spiel real machen (das haben wir gemacht, geht aber noch mehr nächstes mal echte payofss

Thema: Freizeit, ein fremdes Wort auf der DSA. Nicht so hier.

Vgl. counterdiminance strategy 

vgl. Arendt: reziproker altruismus ist minimal. Hoier geht's um die anthropologische Kapazität

Bingo: Arendt/Eichmann Pol Gemeinschaft darf nicht Grund für den Staat sein
Arendt: redet über Massengesellschaft
v.d. Berghe: nepotismus in der Moderne rastat aus.

Humus für Faschismus

Freiheit von öfftl. Zwängen ist Arendt, vgl. mit Rawls


Pace, S. (2004). A grounded theory of the flow experiences of Web users. International Journal of Human-Computer Studies, 60(3), 327-363. doi: 10.1016/j.ijhcs.2003.08.005.

329: Flow is a state of consciousness that is sometimes experienced bypeo ple who are
deeply involved in an enjoyable activity. The experience is characterized by some
common elements: a balance between the challenges of an activityand the skills
required to meet those challenges; clear goals and feedback; concentration on the
task at hand; a sense of control; a merging of action and awareness; a loss of selfconsciousness;
a distorted sense of time; and the autotelic experience
 



