\documentclass[11pt,a4paper,oneside,openright]{article}
	\usepackage{./tex/mystyle}

\hypersetup{
	pdftitle=(Welfare-Held),
	pdfauthor=(Maximilian Held), 
	pdfcreator=(Maximilian Held), 
	pdfsubject=(Term Paper),
	colorlinks=true, 
	linkcolor=maxgreen, 
	citecolor=black, 
	filecolor=maxgreen, 
	urlcolor=maxgreen
}

\title{
	Welfare-Held 
}

\author{
	\href{http://www.maxheld.de}{Maximilian Held}
}
	
\date{\today} 

\begin{document}

\bibpunct[:~]{(}{)}{;}{a}{}{,} %http://merkel.zoneo.net/Latex/natbib.php

%!TEX root=../tax-n-democracy-held.tex

\newacronym{CA}{CA}{British Columbia Citizens' Assembly on Electoral Reform}

\newacronym{BIG}{BIG}{Basic Income Guarantee}

\newacronym{BIGSSS}{BIGSSS}{Bremen International Graduate School of Social Sciences}

\newacronym{BoP}{BoP}{Balance of Payments}

\newacronym{CAFE}{CAFE}{Corporate Average Fuel Economy}

\newacronym{CEEC}{CEEC}{Central and Eastern European Country}

\newacronym{CBA}{CBA}{Cost-Benefit Analysis}

\newacronym{C}{C}{Consumption}

\newacronym{CIT}{CIT}{\hyperref[sec:CIT]{Corporate Income Tax} (p.~\pageref{sec:CIT})}

\newacronym{CPR}{CPR}{Common-Pool Resource}

\newacronym{CME}{CME}{Coordinated Market Economy}

\newacronym{Dual-PIT}{Dual-PIT}{\hyperref[sec:Dual-PIT]{Dual Personal Income Tax} (p.~\pageref{sec:Dual-PIT})}

\newacronym{DQI}{DQI}{Discourse Quality Index}

\newacronym{DNA}{DNA}{deoxyribonucleic acid}

\newacronym{DP}{DP\textregistered}{Deliberative Poll \textregistered}

\newacronym{DWL}{DWL}{Deadweight-Loss}

\newacronym{d}{d}{Depreciation}

\newacronym{EC}{EC}{European Commission}

\newacronym{ECB}{ECB}{European Central Bank}

\newacronym{Ecotax}{Ecotax}{\hyperref[sec:Ecotax]{Quantity Taxation of Energy} (p.~\pageref{sec:Ecotax})}

\newacronym{EU}{EU}{European Union}

\newacronym{EFC}{EFC}{European Fiscal Compact}

\newacronym{EMU}{EMU}{European Monetary Union}

\newacronym{EPL}{EPL}{Employment Protection Legislation}

\newacronym{ESM}{ESM}{European Stability Mechanism}

\newacronym{ET}{ET}{\hyperref[sec:ET]{Expenditure Tax} (p.~\pageref{sec:ET})}

\newacronym{FCE}{FCE}{Final Consumption Expenditure}

\newacronym{FDI}{FDI}{Foreign Direct Investment}

\newacronym{FPE}{FPE}{Factor Price Equalization}

\newacronym{FPTP}{FPTP}{First-Past-the-Post}

\newacronym{FRG}{FRG}{Federal Republic of Germany}

\newacronym{FTT}{FTT}{\hyperref[sec:FTT]{Financial Transaction Tax} (p.~\pageref{sec:FTT})}

\newacronym{G}{G}{Government Spending}

\newacronym{GDR}{GDR}{German Democratic Republic}

\newacronym{GDP}{GDP}{Gross Domestic Product}

\newacronym{GNP}{GNP}{Gross National Product}

\newacronym{GFCF}{GFCF}{Gross Fixed Capital Formation}

\newacronym{HT}{HT}{Head Tax}

\newacronym{I}{I}{Investment}

\newacronym{IFI}{IFI}{International Financial Institution}

\newacronym{ICT}{ICT}{Information \& Communication Technology}

\newacronym{ISI}{ISI}{Import-Substitution-Industrialisation}

\newacronym{KJV}{KJV}{King James version}

\newacronym{LBT}{LBT}{\hyperref[sec:LBT]{Local Business Tax} (p.~\pageref{sec:LBT})}

\newacronym{LDC}{LDC}{Less Developed Country}

\newacronym{LME}{LME}{Liberal Market Economy}

\newacronym{LVT}{LVT}{\hyperref[sec:LVT]{Land Value Tax} (p.~\pageref{sec:LVT})}

\newacronym{NFCF}{NFCF}{Net Fixed Capital Formation}

\newacronym{NHS}{NHS}{National Health Service}

\newacronym{NIT}{NIT}{\hyperref[sec:NIT]{Negative Income Tax} (p.~\pageref{sec:NIT})}

\newacronym{NTT}{NTT}{\hyperref[itm:NTT]{New Trade Theory (p.~\pageref{itm:NTT})}}

\newacronym{NW}{NW}{Net Worth}

\newacronym{NOMA}{NOMA}{Non-Overlapping Magisteria}

\newacronym{MECE}{MECE}{mutually exclusive and comprehensively exhaustive}

\newacronym{MS}{MS}{Member State}

\newacronym{OCA}{OCA}{\hyperref[sec:OCA]{Optimal Currency Area (p.~\pageref{sec:OCA})}}

\newacronym{OECD}{OECD}{Organisation for Economic Co-Operation and Development}

\newacronym{OMC}{OMC}{Open Method of Coordination}

\newacronym{OMO}{OPO}{Open Market Operations}

\newacronym{OSN}{OSN}{\hyperref[sec:OSN]{Ordinary Savings Norm (p.~\pageref{sec:OSN})}}

\newacronym{PAYG}{PAYG}{Pay-As-You-Go}

\newacronym{Payroll}{Payroll}{\hyperref[sec:Payroll]{Payroll Tax (p.~\pageref{sec:Payroll})}}

\newacronym{PD}{PD}{Prisoner's Dilemma}

\newacronym{PCT}{PCT}{\hyperref[sec:PCT]{Progressive Consumption Tax (p.~\pageref{sec:PCT})}}

\newacronym{PIT}{PIT}{\hyperref[sec:PIT]{Personal Income Tax} (p.~\pageref{sec:PIT})}

\newacronym{PPACA}{PPACA}{Patient Protection and Affordable Care Act, a.k.a. ``Obamacare''}

\newacronym{PPF}{PPF}{Production Possibility Frontier}

\newacronym{PPP}{PPP}{Purchasing Power Parities} 

\newacronym{PT}{PT}{\hyperref[sec:PT]{Property Tax} (p.~\pageref{sec:PT})}

\newacronym{QE}{QE}{Quantitative Easing}

\newacronym{RBCT}{RBCT}{Real Business Cycle Theory}

\newacronym{RET}{RET}{Rational Expectations Theory}

\newacronym{RnD}{R\&D}{Research \& Development}

\newacronym{SD}{SD}{\hyperref[sec:WT]{Stamp Duty} (p.~\pageref{sec:WT})}

\newacronym{SGP}{SGP}{Stability and Growth Pact}

\newacronym{SIC}{SIC}{\hyperref[sec:SIC]{Social Insurance Contributions}, de facto capped \hyperref[sec:Payroll]{Payroll Taxes} (p.~\pageref{sec:SIC})}

\newacronym{SME}{SME}{Small and Medium-sized Enterprise}

\newacronym{STV}{STV}{Single Transferable Vote}

\newacronym{SWF}{SWF}{Sovereign Wealth Fund}

\newacronym{Tax LSE}{Tax LSE}{Tax Liability Side Equivalence, equivalently known as \emph{invariance of incidence proposition} or \emph{Dalton's Law}}

\newacronym{TFR}{TFR}{Total Fertility Rate} % the average number of children a woman would have by age 50 based on the current age-specific fertility rates

\newacronym{TFP}{TFP}{Total Factor Productivity}

\newacronym{VAT}{VAT}{\hyperref[sec:VAT]{Value-Added Tax} (p.~\pageref{sec:VAT})}

\newacronym{vNM}{vNM}{\hyperref[sec:rational]{von-Neumann-Morgenstern}}

\newacronym{WT}{WT}{\hyperref[sec:WT]{Wealth Tax} (p.~\pageref{sec:WT})}

\newacronym{Y}{Y}{Output}

\newacronym{Y2C}{Y2C}{\hyperref[sec:Y2C]{yield-to-capital (p.~\pageref{sec:Y2C})}}


%\setcounter{page}{3} %change this!

\maketitle
\thispagestyle{empty}
	\begin{center}	\vspace{15pt}
	{\large Term Paper}\\ 	\vspace{20pt}
		
\begin{tabular*}{0.35\textwidth}{@{\extracolsep{0cm}}rl}
	{\large Student ID:}	&
	{\large 089145}\vspace{10pt} \\
	
	{\large Class:}		&
	{\large Welfare States and Health Wellbeing}\\
	\vspace{10pt}\\
	
	{\large Instructors:}	&
	{\large Prof. Dr. Klaus Hurrelmann}\\
	 &
	{\large Dr. Katharina Rathmann}\\
	\vspace{40pt}\\ 
\end{tabular*}
\end{center}

\begin{abstract}
	%blah
\end{abstract}

%!TEX root=../tax-democracy-held.tex

\section*{How To \ldots} \label{chap:how2}

\paragraph{How to Use This Document.}
I have highlighted some words in \href{chap:how2}{green} to point to further information in footnotes, elsewhere in the document or on the internet.
In the digital version of this document, these highlights hyperlink to their respective place of finding.

I have also hyperlinked in-text citations to their entry in the bibliography.

Many of the e-books I have read do not include the original print-edition page numbers.
I reference these e-books with their (proprietary) \emph{Amazon Kindle \circledR} reading locations, identified as ``Kxxx'', such as in \citet[K50]{McCaffery2002}.
You can:
\begin{enumerate}
	\item
		Find the original source on \href{http://kindle.amazon.com/profile/Maximilian-Held/1675396}{my \emph{Kindle \circledR} account} at\\ \url{http://kindle.amazon.com/profile/Maximilian-Held/1675396}.
	\item
		Find the print-edition page number by typing the respective quote into \url{http://books.google.com}.
	\item
		Contact me to find the original source.
\end{enumerate}

\paragraph{How Not to Use This Document.}
This is an early draft of unpublished work that has not been reviewed. Please do not cite or circulate any of this.

\paragraph{Correspondence.}
Write to \href{mailto:maximilian.held83@gmail.com}{maximilian.held83@gmail.com} or see \\* \url{http://maxheld.de/contact} for more information.


\tableofcontents
	
\listoffigures
	
\listoftables 

%!mainmatter

\newpage

\begin{quote}
	\emph{``B\"uck dich hoch, ja!''}
		\footnote{
			Own translation: \emph{Bend over upwards!}
		}
		\\*
	--- Deichkind: Befehl Von Ganz Unten (2012)
\end{quote}


%some quote? bück dich hoch?

%blah.

%For sections in the course
	%Health and Wellbeing in WS
	%Health Inequalities in WS
	%Comparative Case Studies
	%Recommendations for Public Health and Health Equity Strategies
	
	% There is evidence that public health strategies with an explicit public policy component are relatively successful. With these strategies, the emphasis shifts from the traditional approach of planning, funding and delivery of healthcare services per se (“sector strategy”) to a comprehensive policy which addresses a much wider range of economic, social, cultural, educational and environmental forces that have an impact on the health of individuals and population subgroups. Students will be trained to analyse the results of this “inter sector strategy” with an emphasis on universal public policies. (from the syllabus)
	
%comment on scale-free allocations

%We don't produce for incentives, we produce and cooperate because we are social beings.

%1) Flow experience

%relate this to SM?

%other good case would be US states, because they are smaller -- but it's not clear that savings rate can be meaningfully measured at that level, think about that.

%talk about the importance of ``shame''.

%cite how the easterlin thing levels off

%this will be a study, in \cite{Wilkinson1999}'s words, at the \emph{ecological} level.

%\cite{Wilkinson1999}
	%other reason might be: social cohesion
	%also notes the same two modes of human coexistence (status, power, coercion. vs. mutuality etc). and cites putnam et al (8) as source: the networks in southern italy are based on status.
	%534: suggests that actually, ``Income inequality is more strongly related to health when measured across large areas (countries or states) than across smaler areas (counties or census tracts) because, at the most local level, residential segregation means that the important social heterogeniety across the social hierarchy is lost.
	%great quote 535:  Although there are still deep oppositions of interests between rich and poor, powerful and powerless, exploiter and exploited, all these relationships are now graded continua. Nor does anything controlling the development of human societies say that there will always be dis- tinct social classes clearly demarcated by different functional relations to the pro- ductive system. Even in the top 5 percent of incomes, most people are “mid- dle-class” salaried employees, dependent mainly on selling their labor. On the other hand, 44 percent of American households now own capital investments and, if income from other forms of property were included, the proportion would be still higher; yet most of these people are simultaneously employees (44). However, despite wider share ownership, the bulk of capital is now controlled by institutions such as pension funds, insurance companies, banks, and building societies or mortgage companies.
	%539: Muntaner and Lynch (?) criticize Wilkinson for getting selling out equality to ``cohesion'' etc. That's not true: ``Our role has been pre- cisely the opposite—to show the importance of income distribution both to health and to the nature of civil society. Indeed, I regard the evidence that income distribution gives us a handle on the psychosocial welfare of populations as the most important single feature of these relationships. Instead of being seen as a diversion from tackling income inequality, the link with social cohesion might be regarded as helping to put greater equality back on the political agenda by mak- ing it more attractive.'' -- but this is an important beef
	
%\cite{Patychuk2010} this stuff, focusing about health inequities, is still very much about absence of service or material deprivation, be it of health care. this may be all well and good, but it may miss the bigger problem.

%maybe, instead of TWW, we ought to look at other macro-regimes and performance, including savings rate. Bambra, e.g. \cite{Bambra2007a} has argued that TWW is flawed.

%i should probably consider how the savings rate is likely to correlate with different welfare regimes (if it does so at all), as in:
%\cite{Bambra2007a}: ``This has led some to question the validity of the regimes concept itself as it assumes that most of the key social policy areas within a welfare regime will reflect a similar, across the board, approach to welfare provision; and second, that each regime type itself reflects ‘‘a set of principles or values that establishes a coherence in each country’s welfare package’’.38''

%add that not only should we NOT care only about cash benefits (as \cite{Bambra2005} reminds us), but we should also care about non-welfare welfare outcomes

%Steward & Langer 2007
%I am not sure I agree that horizontal and vertical inequality is, in fact, different things. Maybe vertical inequality is the same, it's just that horizontal inequality is tightly grouped (think network theory)
%Nice definition: inequality between groups is then the consequence of inequality in asset ownership between grups and inequalities in the returns to these assets. Assets include land, financial assets, education, public infrastructure and social capital (5)
%Reasons why inequality persists:
%1) cumulative forces: people who have money find it easier to get more money, to find jobs, borrow, invest et al.
%2) the returns to different kinds of capital (financial, social, human) interact with the level of other capital. You can make more off a given level of financial capital given higehr social capita.
%3) persistent assymetries in social capital, which cause unequal returns on other types of capital. Poor people tend to know more poor people. (also known as: the network effect)
%4) discontinuities in reutnrs to diferent types of capital. Low levels of capital trap you righ thre. This could be due to the type of interaction discussed on the above, OR because of indivisibilities in that capital.
%5) overt or implicit discrimination
%keep in mind as they write: (9) present inequality could be interpreted as a result of past discrimination
%it need not be, as Tilly and brown seem to think, hoarding. Just a network effect will do.
%again, the counterfactual of inequality is wrong. it's not: absence of horizontal inequality between groups. that's bad. the correct one is: everyone according to his ability. Blog this.
%write this: it's not about marxism or relative deprivation. It's just a condition of postindustrial production, that's all.
%contrast my idea of reinforcing inequality to ideas of comparative advantage.
%the very idea of neo-classical economics, that ienqualities would balance out because of diminishing returns to capital, appears to be wrong (29). and here is why: networks. 
%
%Cozzi & Privileggi 2009
%Basic assumption: growth and polarization are correlated
%they argue that what leads to the gross inequalities is in fact, paradoxically, social mobility (and NOT Marxs classes!)
%this is the fractal nature they mean: over time.
% show the self-similarity on page 19. this is great for an intuition
%
%Keller 2005
%"the power-law connectivity distribution seen in scale-free networks seems to emerge as one of the very few universal mathematical laws of life" )wolff et al as cited in Keller 2005: 1061

%\cite{Bambra2005a}
	%187, indeed: ``Health, and its promotion, are profoundly political''
	%188: ``Health is often reduced and misrepresented as health care''
	%188: ``The definition of health that has conventionally been operationalized under Western capitalism has two interrelated aspects to it: health is both considered as the absence of disease (biomedical definition) and as a commod- ity (economic definition). These both focus on individuals, as opposed to society, as the basis of health: health is seen as a product of individual factors such as genetic heritage or lifestyle choices, and as a commodity that individuals can access either via the market or the health system (Scott-Samuel, 1979).''
	
	

%Pace, S. (2004). A grounded theory of the flow experiences of Web users. International Journal of Human-Computer Studies, 60(3), 327-363. doi: 10.1016/j.ijhcs.2003.08.005.
%
%329: Flow is a state of consciousness that is sometimes experienced bypeo ple who are
%deeply involved in an enjoyable activity. The experience is characterized by some
%common elements: a balance between the challenges of an activityand the skills
%required to meet those challenges; clear goals and feedback; concentration on the
%task at hand; a sense of control; a merging of action and awareness; a loss of selfconsciousness;
%a distorted sense of time; and the autotelic experience
%
%compare burnout research and/or relative inequality () to happiness findings: does relative inequality matter, or not?
%
%look at the income decile variables vs. happiness in the WVS, make it individual level.
%Also consider looking at job characteristics.
%
%compare that to veenhoven’s cross-country comparison.

%look at critique of wilkinson/pickett


%\cite{Dwyer2009}
	%thorstein veblen (conspicuous consumption) built this tradition, but also bourdieu (distinction).
		%note (as \cite{Dwyer2009} points out, that bourdieu is about distinction, which need not be as hstrictly hierarchical as cc, for which Veblen has gotten so much crap
	%luxuries become decencies become necessities (also de botton)	%	as per ibid, there is somedisagreement whether a widening of income inequality also leads to more excessive consumption: frank assumes (1999) that this must be so as an aggregate phenomenon, b/c of expenditure ascades: the frame of reference shifts upwards.
	% Slesnick 2001 seems to disagree, as cited in ibid: 340
	% 340: Indeed, Leicht and Fitzgerald (2006) argue that the middle class has been ‘lent what it should have been paid’ and that a middle-class lifestyle was maintained during a time of stagnant income growth for many at the cost of financial security.
	
	%part of my point here is also not to empty the baby of growth with the bathwater, as \cite{Jackson2008} do.
	
%note that the macroeconomic data are bad: GNP doesn't cut it, neither do most of the other available data: state-owned stuff is not included, and GDP is by definition a poor measure.
% look at it over time, that yields more cases

%\cite{Jackson2008}
	%consider doing logarithmic or polynomial correltion for any non-linear effect. That is in fact what I assume: that it IS non-linear
	%711: ``It is important not to read too much into these correlations. For one thing, it is clear that even a strong correlation does not imply causality. Moreover other med- iating factors in reported life satisfaction are left out of the analysis here. The literature on wellbeing is clear that family, friendship, trust in one’s neighbours, security, meaningful work, and a sense of purpose are all important correlates of subjective wellbeing (Helliwell 2003, Layard 2005, Dolan et al. 2006). None of these is necessarily mediated by consumption expenditure.''
	%712: `Nonetheless, it is clear that certain kinds of human tendencies – the pursuit of status, social belonging, identity and meaning, for example – are (in the modern society at least) strongly mediated by material goods and services. Possessions confer status, solidity and meaning to our lives, provide access to the social world and facilitate participation in the creation of that world (Douglas 2006 [1976]). To the extent that this is so, consuming “comes naturally” to the human species, whereas, as Dawkins (2001) has pointed out, sustainability does not.''
	%713: ``Recent evidence has shown how closely health and wellbeing are related to social status even in developed countries (Wilkinson 2000, 2005, Marmot 2005, Marmot and Wilkinson 2005). This would suggest that beyond the economy of consump- tion lies an “economy of wellbeing”. It may not pay to increase consumption in the aggregate, but it pays to be at the top of the pile when consumption is unequally distributed. Evidence''
	%713: ``This positional pathology is exacerbated by another psycho-social phenomenon affecting subjective wellbeing, that of “hedonic adaptation” (Lyubomirsky et al. 2005). As I get richer, I simply become more accustomed to the pleasure of the goods and services my new income affords me. And if I want to maintain the same level of happiness, I must achieve ever higher levels of income in the future just to stay in the same place. A related understanding of human psychology becomes relevant here: in a society where visible signals of excess surround us on every side, the potential discrepancy between what we have and what we would like to have grows ever larger. Even as we chase after the most recent symbols of success, the frontier of success is moving ahead of us. Research supports the idea that this kind of discrepancy between our aspirations and the reality of our lives is psychologically damaging to our quality of life (Michalos 1985, Higgins 1987).''
	
%check out SM literature

%\cite{Frank1950}
	%this is on savings rates, yeah!
	%there are great inequality stats in here, especially before and after tax!
	%also agrees that data is hard to come by: 11: ``only fragmentary data exists''
	%argues that regional proximity matters (13: actually heÄd like more city, less country-level data, but I'm not sure that with pervasive media, that still works.
	%interestingly, frank tests the inverse: "do people who live in high-inequality jurisdictions in fact save at lower rates than those who live in low-inequality jurisdictions" (13)
	%that is a problem for my stats, but also note that savings rates MAY be collectively ruled.
	%also compare this to the tax mix?
	
	
%check out all Frank stuff
%check out all
	
	
	%methodological comments: don't do methodological nationalism, but well, that's the conventional context by which data is ordered. Note that in principle, other regional or non-regional contexts would be possible (for example: an occupation that saves more than another, etc).
	
	%also the dependent variable is still a tad bit problematic: life statisfaction, SWB? But how does that relate to health?

%note the coping strategies from de botton, religion etc.

%also note: status competition is kind of OUR question as a species, and it must not be reduced to a straightforwardly materialist account, especially not a financial matter. We play this game in many ways.

%check out easterlin

%\cite{Koenig2013}
	%suggests an efficiency ground for public provision: if its publicly provided, there won't be conspicuous consumption, as, for example, in education (which is HIGHLY positional!) Note that this is essentially the same point as that for mandatory vacation and working hours: that is publicly provided, too, because otherwise a positional race may ensue.
	
%
%	compare burnout research and/or relative inequality () to happiness findings: does relative inequality matter, or not?
%	look at the income decile variables vs. happiness in the WVS, make it individual level.
%	Also consider looking at job characteristics.
%	compare that to veenhoven’s cross-country comparison.

%garrard 2012
	% briefly reference Rousseau "obsessive being-for-others" with whom "one is forced to compare onself at each instant".
	% 382: ``The possession of positional goods has also been associated with measurable levels of increased well-being, while their absence has been linked with increased stress, status anxiety and other adverse psychological and physical effects such as depression, heart disease and reduced life expectancy''.
	%382``There is some evidence of a connection between blood concentrations of the neurotransmitter serotonin—the so-called ‘happiness hormone’—and dominance. Unusually high levels of serotonin have been found in officers of college fraternities, athletic team captains and officers and crew members on an extended sailing voyage (Frank 1985: 23–8). Richard Wilkinson has linked being in a subordinate position with heightened levels of stress and depressed levels of serotonin (Wilkinson 1996). Increased levels of the central stress hormone cortisol, which can be measured in blood and saliva, have been found in individuals whose self-esteem and social status are threatened, and stress has been positively linked to The status of happiness 383 poor health and increased obesity. Low or declining social status has also been associated in some studies with poor cardiovascular health (Wilkinson and Pickett 2009: 38, 191; Marmot 2004).
	%383: Rousseau may be right, but it might be nature already: While there is some evidence that he was more or less right about this, it seems as though our preoccupation with status and relative position is rooted in our natures which, according to evolutionary psychology, are ‘adaptive responses shaped by man’s biological nature and situation on earth’ (Hirschleifer 1978: 321).
	%. ‘[T]he farther an individual fell in his local pecking order,’ according to economist Robert Frank, ‘the more serious were the threats to his survival,’ which typically provoked feelings of stress and anxiety affecting levels of testosterone and serotonin, as we have already seen (Frank 1999: 136). In such a setting, ‘low social status is an evolutionary dead end,’ to put it bluntly (Wilkinson and Pickett 2009: 204). For
	%``What mattered for survival in this environment was one’s position relative to local rivals with whom individuals and groups were in direct, immediate competition for food and sex, ‘whereas comparisons with others who are distant in time and space are typically irrelevant’, from an evolutionary perspective (Frank 2007: 57). One of the central findings of Robert Frank is that most people today are relatively indifferent to the wealth and status of people far removed from them, whereas they ‘care deeply about where they stand in their various local hierarchies’ (Frank 1985: 107). Hence,''
	%``For example, the voluntary simplicity movement seeks to overthrow ‘the god of positional consumption’ (Schor 1999: 61) in the name of reduced consumption, anti- materialism and sustainable development. A policy of deliberate ‘downshifting’ typically involves working less, earning less and consuming less so that there is more time to better enable people to pursue other goods. Such a lifestyle would be materially poorer than now but richer in non-material, non-positional goods and therefore happier. To the extent that it is possible to reduce the speed of the consumption treadmill so that less time and fewer resources are expended on the joyless pursuit of positional goods and status, we may be able to devote ourselves to more intrinsically satisfying activities and forms of consumption that actually improve our quality of life, such as higher air quality, more urban parkland, cleaner drinker water, reducing violent crime (Frank 1999: 11).''
		%comment on this kind of anti-growth stuff; it might be anti-growth, but anti-spending might be even better.

%note that actually, the utilitarian undertones of ``happiness'' are murky.

%\cite{Linssen2011} find that even in rural India (!), conspicuous consumption seems to depress SWB.
	%they also note (!) 70 that the causality may yet run differently: maybe, people who experience low SWB compensate by doing more conspicuous consumption -- this, luckily, is a problem that should not affect aggregate anaylyses.
	%social comparison is, of course a zero-sum game.
	
	
	
%note the problem of inverse causality: a higher savings rate, and, equivalently, lower debt may also be the EFFECT, not the CAUSE: as people lower in the rank overs pend to emulate those up top.
	
%comment towards the end how this, as much of the questions that really matter, escape empirical analysis, certainly of survey data. We might need (quasi-)experiments, or maybe, just logic.


%!backmatter
	
	%Glossary
		\glsaddall
		\renewcommand*{\glspostdescription}{}
		\printglossaries
	
	%Bibliography
		\bibliography{./tex/library} % Bibliography database file, moga.bib 
		\bibliographystyle{cje} % Bibliography style file
			%this doesn't seem to adequately report working papers.
	
\end{document}