%preamble
	\documentclass[]{beamer}
	\usepackage{../latexstyles/pres}

%\includeonly{./tex/proposal-phd}

	\hypersetup
		{pdftitle=((Mis)understanding the Fine Print of the Social Contract: The Pluralist and Deliberative Politics of Taxation),
		pdfauthor=(Maximilian Held), pdfcreator=(Maximilian Held), pdfsubject=(PhD Thesis),
		colorlinks=true, linkcolor=green, citecolor=black, filecolor=green, urlcolor=green}%change title here, too.


\title[Kurzform]{(Mis)understanding Tax \\
--
\\ The Deliberative and Pluralist Politics of Tax}
\subtitle[Kurzform]{A PhD Proposal}
\author[Max Held]{Maximilian Held}
\institute[BIGSSS]{Bremen International Graduate School of Social Sciences\\ Universität Bremen \& Jacobs University Bremen}
\date[26.05.06]{26. Mai 2006}
%\titlegraphic{\includegraphics[width=2cm,height=2cm]{hulogo}}
\subject{}
\keywords{}

\begin{document}

\begin{frame}
	\frametitle{Table of Contents}
	\tableofcontents
%[currentsection]
\end{frame}

\titlepage

\section{Background}

\begin{frame}
\frametitle{Research Question}

How would people think \emph{differently} about taxation, if they had the opportunity to deliberate it carefully?

% The present research investigates how peoples thinking on taxation changes after they have participated in a citizen conference on the topic, an extensive, small-n deliberative format.
% An appropriate method for the research question is hard to find. Quantitative survey instruments (such as Fishkin's Deliberative Polls) work only for large samples, and greatly constrain participant responses to pre-theorised concepts. Qualitative approaches such as discourse analysis are more open-ended, but risk affording the researcher too much leeway in interpreting data, especially when a specific change in thinking is hypothesised.
% At the same time, operationalisations for what constitutes good deliberation are varied and contradictory (Bohman 1998). Merely procedural (Fishkin) or syntactical standards (Steiner 2012, Steenbergen 2003) fall short of a Habermasian standard of communicative rationality, but more demanding standards risk collapsing deliberative democracy into a substantive theory of justice (Gutmann & Thomspon 2004).
% Dryzek and Niemeyer's (2007) definition of (successful) deliberation as increased meta-consensus and intersubjective rationality may resolve both these problems. Meta-consensus increases if people agree about the nature of competing preferences, values and beliefs at stake. Intersubjective rationality increases if people who share values and beliefs also become more likely to prefer similar policies (and vice versa). This approach serves well to falsify the very tenet of (Habermasian) deliberative democracy, namely, that people can understand differing arguments on their own terms, and may, thereby, approximate a universal rationality.
% Concerned with the structure of subjective political beliefs, Dryzek and Niemeyer's definition lends itself to a q-methodological operationalisation. In a Q study, participants are asked to rank order (many) statements according to some condition of instruction, such as agreement. Rank positions of items (as cases) are then factor analysed over participants (as variables), inverting rows and columns of the data table of conventional (R-type) factor analysis, yielding factors of items whose rank correlates. Factors are then interpreted by comparing the factor scores on items by a computed, archetypical individual with a maximal loading on that factor.
% In the present research, participants are asked to rank-order items of values, beliefs, preferences (taxes) on the economy. Q results are compared before and after the citizen conference on taxation. It is hypothesised that after the deliberation, fewer factors with higher loadings emerge (meta-consensus increases), and that factor scores align more strongly along intersubjectively rational values, beliefs and preferences.


\end{frame}

\section{Method}

\begin{frame}

\end{frame}


\end{document}
