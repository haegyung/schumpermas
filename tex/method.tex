%!TEX root=../tax-democracy-held.tex

method, notes about q

I don'T think I do the kind of prismatic etc. balanced block design as per \cite[30][]{Brown1980} because I don't have ex-ante theories of what the pattern of understandings will be; I just hypothesize that they will change and straighten up.
As a result, an ANOVA is probably not appropriate (ibid.)

Or maybe I can do this, kinda like \cite[35]{Brown1980} on (Robert?) Reich:

also compare 60 in \cite[35]{Brown1980} that might be it

somewhat oddly, thompson refers to beliefs as ``biases'', as though inbiased facts were a possibility even though operant subjectivity can, by definition not be, well, objective and in operant subjectivity on economic matters in particular, is unlikely to arrive at such uncontentious statements.

summary graph on page 69 is very important.

note in a later conclusion that this should be deveoped a) into a broader standard for deliberative effects and b) into a broader study concerning moral development, cognition and thinking about the economy.

Notice that a lot of the ecologically occuring statements on tax are empty, false or non-sensical and thus should \emph{not} be in the sample.

Shalom Shwartz wrote a q-diss!
	Schwartz, S. H. 1978. Reflections on a Q dissertation and its opposition. Operant
Subjectivity. 1: 78-84.

add lord of the flies to reading list

sociology and q methodology?
q methodology bibliography
q method and econ

notice that as per Brown around 75, q methodology might also be used as a tool for deliberators to figure out exactly what kinds of policies they might agree upon!

Item ``taxation is the price we pay for a civilized society'' (o.W. holmes)


Main Effects		Levels
Consciousness 		a) Con I  	b) con II 	c) con III
Values  			d) personal 	e)lega-political 	f) socio-economic

Q-methodology and habermas! Must look into this!

___
McKeown

notice that factor loadings are actually *not* conventionally very important in Q; mostly factor scores. That'd be different for me; I'd wager the most general hypothesis that factor loadings would improve.

run interviews to gather the q concourse?

discuss the whole adapted vs. naturalistic controversy: in some instances, naturalistic items might work; but not for tax, because here some things are just *never* discussed, that's the very idea of this research.
(it is, in fact, questionable, whether the same problem might not also plague other research, as on love. Non-existent counterfactuals are always important – say, on love. You always have to be careful with hegomony in the language. Of course, the balanced-block design takes care of this already)
In my research, I'll make sure that items are readily comprehensible in lifeworld-terms, which, luckily *is* very possible for taxation, because it is about the oikos.
Of course, non-naturalistic, ``adapted'' items bring back the issue of external observation ,or rather, of things that the researcher has put in here. This problem cannot be entirely negated, but q methodology reduces it significantly.
For once, people can still sort items they do not understand or find odd in the middle, near 0, what Stephenson described as the quantum origin of the scale, with 0 information.
But also, no matter which other items were added to the q-sort, or on which other p-sample it would be replicated, the results of this q-sort would still be valid, because it's about operant behavior.

got to look at the dryzek political discourse q thing from 1993

apparently positive or negative can also be a line in the balanced block approach, as in \citep[23]{McKeown2013}

