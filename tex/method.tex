%!TEX root=../tax-democracy-held.tex

method, notes about q

I don'T think I do the kind of prismatic etc. balanced block design as per \cite[30][]{Brown1980} because I don't have ex-ante theories of what the pattern of understandings will be; I just hypothesize that they will change and straighten up.
As a result, an ANOVA is probably not appropriate (ibid.)

Or maybe I can do this, kinda like \cite[35]{Brown1980} on (Robert?) Reich:

also compare 60 in \cite[35]{Brown1980} that might be it

somewhat oddly, thompson refers to beliefs as ``biases'', as though inbiased facts were a possibility even though operant subjectivity can, by definition not be, well, objective and in operant subjectivity on economic matters in particular, is unlikely to arrive at such uncontentious statements.

summary graph on page 69 is very important.

note in a later conclusion that this should be deveoped a) into a broader standard for deliberative effects and b) into a broader study concerning moral development, cognition and thinking about the economy.

Notice that a lot of the ecologically occuring statements on tax are empty, false or non-sensical and thus should \emph{not} be in the sample.

Shalom Shwartz wrote a q-diss!
	Schwartz, S. H. 1978. Reflections on a Q dissertation and its opposition. Operant
Subjectivity. 1: 78-84.

add lord of the flies to reading list

sociology and q methodology?
q methodology bibliography
q method and econ

notice that as per Brown around 75, q methodology might also be used as a tool for deliberators to figure out exactly what kinds of policies they might agree upon!

Item ``taxation is the price we pay for a civilized society'' (o.W. holmes)


Main Effects		Levels
Consciousness 		a) Con I  	b) con II 	c) con III
Values  			d) personal 	e)lega-political 	f) socio-economic

Q-methodology and habermas! Must look into this!

___
McKeown

notice that factor loadings are actually *not* conventionally very important in Q; mostly factor scores. That'd be different for me; I'd wager the most general hypothesis that factor loadings would improve.

run interviews to gather the q concourse?

discuss the whole adapted vs. naturalistic controversy: in some instances, naturalistic items might work; but not for tax, because here some things are just *never* discussed, that's the very idea of this research.
(it is, in fact, questionable, whether the same problem might not also plague other research, as on love. Non-existent counterfactuals are always important – say, on love. You always have to be careful with hegomony in the language. Of course, the balanced-block design takes care of this already)
In my research, I'll make sure that items are readily comprehensible in lifeworld-terms, which, luckily *is* very possible for taxation, because it is about the oikos.
Of course, non-naturalistic, ``adapted'' items bring back the issue of external observation ,or rather, of things that the researcher has put in here. This problem cannot be entirely negated, but q methodology reduces it significantly.
For once, people can still sort items they do not understand or find odd in the middle, near 0, what Stephenson described as the quantum origin of the scale, with 0 information.
But also, no matter which other items were added to the q-sort, or on which other p-sample it would be replicated, the results of this q-sort would still be valid, because it's about operant behavior.

got to look at the dryzek political discourse q thing from 1993

apparently positive or negative can also be a line in the balanced block approach, as in \citep[23]{McKeown2013}

should I describe severeal conditions of instruction, like: what the actual tax system is like, what it should be like? (Maybe not, because that would be true/false)

In q and p sort, the criterion is saturation.

make people sort items on equity and effifiency? That might make some sense (even though my point is that both expand, which really won't work here.)

---

Niemeyer Diss
- check strong democracy barber

Look at dryzek and list 2001 on metaconsensus.

maybe in opposition to niemeyer, I think you can impose one normative theory – rawls, because it's meta.
just like intersubjective rationality.

Dryzek 1990 discourse theory – this seems important, it is what niemeyer is drawing on!

Intersubective rationality -- that people want the same preferences, because of the same values and beliefs –-- is kind of a strong assumption, because we don't strictly speaking know that such a consistency exists (except and in so far as it merely follows from vNM). But we can accept this as a regulative ideal; something to be strived for, not achieved. And if we never get any further to it, or can't, with that would wither the plausibility of deliberative democracy as an utopia, which, after all, depends on this.
Look into vNM to say that we have collective preferences might imply that we also have this kind of consistency.

get pelletier 1999 as example on q in deliberation

get dryzek 1993

discuss deliberative failure as in niemeyer diss6 69f

i should have some kind of stratified sample, too -- family, entrpreneur, worker etc.

tax attorneys, steuerberater?
bund der steuerzahler?
georgist?
pct?
other experts?

discuss how different tax is from the bloomfield track issue; there's just no immediate way to experience tax. It's all abstraction.
It is, in other worlds, inherently system, not lifeworld.

maybe visit the Finanzamt?
visit the tax business of a firm?

ask people which witnesses they thought were most interesting, influential.

don't just to experts, do practicioners, too.

the problem with practicioners is that there is currently no one out there who does these things, worries about them.

OECD tax people maybe?

my tax items will, obviously, still be real and part of concourse – they are just on the very periphery, which is exactly what I am looking at here.

niemeyer diss 158 compares q methodology to making a species survey of a small area in ecology; that's a good analogy why small numbers work. Also refer here to the brown example of calculating the species factor between arm, leg etc length.

remind people when they do the q-sort that they are not at all being tested; not the correctness, nor the consistency of their ranks (between the two points in time) is tested.

need to argue carefully why the actual policies (taxes) should be q-items, too
1) because they're difficult to understand, as opposed to the bloomfield options
2) because q measures what we'd need to know about them anyway.
3) they are many-dimensional, and their vary dimensionality (much like the other q-items) is of interest. They can't be graded on a n access to non-access dimension as bloomfield (214)

check again dryzek and list 2001 for preference structuration

---

\cite{Dryzek1993} is very important, completely a fan.

The most radical meta-standard that could be formulated would indeed be pure q methodology, with a focus on the loadings and numbers of factors, maybe the three-step of dryzek and niemeyer. That's a tall order, and I'm not sure it works, but that would be the ideal.

Maybe it is not an accident that Dryzek stressed so much the reconstructive merit of q (for democratic theory) and went on to be a formative voice for deliberative democracy.

\cite[52]{Dryzek1993} write: ``Toput it another way, our units of analysis, when it comes to gener- alization, are not individuals but discourses. And we soon get to a point where adding individuals to a study does not yield any new informationunless the extra individuals are truly different-another reason for stressing diversity in subject selection.''
In other words, q methodology works with small n, because it is concerned with the dimensionality, or structure of subjectivities, not the prevalence. Notably, q methodology is downward compatible; if you have a lot of people do it, you can become ever more confident in the relative factor *loadings* (which is what you'd be interested, then). Notably, this is not the case the other way around; if you administater an r survey with *lots* of items, it's still not same as q, because it lacks the same unit.

\cite{Niemeyer2013} is also about a local issue, and even shorter (the birdge thing)

also note that focusing only on the overall q loadings would be sort of against the theory; and conflict with judgemental rotation. Still that would be the general effect of which this might be a special case, if the list/dryzek piece has it right.

also notice that

---
notice that \cite{Brown-2006} does something quite dissimilar; the locals are right, always. I can't do that for tax.
But I should also point out that my research interest is quite circumspect.

--

notice that q methodology does sit a little uneasily with critical theory, including habermas, because it (initially) seems to permit *no* false consciousnesses.
This, of course, is because of the items in the q sort are not supposed to be factual questions, but one of preference (ideology being facts wrapped up in norms.) (uh-uh, this is bullshit)

my selections of q sorts is essentially statements that follow habermas' rules for communicative rationality; it would be non-sensical to ask the for the subejctivity of powered speech.
This might at first seem interesting (and in line with my research), but it violates the central tenet of q methodology, namely to take operant subjectivity serious and place it front and center.
the participant, in other words, is always right.
Also, ``testing'' for powered speech, would be hypothesis testing – which is not supposed to happen with q; I would, in effect, be interested in the movement of individual items.
Instead, I am looking at the dimensions.
Also, if there is such a thing as communicative rationality, it should reveal itself in neatly aligned factors, if not before, than after deliberation.
The same logic still holds; If, as predicted, the factor space clears up, something must have prevented it from clearing in the first place. That something is powered speech.

---

\cite{dryzek2005handle} This seems important, right down my alley for why I need q.

describes why in-depth, etc. are all bad; also how I would become a deliberator in an in-depth interview.

--
\cite{Durning1999} cite this guy; that's a pretty damning review at the beginning of positivism.
