%!TEX root=../tax-democracy-held.tex

\begin{longtabu}[]
%	{l
%	l
%	p{3cm}
%	p{2cm}
%	p{5cm}
%	r
%	r}
	{X[]
	X[]
	X[8]
	X[7]
	X[16]
	X[]
	X[]}

\caption{Commented Table of Contents\label{tab:commented-toc-phd}}\\

%firsthead
	\toprule

	\emph{\#}
	&\emph{}
	&\emph{Title}
	&\emph{Function}
	&\emph{Content}
	&\emph{p.}
	&\emph{\%}
	\\

	\midrule

\endfirsthead

%n-heads
	\toprule

	\emph{\#}
	&\emph{}
	&\emph{Title}
	&\emph{Function}
	&\emph{Content}
	&\emph{p.}
	&\emph{\%}
	\\

	\midrule

\endhead

\small

\emph{}
&	\ref{chap:introduction-phd}
&	\nameref{chap:introduction-phd}
&	Introduction
&	\emph{Lays out the structure of the dissertation.}

	%more text?
&	\pageref{chap:introduction-phd}
& 95
\\

\emph{}
&	\ref{chap:proposal-phd}
&	\nameref{chap:proposal-phd}
&	Proposal
&	\emph{Summarizes the planned research as a conventional proposal.}

	%more text?
&	\pageref{chap:introduction-phd}
& 95
\\

\midrule

\ref{part:puzzle}
& 	\emph
&	\nameref{part:puzzle}
&	Proposal
&	\emph{Explicates the research gap.}

	The current crises and performance of both established democracy and welfare states can only be understood, when compared with desirable and doable hypotheticals in democratic fora and tax.
&	\pageref{part:puzzle}
\\

\emph{}
& 	\ref{chap:wanted}
&	\nameref{chap:wanted}
&	Foundations
&	\emph{Lays out the structure, epistemology, ontology and axioms of the dissertation.}

	Social scientists must ask not just what \emph{is}, but what \emph{could} be (First Order Theory) and explain why it is not (Second Order Theory).
	Such hypotheticals must be desirable and doable.
	Desirable and doable are defined.
&	\pageref{chap:wanted}
& 95
\\


\emph{}
&	\ref{chap:mixed-economy}
&	\nameref{chap:mixed-economy}
&	Theoretical Background
&	\emph{Explains for which ends, and by which means market and plan can coexist.}

	Welfare states are properly understood as mixed economies, where government supplements market outcomes by coercive plan.
&	\pageref{chap:mixed-economy}
& 90
\\

\emph{}
&	\ref{chap:3-crises}
&	\nameref{chap:3-crises}
&	Empirical Background
&	\emph{References empirical and theoretical findings on the intertwined crises of democracy, the welfare state and equality.}

	Established democracies are constrained, gridlocked and confused.
Welfare states are unsustainable and/or defunct.
	Economic inequalities are widening.
&	\pageref{chap:3-crises}
& 65
\\


\emph{}
&	\ref{chap:tax-matters}
&	\nameref{chap:tax-matters}
&	Literature Review
&	\emph{Explains how tax matters to the mixed economy, welfare state and democracy.}

	Mixed economies rely on efficient and equitable taxation.
	Democratic government always faces trade-offs in designing welfare states.
	Voters must understand these alternatives, and all materially possible designs must be available for voters.
&	\pageref{chap:tax-matters}
& 80
\\

\emph{}
&	\ref{chap:hypotheticals-matter}
&	\nameref{chap:hypotheticals-matter}
&	Critique of the Literature
&	\emph{Without appreciating doable and desirable hypotheticals, social science becomes latently affirmative.}

This is especially true for welfare state research.
Absent a critical comparison to a hypothetical ideal mixed economy, the second-order account of welfare fails to explain most of the current shortcomings.

	%more text missing?
&\pageref{chap:hypotheticals-matter}
&60
\\

\emph{}
&	\ref{chap:testing-hypotheticals}
&	\nameref{chap:testing-hypotheticals}
&	Research Design
&	\emph{Describes the positive research design to make the hypotheticals empirically falsifiable.}

	Under a different (deliberative) democratic process, a random sample of ordinary voters will understand the mixed economy differently and prefer a different (progressive consumption, land-value) tax.
&	\pageref{chap:testing-hypotheticals}
&65
\\

\midrule

\ref{part:tax}
&	\emph{}
&	\nameref{part:tax}
&	Theory
&	\emph{Reviews mainstream economic thinking about desirable and doable taxation.}

	The first order theory of tax asks, and answers what the most desirable, but doable tax is.
&	\pageref{part:tax}
&
\\


\emph{}
&	\ref{chap:desirable-tax}
&	\nameref{chap:desirable-tax}
&	Theory (Review)
&	\emph{Reviews mainstream economic thinking about efficiency and equity of taxation.}

	A desirable tax offers one of the higher possible trade-offs between efficiency and equity.
&	\pageref{chap:desirable-tax}
& 80
\\


\emph{}
&	\ref{chap:doable-tax}
&	\nameref{chap:doable-tax}
&	Theory (Analysis)
&	\emph{Lists factual complications and contradictions of taxation.}

	A doable tax is personal, falls on inelastic bases, has a well-defined incidence, least affects liquidity choices and is agnostic towards the source of income.
&	\pageref{chap:doable-tax}
& 85
\\


\emph{}
&	\ref{chap:better-tax}
&	\nameref{chap:better-tax}
&	Theory (Synthesis)
&	\emph{Evaluates real and hypothetical taxes on the aforedeveloped criteria of desirability and doability.}
	A \glsfirst{pct} and \gls{lvt} emerge as by far the best, most desirable and doable taxes.
	Other, existing taxes violate efficiency, equity, or both and display manifold unintended consequences and unavoidable loopholes.
&	\pageref{chap:better-tax}
& 85
\\

\midrule

\ref{part:democracy}
&	\emph{}
&	\nameref{part:democracy}
&	Theory
&	\emph{Reviews normative, theoretical and empirical political science on democracy.}

	The first order theory of democracy asks, and answers what the most desirable, but doable form of democracy is.
&	\pageref{part:democracy}
&
\\

\emph{}
& 	\ref{chap:desirable-democracy}
&	\nameref{chap:desirable-democracy}
&	Theory (Review)
&	\emph{Briefly reviews normative political theory on democracy.}

	A desirable form of democracy provides effective participation, control of the agenda, voting equality and enlightened understanding \citep{Dahl-1989-aa}.
&	\pageref{chap:desirable-democracy}
& 10
\\

\emph{}
&	\ref{chap:doable-democracy}
&	\nameref{chap:doable-democracy}
&	Theory (Analysis)
&	\emph{Briefly reviews public choice theory and political psychology.}

	Some likely micro- and macropolitical dynamics divert real existing, formally democratic government from normative desiderata.
	Doable democracies face different, sometimes unattractive, trade-offs between participation, deliberation and political equality \citep{Fishkin2009}.
	In complex societies, the dysfunctions and trade-offs may be harsher, still.
&	\pageref{chap:doable-democracy}
& 10
\\


\emph{}
&	\ref{chap:better-democracy}
&	\nameref{chap:better-democracy}
&	Theory (Synthesis)
&	\emph{Briefly reiterates the deliberative theory of democracy, and describes prominent institutional designs.}

	Deliberative democracy promises to overcome many of the trade-offs and dysfunctions of pluralist democracy, by stressing intersubjective understanding and communicative action in lieu of pre-social preferences and power in speech.
&	\pageref{chap:better-democracy}
&10
\\

\midrule

\ref{part:tax-democracy}
& 	\emph{}
&	\nameref{part:tax-democracy}
& 	Theory (Original)
&	\emph{Develops a second order theory of democratic choice of tax.}

	Deliberative democracy is a good \emph{method} to test one explanation for the absence of a better tax.
	Deliberative democracy is also \emph{more} than a method:
	deliberation and \gls{pct}/\gls{lvt} are conceptually related.

	Tax is a good \emph{case} to test deliberative democracy.
	Tax is also \emph{more} than a case:
	it is \emph{the} social contract in capitalism.
&	\pageref{part:tax-democracy}
&10
\\


\emph{}
&	\ref{chap:no-better-tax}
&	\nameref{chap:no-better-tax}
& 	Theory (Original)
&	\emph{Reviews and develops alternative second order explanations for why we do not have a better tax.}

	We may not have a better tax, because progressive taxation is plagued by a global cooperation problem, because of path dependency, because of domestic political dysfunctions or, simply, because people do not want it.
	I try to falsify only the last explanation, and theorize how different democratic processes may alter, suppress or confuse the will of the sovereign.
&	\pageref{chap:no-better-tax}
&55
\\


\emph{}
&	\ref{chap:tax-under-pluralism}
&	\nameref{chap:tax-under-pluralism}
&	Theory (Hypotheses)
&	\emph{Hypothesizes falsifiable, popular misunderstandings of tax and the mixed economy that may divert voters away from the \gls{pct} and \gls{lvt}.}

	Voters incorrectly think
	\begin{inparaenum}
		\item that a mixed economy cannot have an arbitrary savings rate,
		\item that nominal variables reflect actual savings,
		\item that a mixed economy cannot have an arbitrary state-market mix,
		\item that non-natural persons can be taxed and
		\item fail to aggregate different taxes, indirect taxes and ``social contributions''
	\end{inparaenum}.%add hrefs to the above?
&	\pageref{chap:tax-under-pluralism}
&35
\\


\emph{}
&	\ref{chap:common-grounds}
&	\nameref{chap:common-grounds}
& 	Theory (Outlook)
&	\emph{Develops conceptual and theoretical linkages between the \gls{pct}, \gls{lvt} and deliberative democracy.}

	The \gls{pct}/\gls{lvt}  and deliberative democracy embody and enforce a similar standard of justice as fairness \citep{Rawls-1971}.
	The \gls{pct}/\gls{lvt} enable the kind of equality on which deliberative democracy relies.
	The \gls{pct}/\gls{lvt} and deliberative democracy both build, and require deep cooperation.
&	\pageref{chap:common-grounds}
&15
\\

\bottomrule
\end{longtabu}