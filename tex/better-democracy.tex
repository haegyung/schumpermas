%!TEX root=../tax-democracy-held.tex

\chapter[Better Democracy]{A Better Democracy} \label{chap:better-democracy} 

\begin{quote}
	\emph{``Parliament is not a congress of ambassadors from different and hostile interests; which interests each must maintain, as an agent and advocate, against other agents and advocates; but parliament is a deliberative assembly of one nation, with one interest, that of the whole; where, not local purposes, not local prejudices ought to guide, but the general good, resulting from the general reason of the whole. You choose a member indeed; but when you have chosen him, he is not a member of Bristol, but he is a member of parliament.''}\\*
	--- Edmund Burke (1729 -- 1797), 1774 Speech to the Electors of Bristol as cited in \citep{Burke1886}
\end{quote}


%ok, so somin is officially an asshole. He wants more Hayek. Because th commons can't be governed, let's have more market.



\section{Deliberative Theory}

%wikipedia: Deliberative democracy differs from traditional democratic theory in that authentic deliberation, not mere voting, is the primary source of legitimacy for the law.

%wikipedia: "Im Zentrum der Theorie der deliberativen Demokratie steht das Legitimationsideal der öffentlichen Beratung politischer Fragen. Der Kern dieser Theorie ist daher eine Kritik der liberalen Theorie der Demokratie, vor allem der rein aggregativen Sichtweise und der Übertragung des Marktmodells auf die Politik."

%wikipedia on Habermas re: deliberation
	%wikipedia: Diskurse vollziehen sich öffentlich bzw. in der Öffentlichkeit:
	%„Die Öffentlichkeit lässt sich am ehesten als ein Netzwerk für Kommunikation von Inhalten und Stellungnahmen, also von Meinungen beschreiben“ (Habermas 1992: 436). Öffentlichkeit ist also kein vorgefundener Raum, sondern muss durch ein interessiertes Publikum und durch kommunikativ handelnde Teilnehmer erst hergestellt werden. Öffentlichkeit besitzt bei Habermas drei Funktionen:
	%Erkennen und Wahrnehmen gesamtgesellschaftlicher Probleme
	%Thematisieren und Herantragen dieser Themen an die Entscheidungsträger im politischen Zentrum
	%Kontrolle des politischen Zentrums.
	%Die nichtstaatlichen wie nichtökonomischen Akteure der Zivilgesellschaft (oder: der „zivilgesellschaftlichen Öffentlichkeit“) als „(...) das Substrat jenes allgemeinen, aus der Privatsphäre gleichsam hervortretenden Publikums von Bürgern, die für ihre gesellschaftlichen Interessen und Erfahrungen öffentliche Interpretationen suchen und auf die institutionalisierte Meinungs- und Willensbildung Einfluß nehmen“ (Habermas 1992: 444) sollen diese Funktionen übernehmen (nicht näher thematisiert werden soll hier die sog. „vermachtete Öffentlichkeit“, in der sich etwa finanzstarke Lobbygruppen wiederfinden würden).
		%key point: you can have civil society, but only in the deliberation, that is the control of power.

%wikipedia on cohen
	%Joshua Cohen, a student of John Rawls, outlined conditions that he thinks constitute the root principles of the theory of deliberative democracy, in the article "Deliberation and Democratic Legitimacy" in the 1989 book The Good Polity. He outlines five main features of deliberative democracy, which include:
	%An ongoing independent association with expected continuation.
	%The citizens in the democracy structure their institutions such that deliberation is the deciding factor in the creation of the institutions and the institutions allow deliberation to continue.
	%A commitment to the respect of a pluralism of values and aims within the polity.
	%The citizens consider deliberative procedure as the source of legitimacy, and prefer the causal history of legitimation for each law to be transparent and easily traceable to the deliberative process.
	%Each member recognizes and respects other members' deliberative capacity.

%This can be construed as the idea that in the legislative process, we "owe" one another reasons for our proposals.

%wikipedia on cohen 
	%Cohen presents deliberative democracy as more than a theory of legitimacy, and forms a body of substantive rights around it based on achieving "ideal deliberation":
	%It is free in two ways:
	%The participants consider themselves bound solely by the results and preconditions of the deliberation. They are free from any authority of prior norms or requirements.
	%The participants suppose that they can act on the decision made; the deliberative process is a sufficient reason to comply with the decision reached.
	%Parties to deliberation are required to state reasons for their proposals, and proposals are accepted or rejected based on the reasons given, as the content of the very deliberation taking place.
	%Participants are equal in two ways:
	%Formal: anyone can put forth proposals, criticize, and support measures. There is no substantive hierarchy.
	%Substantive: The participants are not limited or bound by certain distributions of power, resources, or pre-existing norms. "The participants…do not regard themselves as bound by the existing system of rights, except insofar as that system establishes the framework of free deliberation among equals."
	%Deliberation aims at a rationally motivated consensus: it aims to find reasons acceptable to all who are committed to such a system of decision-making. When consensus or something near enough is not possible, majoritarian decision making is used.
	%In Democracy and Liberty, an essay published in 1998, Cohen reiterated many of these points, also emphasizing the concept of "reasonable pluralism" – the acceptance of different, incompatible worldviews and the importance of good faith deliberative efforts to ensure that as far as possible the holders of these views can live together on terms acceptable to all.[9

%maybe include Caplan's more market here?
%maybe include participatory democracy here?

%!TEX root=../thesis.tex

\begin{table}
	\caption{Pluralist and Deliberative Democracy}
	\label{tab:pluralist-vs-deliberative}
	\small
	\begin{center}
	\begin{tabular}{lcc}
		\toprule 
		 & \emph{Pluralist Democracy} & \emph{Deliberative Democracy}\\
		\midrule
		\emph{Knowledge} & Miracle of Aggregation & Schools of Democracy \\ [10pt]
		\emph{Legitimacy} & Special interest & (Alternative!) Common Good\emph{s} \\ [10pt]
		\emph{Equal Participation} & Representation & Veil of Ignorance \\ [10pt]
		\emph{Speech} & Power & Communicative Action \\ [10pt]
		\emph{Preferences} & Pre-social & Intersubjective\\
		\bottomrule
	\end{tabular}
	\end{center}
\end{table}

%note this odd thing: my justification of tax and deliberation is Rawlsian (liberal), but deliberative democracy is NOT liberal, and it often has communitarian overtones (positive rights) -- that is very odd. I need to get into that. maybe the solution is that communitarianism vs individualism are just extreme answers to an inherently contingent human nature: just whether we are individuals, or groups is what needs an institutionally mediated answer, it's not either or, it's depending on ... and maybe deliberation and tax do just that.

%Rosenberg's / Toqueville's school as democracy and vice versa is the operative metaphor, this is profound. Make a visualization about it?

%remember: deliberation is about generating preferences (48), electoral (pluralist?) democracy is about aggregating preferences
%   * also cite, get: "Why is deliberative democracy better than aggregative democracy?" (Guttman and Thompson 2004: 13)

%a dysfunctional, if hegemonic, political process.
%In their proposal for \Empowered Deliberative Democracy" Fung and
%Wright (2001: 17) explicitly suggest what underlies much of the current
%designs:
%1. A focus on specic, tangible problems
%2. Involvement of ordinary people aected by these problems and ocials close to them
%3. The deliberative development of solutions to these problems.
%(ibid., all emphases

%Steiner 2008: 190 says "it is better to keep the two models separate since they are based on diametrically opposed views of basic human nature"

%\begin{quote}
	%\emph{``Einmal in vier Jahren''}
	%--- Die Toten Hosen
%\end{quote}

%Sunstein1991: 4 "People are taken as they are, not as they might be."
	%I might want to point out that Sunstein is a more elaborate formulation of the endogeneity problem.

%Mouffe 1999: I might argue that if there's anything where she maybe right, it should be tax and the political economy. That's where power should be straightforward. It's also where I might show that it actually CAN work. Tax is also where there might indeed be some zero sum games; as she says on p 756 "pluralist politics should be envisaged as a mixed game, in part collaborative and in part conflictual and not as a wholly co-operative game as most liberal pluralists would have it.

	%Here's another idea: do aggregative (deliberative poll) first. Large-scale, but aggregative, no consensus. Then, for a post-doc, do again a consensus conference. The second step. You might even do it as a panel design. See what people did in the meantime.

%Ryfe 2005
	%Cite this guys for the logic why a random sample is better.

%There's a bit of a problem with Fishkins DP design; he doesn't actually want a coherent movement in one direction (no homogeniziation, no polarization). that's kind of a problem with my hypothesis for PCT-preference.

\begin{quote}
	\emph{''All power corrupts, but we need the electricity.''}
\end{quote}

%There are some things that I see different than Fishkin does, or I think my project would take a different perspective. His are usually pretty mundane topics, and pretty simple items that he tests in terms of knowledghe. The deliberated topics are topics that CANNOT be decided by expert (health care rationing, etc.).
%My topic (tax) is also debated by experts, indeed, experts seem to have the home advantage. But my point is: we need to check the work of the experts. ordinary citiznes need to "auf die Finger schauen". We'll see whether the DP or deliberative fora in general can actually do this.

%a dysfunctional, if hegemonic, political process.
%In their proposal for \Empowered Deliberative Democracy" Fung and
%Wright (2001: 17) explicitly suggest what underlies much of the current designs:
%1. A focus on specic, tangible problems
%2. Involvement of ordinary people aected by these problems and ocials close to them
%3. The deliberative development of solutions to these problems. (ibid., all emphases added)

%also, cite for empirical support for successful deliberation REykowski 2006.

%wolves quote on democracy
%complexity quote sphinx

%Also, consider that all the stuff on ideational research, is, in the end, a description of its dysfunctions. Deliberative democracy is a prescription to go beyond these disfunctions. IDeas become a dysfunction when they live on their own, when they no longer serve people, and when they are imbued with power and interest.

%Delli Carpini et al 320: People in conflict will set aside their adversarial, win-lose approach and understand that their fate is linked with the fate of the other, that although their social identities conflict they “are tied to each other in a common recognition of their interdependence” (Chambers 1996; Pearce&Littlejohn 1997; Yankelovich 1991).

%A Normatively Desirable, Hypothetical Political Process:
%Deliberative Democracy

%I need to attend a deliberation

%I have doubts about the deliberative poll. Can it be upgraded, enhanced to cover substantive justice, a classroom and Rosenberg?

%Three parts:
%   * cognition: insufficient and unequal
%   * emotion: empathy and caring about a oommon good can't be just cognitive
%      * it's about caring, feeling part of a community (14)
%   * communication and discourse
%      * is customarily (in deliberation) assumed to be mostly successful, and a neutral vehicle for transporting cubjectively constructed views back and forth


%Why do I need Deliberation?

%Do I have a macropolitics? Or just a micropolitics? (I think I only have the latter)

%still need to argue the tax case (why this is a good case for deliberation)

%at the end of the day, there needs to be an engineer, not just discourse

%deliberative democracy is the opposite to economic theories of democracy (says Loftager)

%Note in the trust and deliberation discussion Ostrom's conditions to solve CPR problems (communication, small-scale, trust, localized physical setting)

% I need a democratic theory questionreason for doing a deliberative poll
% develop desiderata into a questionnaire
% funding: plan A and plan B
% standards: Fishkin (R)
% cognitive ability / Rosenberg
% why tax, and why tax must be more than a case
% what's next: pluralism has problems, if there is a delta in attitudes,knowledge, BUILD theory, build critical hypotheses
% convincing link between tax and deliberation, THEORY. Assumed link: if you'd want to hide injustice, you'd best do it in a really complex system, like tax.
%matthijs says: reflect n the inability of politicians to do stuff. "Why is nothing done? is it maybe because there is something wrong with our political system?. Show the options that are on the table (take Fishkin's typology here!) Note the more radical ideas that were out there (Kirchhof)
%respond to matthijs: no, the PCT cannot be implemented nationally, just like any other progressive tax. still, I think this should be an enterprise in finding sth first-best.
%qualify Fishkin: I need more complexity in preparatory material, other role for expert
% look at cognitive limitations for deli beration (Rosenberg), maybe adopt a more expansive view of deliberative democracy
% project management
% fiscal soc. 
% do graph with overlapping disciplines: welfare state, fiscal soc, public finance, 
% why tax is a good case and deliberation a good method: huge mismatch (this is franzi)
% why tax is more than a case, dd more than a method: heuristics/biases in tax / 
% stephens: got to give them numbers. cant do this myself. maybe someone has?
%will need to read: habermas
% offe: write a 2nd order theory on the reform suggestion (such as offe 2009)
% people and I need to deconstruct, than reconstruct tax
% preference structuration is the word
% "we are not as divided, as our politics suggest"
% democracy is not being, it is becoming. It is easily lost, but never finally won.

%politics is like the fabled sphinx: it devours all those who cannot solve its riddles.

%local deliberative democracy: diffidence by design?
%the shortfalls of (some) deliberative designs: no congruency because of macro windmills. 'Bean 'n Rice Works"
% politicization!
% very important in both tax and deliberation: inequality.
%savings norms

%challenges on deliberation: sanders / talking stories, not just rational stuff. Consider groupthink, Noelle-Neumann

%think deeply about rational choice and deliberative democracy. this is in response to somin

%Barber (1984: 175 as cited in Sanders): ``The participatory process of self-legislation that characterizes strong democracy attempts to balance adversary politics by nourishing the mutualistic art of listening. ``I will listen'' means to the strong democrat not that I will scan my adversary's position for weaknesses and potential trade-offs, nor even (as a minimalist might think) that I will tolerantly permit him to say whatever he chooses. It means, rather, ``I will put myself in his place, I will try to understand,  I will strain to hear what makes us alike, I will listen for a common rhetoric evocative of a common purpose or a common good.'' ''

\section{Deliberative Models}
%There's a bit of a problem with Fishkins DP design; he doesn't actually want a coherent movement in one direction (no homogeniziation, no polarization). that's kind of a problem with my hypothesis for PCT-preference.

%There are some things that I see different than Fishkin does, or I think my project would take a different perspective. His are usually pretty mundane topics, and pretty simple items that he tests in terms of knowledghe. The deliberated topics are topics that CANNOT be decided by expert (health care rationing, etc.).

    % * standards (Fishkin covers that)

%The view subscribed to here is that of an engineer. You look at what you've got. You consider the dysfunctions you know about. You add an element (a moderator) that you suspect might help. This is the key difference to juries. Juries are not moderated. Deliberative Fora are. Maybe cite the Magnolia piece ("play as it leads".)
	%This is the key difference to juries. Juries are not moderated. Deliberative Fora are. 

%Also consider: groupthink, and other dysfunctions (Kuran, Noelle-Neumann), and how to counter this. It seems, what would be important would be GOOD moderators, a place that is curiously underinvestigated as someone wrote in the special issue.

%Sanders (emphasis added): ``democrats [\emph{need}] to listen as well as to talk in their deliberation''

%This design combines the electoral part of democracy with the talking part. (this is Chambers 2003: 308)

%Cite that Sanders makes a good point, pointing out the dysfunctions that juries display. It's a good, if somewhat outdated review. Will need to find fresher literature.
%Sanders 369: ``I should say that I am not entirely against deliberation. But I am against it for now: I think it is premature as a standard for American democrats, who are confronted with more immediate problems''.
%Sanders 370: ``The invitation to deliberate has strings attached. Deliberation is a request for a certain kind of talk: rational, contained, and oriented to a shared problem. Where antidemocrats have used the standards of expertise, moderation and communal orientation as a way to exclude average citizens from political decision-making, modern democrats seem to adopt these standards as guides for what democratic politics should be like. And the exclusionary connotations of these standards persist.''
%Sanders 372: ``What is fundamental about giving testimony is telling one's own story, not seeking communal dialoge. Although hooks refer to the development of a ``common literacy,'' this voice is common to a group that is usually excluded from the discourse of the dominant, and the voice that contributes to a ``common literacy'' is posed by hooks in opposition to, and as a criticism of, this dominant discourse. There's no assumption in testimony of finding a common aim, no expectation of a discussion oriented to the resolution of a community problem. Testimony is also radically egalitarian" the standard for whethera view is worthy of public attention is simply that everyone should have a voice, a chance to tell her story.'' 
		%This might be a problem. If the rich have a story to tell, too, we might end up with pluralism as we started.
		
		%Mutz (2008) is all about how it is important to make deliberative democracy falsifiable. I'm doing it, a little. (Note the link between falsifiability and substantive standards of justice. You can't do without it.)
%Also, my cop-out is: a LITTLE fairer will be fine. You can't make falsifiability into a latently affirmative business, if you really fall for SES creeping in (as Olaf always insisted it would). So you've gotta get some kind of balance here. Cite Offe (via Bosch) on his brilliant charicature on how a theory CAN innundate itself against criticism. It may be helpful to simply point out here that what holds for deliberation (is it falsifiable?, nah, maybe not so much with substantive standards) ALSO holds for pluralism (are these, in the economists language, ``revealed preferences'')? (revealed preferences works for the market, not sure whether that belongs in here)

%Deliberation is "still in large part a critical and oppositional idea" (Bohman and Regh 1998, p 422: as cited in Mutz 528). I totally agree.

%Thomson points out that deliberation was unlikely to ever fully negate inequality, but also (Thompson 2008):
%``Empirical research that simply reinforces the general conclusion that deliberation falls short of the standards of equality is therefore not very illuminating. Research that shows specifically what conditions and changes might mitigate inequality can be useful. Even more valuable, and less common, is research comparing the inequalities in deliberative forums with inequalities in other political settings. Because so much of democratic politics is pervaded by inequality, the more fundamental question is comparative: To what extent do deliberative forums satisfy various standards of equality more or less effectively than other political processes?''

%Also: ``[Deliberative] Theory challenges political reality. It is not supposed to accept as given the reality that political science purports to describe and explain. It is intended to be critical, not acquiestent".

%Also: ``Some researchers have assumed that they can dispose of deliberative theory by showing that political discussion often does not produce the benefits that theorists are presumed to claim for it. They extract from isolated passages in various theoretical writings a simplified statement about one or more benefits of deliberative democracy, compress it into a testable hypothesis, find or (more often) artificially create a site in which people talk about politics, and conclude that deliberation does not produce the benefits the theory promised and may even ...'' (tbc)

%So there are (also in Thompson) several reasons why consensus may not even be desirable
%1) deliberative democrats disgagree on whether they want it (Gutmann and Thompson 2004): ``Exposing and even intensifying disagreemnents may be desirable in many circumstances. ''
%2) ``it is difficult to distinguish consensus from compromise'' (Steiner et al 2004: 91-92, as cited in Thompson 2008)

\section[Autonomy and Equality]{Autonomy and Equality \\--- Schools for Democracy}

\begin{quote}
	\emph{Can the subaltern speak?}\\**
	--- Gayatri Chakrovorty \cite{Spivak-1988-aa}
\end{quote}

%Rosenberg 2007: Rosenberg has the radical inter-subjective formulation that I think I'm looking for:
	%6: ``In this deliberative conception, equality and autonomy require each other. On the one hand, equality is a necessary precondition of autonomy. It is only in a cooperative exchange between equals that the self-expression and critical selfreflection required for the self-reflective construction of one’s understandings and interests is possible.
	%10 Where the self dominates, self-criticism truncates and narrows. Where the other dominates, self-expression is suppressed. In either case, the self that is constructed is a distortion and any true autonomy is compromised. On the other hand, equality requires autonomy. Deliberative equality is equality of effective participation.11
	%10: Developmental psychology raises further questions regarding the adequacy of deliberative theorists’ characterization of cognition. In the deliberative democracy literature (and in most social psychological research), cognition is typically conceived as a universally shared set of distinct skills or capacities. Following the work of Piaget27 and Vygotsky,28 the research on cognitive development offers a more integrative and differentiated view of cognition. Cognition is not regarded simply as a matter of calculation or arranging representations, but more essentially as a constructive activity. This is understood in pragmatic terms, that is, with reference to an individual’s attempt to operate on the world around her. In this attempt, the individual conceives of the acts, items and people involved in term of the role each plays in her purposive activity
	%12: In this sense, the research suggests that deliberations be regarded as remedial institutions

%I have doubts about the deliberative poll. Can it be upgraded, enhanced to cover substantive justice, a classroom and Rosenberg?

%Deliberation (12) "should not be conceived sumply as settings for free exchange in which citizens capacities for reflecton and engagement can be realized. " "[t]he research suggests that deliberations be regarded as remedial institutions."

%13: "deliberations must be more than remedial, they must be sites for political education and development"

%I need to attend a deliberation.

%Three parts:
%   * cognition: insufficient and unequal
%   * emotion: empathy and caring about a oommon good can't be just cognitive
%      * it's about caring, feeling part of a community (14)
%   * communication and discourse
%      * is customarily (in deliberation) assumed to be mostly successful, and a neutral vehicle for transporting cubjectively constructed views back and forth

% THIS IS A GREAT QUOTE. This is what i want. 20: ``Deliberative fora must not be regarded simply as empty stages that provide a venue for the realization of citizenship; nor must the design of these deliberative stages focus simply on the removal of obstructions that may inhibit freedom or give unequal scope for maneuver. Instead, deliberation must be understood as a site for the construction and transformation of citizenship. In deliberation, citizens are made as well a realized. The operative metaphor here is that of a school, but of particular kind. The educational goal is not the transmission of specific beliefs and values, although these are by no means irrelevant. Rather the central aim must be to foster the requisite cognitive development for a fuller autonomy, a greater communicative competence and a better ability to engage in a collaborative effort to make good and just public policy''.

%Note the connection to ``schools for democracy'' (de Toqueville?)

%25: ``Here, care must be taken to recognize individual differences in cognitive and emotional development and to consider the different ways in which deliberations may have to be institutionalized to meet their individual needs. To facilitate deliberation in a way that fosters the desired complementarity of intersubjective engagement and the desired level of sympathetic emotional connection, freedoms may be abridged and inequalities introduced.''

%I need a teacher to work with me on this. and someone who does games to get people to feel good about each other. and someone who does small-group psych.

%25: ``the potential for abuse [by facilitators] is real, and crafting an appropriate conceptual and institutional response will be difficult'' 

%Rosenberg 2004: 12: 12: ``In my view, the value of the focus on deliberation is that it leads democratic theory beyond a consideration of individuals as essentially asocial agents that act simply to maximize their personal interests under conditions of collective action. However I do not believe that the deliberative democrats have gone far enough in their consideration of the social character of the individual, either as a subject or an agent. In this regard, they have a tendency to continue to characterize cognition as an essentially psychological attribute, one that reflects a basic and universal human nature. They also tend to reduce the emotional bond between people to a secondary question of private''

%Gutmann, A., & Thompson, D. F. (2002). Deliberative Democracy Beyond Process. The Journal of Political Philosophy, 10, 153-174.
	%Yes, I agree with these guys. It's not just about process.
	%There is some deep link between the nature of man and deliberation. I think they key is reciprocity, intersubjectivity and "social beings". We're deeply social, both in terms of our nature (psych), as well as in our politicel ecoomy.
	%Deliberation resolves, and requires this acceptance: it's about intersubjectivty. If you're a hyperliberal, you don't care about either of those.

%Here's some Goodin:
	%Overview of what's out there in terms of formats:
		%   * Deliberative Polls (US / Fishkin)
		%   * Citizen Juries (US, abroad)
		%   * Planning Cells (Germany)
		%   * Consensus Conferences (Denmark)
			%      * are, says ibid, typically run together with media outlets (!) who publish the poll results
			%      * I think I also want a written document.
		%   * AmericaSpeaks, 21st Century Town Hall Meetings
		%   * National Issues Forum

%Here are list of impacts that Deliberation may have on the public:
	%Impact may come in the form of actually making policy, being taken up in the policy process, informing public debates, market-testing of proposals, legitimation of public policies, building confidence and constituencies for policies, popular oversight, and resisting co-option.

%Follow up on Dahl, Footnote 4

% Baccaro 200
	%this is pretty minimal stuff.     
	%cite PDPA as small-scale deliberation
	%Habermas, on the other hand seems to argue that deliberators can't get into the business of government anyway.
	%"communicative action, is action oriented to reaching understanding" (Habermas 1996, Norms and Facts)
        
	%Social coordination through communicative action is possible, Habermas argues, in societies characterized by ‘strong archaic institutions’ and ‘small and relatively undifferentiated groups’ (Habermas, 1996, p. 25). This is because communicative action always takes place against the backdrop of a shared lifeworld, which in traditional societies provides ‘a reservoir of taken for granteds, of unshaken convictions that participants in communication draw upon in cooperative processes of interpretation’ and from which ‘processes of reaching understanding get shaped’ (Habermas, 1987, pp. 124 and 125, respectively).
        
	%However, coordination through communicative action becomes improbable in post-traditional societies, due to the emergence (strictly related to rationalization and modernity) of self-centred action based on strategic calculations. In these societies, ‘in which unfettered communicative action can neither unload nor seriously bear the burden of social integration falling to it’ (Habermas, 1996, p. 37), law emerges as a functional necessity exactly to ‘lighten the tasks of social integration for actors whose capacities for reaching understanding are overtaxed’ (Habermas, 1996, p. 34).
        
	%6It should be noted that some of the discourses which are produced in the informal public sphere conform poorly to the ‘gentlemen’s club’ model featured in some theories of deliberation (i.e. rational, poised and articulate), which, according to critics, ‘involves communication in the terms set by the powerful’ (Dryzek, 2000, p. 70), and resemble much more the ‘agonistic’, partisan discourses of a trial setting. Groups do not seek to persuade each other but seek to influence the court of public opinion by forcefully asserting the issues, values and interpretations that they believe should be binding for everybody, sometimes even through ‘sensational actions, mass protests, and incessant campaigning’ (Habermas, 1996, p. 381).
        
	%This characterization shares with critics of deliberation, in particular with Carl Schmitt’s Crisis of Parliamentary Democracy (1985), and Habermas’ own earlier work (for example, 1989), a stance of low expectation about the possibility of deliberation in formal settings, and it sees action aimed at reaching understanding as more likely to occur in the informal public sphere, where the preferences of ordinary citizens are still malleable, pragmatic constraints as to what is possible and feasible are less pressing, and the possibility for civil society groups to build communicative power by articulating moral alternatives to the
	
%Azmanova 2010
	%* 49: Jürgen Habermas has called “the unforced force of the better argument.” 2
	%   * get 2
	%   * This is a key concern: can deliberation overcome the crap already in people's heads?
	%   * 49: So the question is: How do we know that public deliberations are really free of power asymmetries, ideological idiom, and various forms of manipulation; that deliberative polling engages communicative, rather than instrumental rationality?
	%   * a good, fair argument is (This is citing Fishkin, as citied in ibid.: 50; "Reflexivity is attained thanks to five procedural conditions (‘good conditions’) of deliberation: 
	%      * (1) reasonably accurate information; 
	%      * (2) substantive balance; 
	%      * (3) diversity; 
	%      * (4) conscientiousness; 
	%      * (5) equal consideration. 
%      * These five conditions approximate the environment in which the deliberative polls are conducted to what Jürgen Habermas has described as ‘the ideal speech situation.’ 21
 %  * apparently I need to read Thomas Kuhn, as cited in ibi.
%   * I am more optimistic than Bourdieu (being determines thinking, das sein bestimmt das bewusstsein).
        
%Note for thesis: Fishkin's is a procedural only view of deliberative democracy; there's no justice component, no common good component, no assumptions about how we're supposed to interact. I think that's fine, it's just not enough.

%Freeman, S. 2000

	%Note for thesis: Fishkin's is a procedural only view of deliberative democracy; there's no justice component, no common good component, no assumptions about how we're supposed to interact. I think that's fine, it's just not enough.
	%Why not make this bigger?

	%This is great text for the political philosophy implications of deliberation.

	%The aggregative view of democracy is much like a marketplace; it's the economics of poliical science.

	%I agree with Caplan that it's actually worse than a market, because it doesn't have prices, but only preferences and bare majority rule. %cite hayek on this: central processing machine

	%Bingo 375: "In contrast to this and other aggregative conceptions, the ideal of de- liberative democracy says that in voting it is the role, perhaps the duty, of democratic citizens to express their impartial judgments of what conduces to the common good of all citizens, and not their personal pref- erences based on judgments of how measures affect their individual or group interests"

	%Bingo377: here's the disagreement with Fishkin: " No reasonable conception of democracy would advise us to vote without thinking at all about how alternatives might affect relevant in- terests (whether our personal interests or the common interest). This would be irrational. In this regard what distinguishes deliberative from aggregative conceptions is not deliberation per se, nor the fact that one involves voting but the other does not (contrast Przeworski, DD2). Nor is it even that the deliberative view involves discussion while the aggregative view does not. "

	%There's also a more expansive list of other rights that deliberative democracy requires, including some basic economic and educational opportunities (see 381)

	%I think there's really two groups of reasons for deliberation:
		  %1st lack of information (cooperation problem, this is Kaplan) (this is the "Millian argument" as cited in ibid. 383
		%2nd   * lack of cooperation in substantive terms (bare majority rule aint enough) (I find ibids definition on 383 too shorthand; it's not just about minority protection, it's about an awful lot more than that).
		%3rd   * legitimacy (ibid 383) because people understand when they loose out, why the lose out
		%   * 4th: make principled arguments, "extends peoples imagination" (Goodin as cited ibid), "engages moral sentiments" (Rawls as cited ibid). "tempers self-interest" (Mill as cited ibid.)
		%   * 5th: autonomy by enlightenment: (sunstein as cited in bibid 384 "Political autonomy can be found in collective self-determination, as citizens decide, not what they 'want,' but instead who they are-what their values are and what those values require. "21 

	%ibid 388 calls this the difference between procedural and epistemic views of democracy.Procedural views emphasize fairness of demo- cratic procedures, whereas epistemic views emphasize substantive jus- tice of outcomes. A pure procedural conception of democracy has no standard independent of the voting procedure itself to determine the justice of outcomes.36 A pure epistemic conception says that justice is entirely independent of procedures for deciding what is just, so that the procedure that best approximates substantive justice is itself right (le- gitimate or just). 

	%There's hope for disagreeing, not tolerating stuff: ok: 401: "Rawls's ideal of public reason then responds to the "fact of reasonable pluralism." Reasonable pluralism-not pluralism per se-defines the pa- rameters and the reach of public reason. This means three things (at least). first, Rawls's idea of public reason does not rely on autonomy or any other "comprehensive" account of the human good; nor does it depend on an account of the origin of moral duty (in reason, emotion, or God's wis- dom or will.) The idea that autonomy, God, utility, human perfection, or some other comprehensive value, is a condition of justice and the hu- man good must be given up within public reasoning, for these values cannot provide bases for agreement on a common good among demo- cratic citizens "

	%402: "Whereas expression of such unreasonable views is to be tolerated, their effects are to be contained by society. There is no requirement that they be respected or accommo- dated by public reason "

	%So here's the punch line (this needs to be argued very carefully): I think an economy, and in particular a democracy rooted in self-intersted are suboptimal and inequitable. Confer game theory. Argue this acrefully.

%---
%Mansbridge 2010
	%   * points out that deliberative polling mostly does not include radical solutions, perspectives (that could be a problem)
	%   * 55: t"he deliberators with radical left or right alternatives that arenot within the currently feasible political process. Including
	%such options is not practical in a context in which the funding and frame for Deliberative Polls and their like are provided by governments, the mainstream media, or mainstream foundations. But including such alternatives may be a desirable long-run goal. At the moment Deliberative Polls do as good a job as any of their alternatives in the quality of the options and materials they provide".

%---
%Fishkin
	%* 69: himself says that DP may be quite similar to a focus group, at least in terms of the moderator.

%---
%Barabas, J. (2004). How Deliberation Affects Policy Opinions. American Political Science Review, 98(04), 687-701. doi: 10.1017/S0003055404041425.
%   * here again are the sources for bad attitudes:
%      * Delli Karpini & Keeter 1996: They know little about American Politics
%      * Converse 1964: non-attitudes, unstable

%---
%Fishkin 2010: when the people speak
%   * There's a Trilemma:
	%      * Political Equality
	%      * Deliberation
	%      * Mass Participation
%      * ... you can only have two.
   %   * There are two fixes:
      %   * make it small scale (Mansbridge 2010: 59)
	%         * or make it Deliberation Day (Fishkin & Ackermann, as cited in: Mansbridge 2010: 59)
	
%---
%Farrar, C., Fishkin, J. S., Green, D. P., List, C., Luskin, R. C., & Levy Paluck, E. (2010). Disaggregating Deliberation’s Effects - An Experiment within a Deliberative Poll. British Journal of Political Science, 40(02), 333. doi: 10.1017/S0007123409990433.
	%   * (334) 3 hypothesis
	   %   * attitudes (=continuous dispositions towards policy alternatives) change (individua/gross and aggregate/net) 
%      * preferences (ordinal rankings of alternatives) become closer to single-peakedness (no cyclical majorities of the sort identified by Condorcet and Arrow)
   %   * both effects are stronger (interaction!) for less salient issues (tax? vs. Stuttgart 21)
%   * 335: the driving effect is learning. the people who learn the most change the most and are the most single-peaked
%   * now this article tests salience vs. deliberation (people are randomly assigned to 2 issues, salient/non-salient)
%   * single-peakedness means there is a Condorcet winner (an alternative that beats, or is tied with, all others in pairwise majority voting)
%   * people may become more single-peaked for different reasons: 
%      * adopt what (elites) other people think
%      * acquire a more shared, intersubjective understanding
%      * indepedently excogitate a natural ordering compelled by logic
 %  * i'm not happy about the scope of the deliberation; they seemed very micro, very confined. I would think that my setting (a mixed economy) is a lot broader and requires a lot less axioms.

%---
%Fishkin, J. S. (2010). Response to Critics of When the People Speak- The Deliberative Deficit and What To Do About It. Symposium A Quarterly Journal In Modern Foreign Literatures.
%   * interesting idea from Sanders (as cited in ibid. 68): we don't understand the role of the moderator. "What springs most immediately to mind are examples from schools, not politics ..."

%---
%Smith
%53:``At its heart, a deliberative polity promotes political dialogue aimed at mutual understanding, which ‘does not mean that people will agree, but rather that they are motivated to resolve conflicts by argument rather than other means’.17 Hence, what is fundamental to democratic dialogue is ‘deliberative’ as opposed to ‘strategic’ or ‘instrumental’ rationality.''

%---
%Fishkin, the Book:
%From the standpoint of some democratic theories these practices are entirely appropriate. They are just part of the terms of political competition between parties and between organized interests.3 But from the perspective outlined here-deliberative democracy-they detour democracy from the dual aspiration to realize political equality and deliberation. AndRead more at location 95   • Delete this highlight
	%Add a note
%A democracy in which we all had substantive information would seem to take too many meetings. Second,Read more at location 103   • Delete this highlight
%Add a note
%A democracy in which we all had substantive opinions would also seem to take too many meetings. Read more at location 109   • Delete this highlight
%Add a note

%Actually talking-and listening to others-across the boundaries of political disagreement would seem to take too much effort and too many (potentially unpleasant) meetings.' Perhaps,Read more at location 112   • Delete this highlight
%Add a note

%Second, public opinion in mass society may be open to manipulation because of the public's low information levels. If people have little background information, then foregrounding particular facts may be persuasive when people have no idea of the broader context. Clean coal advocates make a powerful case for the benefits of clean coal compared to dirty coal, but the mass public has little idea that clean coal is much dirtier than natural gas (as well as other alternatives like renewable energy). Selective invocation of true facts (such as that clean coal is cleaner than dirty coal) without a context where those facts can be compared to others (how clean coal compares to other energy alternatives) can allow advocates to manipulate opinion.10 Third, when people have little information they may easily fall prey to misinformation. Even when contrary information was in the public domain, assertions that Iraq was responsible for 9/11 apparently carried weight when it was shrouded in the protective glare of national security. Fourth, a strategy of manipulation that is probably more common than misinformation is strategically incomplete but misleading information. If one argument based on true but misleadingly incomplete information has high visibility through expensive advertising and the counter to it never gets an effective audience, then the public can be seriously misled. Fifth, another key strategy of manipulation is to "prime" one aspect of a policy, making that dimension so salient that it overwhelms other considerations. In effect, a candidate or policy advocate changes the terms of evaluation so that the issue on which his or her side does best becomes the one that is decisive.11 Read more at location 124   • Delete this highlight
%Note: these are the dysfunctions of liberal democracy; given its limitations. Compare, in particular the last one, with my piece on neo-Downsian voting. Edit

%By priming a dimension, whether crime or character or national security, the incident can be intentionally employed to change (or further emphasize) the terms of evaluation to the neglect of other issues.12 As campaigns (and outside actors) compete to reshape the playing field, the result is literally MAD or what might be termed mutually assured distraction. Read more at location 136   • Delete this highlight
%Add a note

%In our democratic experience thus far, the design (and possible reform) of democratic processes has confronted a recurring choice between institutions, on the one hand, that express what the public actually thinks but usually under debilitated conditions for it to think about the issues in question, as contrasted with institutions, on the other hand, that express more deliberative public opinion-what the public would think about an issue if it were to experience better conditions for thinking about it. TheRead more at location 261   • Delete this highlight
%Add a note

%The hard choice, in other words, is between debilitated but actual opinion, on the one hand, and deliberative but hypothetical opinion, on the other. OneRead more at location 264   • Delete this highlight
%Note: bingo! Edit

%The idea is that if a hypothetical situation is morally relevant, why not do a serious social science experiment-rather than merely engage in informal inference or armchair empiricism-to determine what the appropriate hypothetical situation might actually look like? And if that hypothetical situation is both discoverable and normatively relevant, why not then let the rest of the world know about it? Just as John Rawls's original position can be thought of as having a kind of recommending force, the hypothetical representation of public opinion identified by the Deliberative Poll also recommends to the rest of the population some conclusions that they ought to take seriously.57 TheyRead more at location 447   • Delete this highlight
%Note: bingo. Counterfactual + Rawls . Edit

%It may well be that if a deliberative design requires a consensus "verdict" as in a jury, then a combination of social pressure and bargaining may yield results that depart from the conscientious judgments of the deliberators. However,Read more at location 614   • Delete this highlight
%Note: interesting thought: secret vote and NO consensus serves to avoid pressures for bargaining and/or compromise. Or is this maybe a good thing? Edit

%Chart II.Read more at location 720   • Delete this highlight
%Note: Consider adding this chart, or visualizing it in some other way. Edit

%The idea of microcosmic deliberation is to take a relatively small, face-to-face group which everyone has an equal chance of being part of, and provide it with good conditions for deliberating on some policy or political issue. Citizens Juries, like Deliberative Polls use public opinion research methods to gather a sample to deliberate. But the Citizens Jury is more akin to a single discussion group in that the size is comparable to that of a modern jury-twelve or perhaps eighteen or twenty-four.34 A benefit of such a group is that it can continue to meet in a local community for an extended period, sometimes for several days or on successive weekends. The "jurors" hear testimony, call witnesses, ask for evidence, and at some point come up with recommendations to some local or governmental authority. The limitation of the process is that with such small numbers it is not possible to establish the statistical representativeness of the deliberating group. Citizens juries are too small for there to be a scientific basis for connecting their conclusions to the hypothetical informed opinion of an entire society, to what the country would decide if it were better informed-even though the results of Citizens juries are often represented in that way. However, the now extensive experience in both the United States and the United Kingdom with Citizens juries adds to our picture of citizen competence with complex policy issues-once citizens find themselves in a social context that supports deliberation.35 Read more at location 843   • Delete this highlight
%Note: if I can't get enough money for a deliberative poll, I'll do a "Citizen Jury" Edit

%Our other focus for how to fill out Quadrant II, microcosmic deliberation, is a very old practice, but one that has only recently revived. If it is to acquire credibility, it needs buttressing with systematic investigation. Social science can be employed to give credibility to the claim that a particular strategy of institutional design has been realized to give expression to deliberative democracy-to the combination of political equality and deliberation. The aspiration is to undertake a research agenda that credibly explores the conditions under which deliberation might be realized by ordinary citizens who constitute a credible microcosm. The tighter this connection, the more transparent it is; the more evidence there is that it has been achieved without distortion, the more force there is to the claim that we can accept a realization of what the people would think (Quadrant II), rather than what they actually do think when they are not thinking very much (Quadrant IV). Now we will turn to an overview of some initial efforts in this direction. Read more at location 1373   • Delete this highlight
%Note: so this is what Fishkin wants to do. Is this want I want to do? Maybe. PLUS tax. Edit

%First, if and when this combination is achieved, how inclusive is it? In what way can it represent all the relevant voices or perspectives in the population? Second, if and when this combination is achieved, how thoughtful is it? We need to look at specific indicators of deliberative quality to evaluate the process and ensure that the results really are driven by consideration of the merits of competing arguments and not distorted by some pattern of domination or group psychology. Third, if and when this combination is achieved, what effects does it have? What effects does it have on participants or on the broader public dialogue? Most importantly, can it be situated in the policy process or the public dialogue in such a way that it has some effect on policy? Fourth, under what social and political conditions can any of this be accomplished? EvenRead more at location 1381   • Delete this highlight
%Add a note

%Social science must form the basis for defending the inference that a given design is producing its conclusions through the normatively appropriate deliberative processes (questions of internal validity) and that it is in principle generalizable to the larger population (questions of external validity). TheRead more at location 1408   • Delete this highlight
%Note: these are the research goals. Edit

%The problem is that any microcosmic deliberation taking place in a modern developed society will be one in which there are significant social and economic inequalities in the conduct of ordinary life in the broader society. It seems difficult or impossible to "bracket" these inequalities-for participants to behave "as if" they do not exist.6 Indeed the problem goes deeper. The possibility of doing so is the challenge of the "autonomy of the political," namely, whether or not equality can hold sway in politics in a world in which inequality rules in economic and social relations. The viability and legitimacy of the liberal-democratic project may turn on the answer. HowRead more at location 1431   • Delete this highlight
%Note: bingo. This is indeed the huge question. It's great that Fishkin acknowledges the problem. We'll see what we can do about it. Edit

%But Young's point is that there are more subtle forms of exclusion that turn on manners of speaking and listening. Some people, even if formally included, may not have their voices, if they speak at all, taken seriously. They may give off cues that indicate they are not well informed or not worth listening to. Those who are accustomed to every advantage in the conduct of their everyday lives may be more assertive in pressing their views on others and less open to listening to those without similar advantages.' They may also be more accustomed to orderly forms of reason-giving argument that weigh with other participants. Or so the argument goes. TheRead more at location 1438   • Delete this highlight
%Add a note

%Changes in collective consistency. The literature on public choice, from the Marquis de Condorcet in the eighteenth century through William H. Riker, Kenneth Arrow, and modern practitioners confronts the problem that democracy can lead to cycles. In pairwise comparisons majorities can move from A to B to C and back to A again. When this is the case, agenda manipulations can arbitrarily determine the outcome. Any claims to a reasoned public will formation seem undermined. However, when preferences conform to an underlying dimension, say left-right as an example, then they are said to be "single peaked" and cycles are not possible.12 There has been considerable speculation that when participants deliberate together, the percentage of the participants who come to share the same single-peaked dimensions increases, making cycles less likely or virtuallyRead more at location 1479   • Delete this highlight
%Note: this is my stuff. Edit

%In addition, in the eight Texas projects on energy choices, the percentage willing to pay more on their monthly utility bills in order to provide wind power to the whole community rose by about thirty points, averaged over the eight projects. And the percentage willing to pay more on their monthly bills in order to provide conservation efforts for the community (demand-side management) also rose about thirty points. TheRead more at location 2013   • Delete this highlight
%Note: note that this is the kind of common-good argument that I was looking for. Bingo. Edit

%And introducing deliberation into the schools would have lasting benefits as a method for reaching a broader public.60 ChangesRead more at location 2034   • Delete this highlight
%Note: compare this to the democratic school via This American Life. Edit

%Second, there are some protections against corruption and capture. The Texas projects on energy choices were part of a regulatory process, Integrated Resource Planning, that affected hundreds of millions of dollars of investment. YetRead more at location 2230   • Delete this highlight
%Add a note

%A legacy of conflict or deep differences of identity may leave them inured to any appeals about a shared public good-precisely because their minds and hearts are closed to any shared future with the opposing community. ForRead more at location 2267   • Delete this highlight
%Note: this is beautiful language, and adequate language "their minds and hearts are closed to any shared future with the opposing community" Edit

%---
%---
%Gutman, the book:

%1) Its first and most important characteristic, then, is its reason-giving requirement. The reasons that deliberative democracy asks citizens and their representatives to give should appeal to principles that individuals who are trying to find fair terms of cooperation cannot reasonably reject. The reasons are neither merely procedural (“because the majority favors the war”) nor purely substantive (“because the war promotes the national interest or world peace”). They are reasons that should be accepted by free and equal persons seeking fair terms of cooperation.Read more at location 163   • Delete this highlight

%2)  second characteristic of deliberative democracy is that the reasons given in this process should be accessible to all the citizens to whom they are addressed. To justify imposing their will on you, your fellow citizens must give reasons that are comprehensible to you. If you seek to impose your will on them, you owe them no less. This form of reciprocity means that the reasons must be public in two senses. First, the deliberation itself must take place in public, not merely in the privacy of one’s mind. In this respect deliberative democracy stands in contrast to Rousseau’s conception of democracy, in which individuals reflect on their own on what is right for the society as a whole, and then come to the assembly and vote in accordance with the general will.2 The other sense in which the reasons must be public concerns their content. A deliberative justification does not even get started if those to whom it is addressed cannot understand its essential content. It would not be acceptable, for example, to appeal only to the authority of revelation, whether divine or secular in nature.Read more at location 177   • Delete this highlight
%Add a note

%3) The third characteristic of deliberative democracy is that its process aims at producing a decision that is binding for some period of time. In this respect the deliberative process is not like a talk show or an academic seminar. The participants do not argue for argument’s sake; they do not argue even for truth’s own sake (although the truthfulness of their arguments is a deliberative virtue because it is a necessary aim in justifying their decision). They intend their discussion to influence a decision the government will make, or a process that will affect how future decisions areRead more at location 199   • Delete this highlight

%4) This continuation of debate illustrates the fourth characteristic of deliberative democracy—its process is dynamic. Although deliberation aims at a justifiable decision, it does not presuppose that the decision at hand will in fact be justified, let alone that a justification today will suffice for the indefinite future. It keeps open the possibility of a continuing dialogue, one in which citizens can criticize previous decisions and move ahead on the basis of that criticism. Although a decision must stand for some period of time, it is provisional in the sense that it must be open to challenge at some point in the future. This characteristic of deliberative democracy is neglected even by most of its proponents.Read more at location 212   • Delete this highlight
%Note: and maybe it should be rejected. it risks infinite regress it does not allow for substantive standards to criticize deliberations. in effect everything becomes deliberative. there is no explicitly non liberal pluralist morm of justice. Edit

%Practicing the economy of moral disagreement promotes the value of mutual respect (which is at the core of deliberative democracy). By economizing on their disagreements, citizens and their representatives can continue to work together to find common ground, if not on the policies that produced the disagreement, then on related policies about which they stand a greater chance of finding agreement.Read more at location 231   • Delete this highlight
%Note: ok so youre supposed to agree where you can. Edit

%But here the point to keep in mind is that the democratic element in deliberative democracy should turn not on how purely procedural the conception is but on how fully inclusive the process is. While deliberation is now happily married to democracy—and Habermas deserves much of the credit for making the match—the bond that holds the partners together is not pure proceduralism. What makes deliberative democracy democratic is an expansive definition of who is included in the process of deliberation—an inclusive answer to the questions of who has the right (and effective opportunity) to deliberate or choose the deliberators, and to whom do the deliberators owe theirRead more at location 274   • Delete this highlight
%Note: yes yes yes. true. its not just about deliberation but nclusiveness. but what kind of inclusiveness? its not just quant its also quality. also how is this different from enlightened self interest? and what if shawn is right that not everyone can do that? Edit

%theories.12 First-order theories seek to resolve moral disagreement by demonstrating that alternative theories and principles should be rejected. The aim of each is to be the lone theory capable of resolving moral disagreement. The most familiar theories of justice—utilitarianism, libertarianism, liberal egalitarianism, communitarianism—are first-order theories in this sense.Read more at location 336   • Delete this highlight
%Note: i think deliberation needs a first order theory too Edit

%In contrast, deliberative democracy is best understood as a second-order theory. Second-order theories are about other theories in the sense that they provide ways of dealing with the claims of conflicting first-order theories.Read more at location 340   • Delete this highlight
%Add a note
%ways of dealing with the claims of conflicting first-order theories. TheyRead more at location 342
%Note: the link betweem deliberative theory and fist order theory is the following: inclusivenessis hardly a given. to compensate this you need a theory of justice as well as schools for democracy. Edit

%electoral process is modeled on the analogy of the market. Like producers, politicians and parties formulate their positions and devise their strategies in response to the demands of voters who, like consumers, express their preferences by choosing among competing products (the candidates and their parties).Read more at location 361   • Delete this highlight
%Note: yes sadly. criticize this with downs. Edit

%The second aggregative method gives less deference to the votes and opinions of citizens: officials take note of the expressed preferences but put them through an analytic filter—such as cost-benefit analysis—which is intended to produce optimal outcomes. In some versions of this process, preferences based on misinformation or faulty heuristics can be corrected, and sets of preferences that produce irrational results (such as cyclical majorities) can be modified. This method originates in classical utilitarianism and owes its contemporary pedigree to welfare economics.Read more at location 365   • Delete this highlight
%Note: aka social choice the first mechanism is aka ppluralism Edit

%What these methods have in common—and what defines aggregative conceptions—is that they take the expressed preferences as the privileged or primary material for democratic decision-making. Preferences as such do not need to be justified, and aggregative conceptions pay little or no attention to the reasons that citizens or their representatives give or fail to give.Read more at location 373   • Delete this highlight
%Add a note

%The Commission might have reverted to the first method recommended by aggregative democrats—conducting a survey or referendum and taking the results as final. But the Commission realized that public opinion on this complex set of issues was inchoate, and would depend on how the questions were phrased.Read more at location 418   • Delete this highlight
%Note: same with tax. this is about oregon health care ranking Edit

%Deliberative theorists who favor a more substantive conception deny that procedural principles are sufficient. They point out that procedures (such as majority rule) can produce unjust outcomes (such as discrimination against minorities). Unjust outcomes, they assume, should not be justifiable on any adequate democratic theory.Read more at location 528   • Delete this highlight
%Add a note
%assume, should not be justifiable on any adequate democratic theory. ARead more at location 531
%Note: heres a nice idea. shawn shows that deliberation without substantive justice doesnt work becase its beyond the human ability at least today under inequality. the social mind and sharpe schwartz show that we can do more. hence we need and can do substantive deliberaion. i think frank does something similar for tax. and so does mccaffery. this could be a nice structure for my diss. Edit

%Michael Walzer, a critic who raises this kind of objection, asks deliberative democrats: “Is this our utopia—a world where political conflict, class struggle, and ethnic and religious differences are all replaced by pure deliberation?”Read more at location 1086   • Delete this highlight
%Add a note

%Mutual justification means not merely offering reasons to other people, or even offering reasons that they happen to accept (for example, because they are in a weak bargaining position). It means providing reasons that constitute a justification for imposing binding laws on them. What reasons count as such a justification is inescapably a substantive question. Merely formal standards for mutual justification—such as a requirement that the maxims implied by laws be generalizable—are not sufficient. If the maxim happens to be “maximize self-or group-interest, ” generalizing it does not ensure that justification is mutual. Something similar could be said about all other conceivable candidates for formal standards. Mutual justification requires reference to substantive values. We can see more clearly why mutual justification cannot proceed without relying on substantive values by imagining any set of reasons that would deny persons basic opportunities, such as equal suffrage and essential health care. Even if the reasons satisfied formal standards, they could not constitute a mutual justification because those deprived of the basic opportunities could reasonably reject them. Denying some persons suffrage is a procedural deprivation that is inconsistent with reciprocity: we cannot justify coercive laws to persons who had no share in making them. Similarly, denying persons essential health care is a substantive deprivation that cannot be justified to the individuals who need it. That such denials are unacceptable shows that the mutual justification is neither purely formal nor purely procedural.Read more at location 1823   • Delete this highlight
%Add a note

%Although reciprocity is a foundational value in deliberative democracy, it does not play the same role that first principles, such as utility or liberty, play in theories such as utilitarianism or libertarianism.Read more at location 1846   • Delete this highlight
%Add a note

%An important implication of reciprocity is that democratic deliberation—the process of mutual reason-giving—is not equivalent to the hypothetical justifications proposed by some social contract theories. Such justifications may constitute part of the moral reasoning to which some citizens appeal, but the reasoning must survive the test of actual deliberation if it is to ground laws that actually bind all citizens. Moreover, deliberation should take place not only in the private homes of citizens or the studies of philosophers but in public political forums. In this respect, deliberative theory proposes a political ideal that is process-dependent, even if its content is not exclusively process-oriented.Read more at location 1854   • Delete this highlight
%Add a note

%Reciprocity is to justice in political ethics what replication is to truth in scientific ethics. A finding of truth in science requires replicability, which calls for public demonstration. A finding of justice in political ethics requires reciprocity, which calls for public deliberation. Deliberation is not sufficient to establish justice, but deliberation at some point in history is necessary.Read more at location 1870   • Delete this highlight
%Add a note

%NICE recommend the new taxane drugs for chemotherapy, which do not cure but, as one MP put it, “can add years to life at a cost of about £10, 000 per year”? If NICE recommends against prescribing expensive new drugs that can provide some health-care benefits to patients, will it thereby be shielding the Government from pressure to increase the total NHS budget? If NICE recommends in favor of the NHS prescribing these drugs, will it thereby be forcing NHS not to fund some other existing and highly valuable treatments (or pressuring the Government to increase funding for the NHS)?Read more at location 2078   • Delete this highlight
%Add a note
%(or pressuring the Government to increase funding for the NHS)? TheRead more at location 2082
%Note: here is an interesting qestion as to what happens wen you leaveout the macro. this nice businss seems to be mostly about drugs. interestingly theres no mention of the market pricing of drugs which are often natural monopolies. Edit

%Reciprocity sets four standards or criteria to assess decision-making about health care: the justifications that decision-makers give should consist of reasons that are accessible, moral, respectful, and revisable.Read more at location 2518   • Delete this highlight
%Add a note

%(That is why cases limited to a single HMO cannot tell the whole story about social justice with respect to medical care.) Deliberation in a different forum may be necessary to address the problem. The best alternative to deliberation within HMOs may be publicly accountable deliberation in Congress, which has the power to set the legal parameters within which all HMOs must operate.Read more at location 2592   • Delete this highlight
%Add a note

%First, the reasons that decision-makers give should be accessible.Read more at location 2608   • Delete this highlight
%Note: do this for tax Edit

%Moral Reasons Reciprocity demands more than accessible reasons. Self-interested reasons—or reasons that serve the interests of one’s employer—are among the most conspicuously accessible.Read more at location 2656   • Delete this highlight

%The criterion of mutually respectful reasoning helps distinguish a reciprocal perspective from another kind of moral perspective, which bases itself on the criterion of impartiality. Reciprocity stands between prudence, which demands less from justifications, and impartiality, which demands more. Impartiality insists that reasons be impersonal. It requires citizens to suppress their own personal perspectives and partial projects when setting social policies and procedures. The prime example of an impartialist approach is utilitarianism. In practice, it favors expert decision-making and implies that the medical professionals in DesertHealth should have the final say as long as their judgment is consistent with general professional opinion. The preferred impartialist method is neither bargaining nor deliberation, but demonstration, which aims as far as possible to establish the truth of a comprehensive moral view. In the face of moral disagreement, impartiality tells citizens and officials that they should affirm the view most consistent with the true morality as determined by impersonal justification. There is no further moral need for mutual respect or for actual political deliberation. The trouble with impartiality is that it does not take moral disagreement seriously enough. More precisely, it fails to provide a satisfactory way to deal with the moral disagreements that inevitably remain on many issues when expert opinion on the technical and medical problems or the demonstration by a comprehensive moral philosophy such as utilitarianism are complete. In the face of a fundamental moral disagreement such as funding fetal-tissue research, expensive organ transplants, or experimental diagnostic tests such as PUREPAP, the impartialist approach can declare only one side (or no side) correct. If one side is correct, it provides no reasons for recognizing moral value on the other side. It therefore offers no way, other than agreement, for the other side to respect the decision on moral grounds.Read more at location 2742   • Delete this highlight
%Add a note

%“[O]nly love has the power to forgive, ” Hannah Arendt wrote. And, “Love, by its very nature, is unworldly, and it is for this reason rather than its rarity that it is not only apolitical but antipolitical.”39 But while reciprocity does not aim at encouraging love among citizens, it does aim at developing some degree of respect among citizens. “[W]hat love is in its own, narrowly circumscribed sphere, ” Arendt concluded, “respect is in the larger domain of human affairs.” Respect is less personal than love. It does not require intimacy or closeness, or even an admiration of a person’s achievements or particular qualities. Respect is a civic acknowledgment: the recognition that others are our fellow citizens and that we are willing to treat them as such, as long as they demonstrate a willingness to reciprocate.Read more at location 3284   • Delete this highlight
%Add a note

%---
%Fishkin, J. S., & Luskin, R. C. (2005). Experimenting with a Democratic Ideal: Deliberative Polling and Public Opinion. Acta Politica, 40(3), 284-298. doi: 10.1057/palgrave.ap.5500121.

%   * 285 political equality: "equal consideration of everyone's preferences, where 'everyone' refers to some relevant population or demos, and 'equal consideration' means a process of equal counting so that everyone has the same 'voting power' (an equal chance of being the decisive voter, when the voters are described anonymously, without reference to past voting patterns or current preferences)
%      * aka a random sample! (!)
%   * 285 deliberation is " 'weighing', which could be collective, individual, or both - involving discussion, rumination, or both. "
%      * informed
%      * balanced
%      * conscientious
%      * substantive
%      * comprehensive
%   * disagrees with Cohen 1997, Gutmann and Thomson 2003, NO particular style or quality of thought, much less the acceptance of any given premise. These are stricly procedural (!).
%   * (not sure I like that)
%   * 286: cites Gallup's (1939) hopes that the media plus surveys would do the trick to restore New England townhall-style deliberation; but it didn't. Maybe the media failed us here? 
%   * my idea:
%      * there are two more dimensions of democracy
%         * feasibility / can it be done, immediately, by the deliberative forum?
%         * potent polity / determination of the agenda, et al
%      * read up on jury stuff
%      * fishkin: This ain't a jury. There's no verdict, no consensus required. Also, there's a secret vote.
%      * I'm not sooo sure I agree with this. Maybe we'd have to link deliberative democracy more closely to legislation. What about putting it up for a referendum?
%   * ibid 2pdf, Fishkin seems to prefer deliberation that is "attainable" - but why?, 286: "not behind a hypothetical veil of ignorance", but in the actual world of politics and policy.
%      * I'm not happy about this.
%   * bingo, I think 294: ‘Thought experiments’ imagining what people would decide under morallyrelevant hypothetical conditions have become a staple of contemporary political theory (Fishkin, 1992). But why not move beyond armchair empiricism? If a hypothetical situation is morally relevant, why not do a real social science experiment to see what the appropriate hypothetical might actually look like? And if that hypothetical is both discoverable and normatively relevant, why not then let the rest of the world know about it?

%   * awesomeness, 294: deliberatives poll combine the best of experiments (internal validity) and surveys (external validity)
%   * follow up on:
%      * Cohen 1997
%      * Gutmann and Thomposn 2003\

%---

%Levinson, S. (2010). Democracy and the Extended Republic - Reflections on the Fishkinian Project. The Good Society, 19(1), 63-67.
%   * reviewing Fishkin: When the People Speak
%   * 66: There are, Fishkin notes, “normally strong disincentives for mass public opinion to be very deliberative.” 11 These can range simply from what economists would call the “search costs” involved in becoming suitably informed about public issues to the collective-action problems, also much emphasized by economists, that promote “free-riding” by most of the public on the relatively small number of individuals who are willing to invest their scarce time and energy (and money) into the demands of genuine republican citizenship. Some use these economistic insights to discredit the very project of popular democracy. 12
%   * Trilemma
%      * Political equality
%      * Mass participation
%      * deliberation
%Fishkin is no utopian. Indeed, a very important part of his argument is that one must choose among the horns of his trilemma. There is, therefore, the sad realization that “the fundamental principles of democracy do not add up to such a single, coherent ideal to be appropriated, step by step. … Achieving political equality and participation leads to a thin, plebiscitary democracy in which deliberation is undermined. Achieving political equality and deliberation leaves out mass participation. Achieving deliberation and participation can be achieved for those unequally motivated and interested, but violates political equality.” 13

%uh-oh, fishking may be to careful?

%66: "Rather, the question is whether, in effect, Fishkin should embrace a more robust version of “deliberative assemblies” as at least a complement and perhaps, in certain contexts, a replacement for representative democracy that relies on elections as a mechanism of selecting decision makers. And, if he did embrace such a version (and vision), would he find himself the target of hostility by those who might give him a hearing if his announced aims are significantly more modest?"\

%in a similar vein, 66f: "One of my favorite political dicta is John Roche’s emendation of Lord Acton’s famous, adage: “Power corrupts, and the possibility of losing power corrupts absolutely.” To the\

%Get in touch with this guys. He is great. He wants subversive Fishkinians.

%Comment in my piece that I agree with this guy; we need to argue and engage how this project can engage, complement or replace pluralist/representative democracy. I don't know how to do this yet, and that will stay beyond what is possible in my project.
%BUT: tax is a pretty thorny issue. It'll ruffle some feathers. Maybe it's so much at the heart of the representative apparatus that you get to see some ot this stuff.

%---

%Benhabib

%BINGO: 71 Furthermore, much political theory under the influence of economic modelsof reasoning in particular proceeds from a methodological fiction: this is .  the methodological fiction of an individual with an ordered set of coherent preferences. This fiction does not have much relevance in the political world.On complex social and political issues, more often than not, individuals may have views and wishes but no ordered set of preferences, since the latter would imply that they would be enlightened not only about the preferences but about the consequences and relative merits of each of their preferred choices in advance. It is actually the deliberative process itself that is likely to produce such an outcome by leading the individual to further critical reflection on his already held views and opinions; it is incoherent to assume that individuals can start a process of public deliberation with a level of conceptual clarity about their choices and preferences that can actually result only from a successful process of deliberation. Likewise, the formation of coherent preferences cannot precede deliberation; it can only succeed it. Very often individuals' wishes as well as views and opinions conflict with one another. In the course of deliberation and the exchange of views with others, individuals become more aware of such conflicts and feel compelled to undertake a coherent ordering.

%---
%Fung 2003

%There are big tectonic plates moving, that are causing the dysfunctions of democracy (agreed). 339.

%339: minipublics "look like, because they are, exercises in “reformist tinkering” rather than “revolutionary reform.”5

%This requires an account of how democracy used to work (I take this from the one really old study in the Roosevelt election, Carpini?: at that point, interests and voting aligned. Questions where, essentially, pretty simple. And maybe even zero-sum.
%We should be grateful that that is no longer the case.
%Tax is a case in point; there used to be a clear-cut case in tax, when capital was a class in itself (savings norms!), and when unions were strong wage bargainer (incidence of corporate tax!).

%Note that this is a good reason for tax as a case: It's about as large-scale as it gets. (like monetary policy, maybe, or trade?). Also, the mismatch between public discourse and the actual abstractions governing tax is hardly ever as colossal as here (or so I hope) ( Make a pun about this; that I don't hope other policy areas are affected by similar problems).

%have an outing: there's lots of stuff that I'd have to do for the large-scale deliberation that isn't actually PhD work (think invitations, catering, media work). So there's good reason to do this externally.

%This really is a key question: how do you scale up deliberation? Can you? should you? 365:
%How to make sure it's not beans-n'-rice democracy?

%---
%Sausgruber, R., & Tyran, J.-R. (2011). Are we taxing ourselves? - How deliberation and experience shape voting on taxes. Journal of Public Economics, 95(1-2), 164-176. Elsevier B.V. doi: 10.1016/j.jpubeco.2010.10.002.
%Cite Sausgruber and Tyran, to show that people may need some instruction before they deliberate away.
%In their research, JUST deliberation actually makes matters worse.

%Feature Table in Elster page 7

%Feature great Burke quote from Elster, it's on EverNote

%Elster says this: the better analogy for politics may be engineering, not science. It's not so much ultimate truth, but somethign that'll do.

%Elster: "the importance of time in political life implies that, in addition to deliberation, voting as well as bargaining inevitably has some part to play"

%Tilly has this (somewhat obscure) argument that tax and democracy belong together; without tax compliance there can be no support.

%---
%Marcus, G. E., Neuman, W. R., & MacKuen, M. (2000). Affective intelligence and political judgment. Book. Chicago, IL: University of Chicago Press.
%1: we care about politics "when our emotions tell us to"

%2: "to idealize rational choice and to vilify the affective domain is two misunderstand how the brain works"

%2: affect and reason are complementary.

%4: turn the power of Willie Horton (a racialized crime incident that was exploitet for a conservative campaign), the black arts into the white arts of politics (this is me, no quote)

%Dalton 2004 on growing voter disenchantment. This is part of the puzzle.

%a dysfunctional, if hegemonic, political process.
%In their proposal for \Empowered Deliberative Democracy" Fung and Wright (2001: 17) explicitly suggest what underlies much of the current designs:
	%1. A focus on specic, tangible problems
	% 2. Involvement of ordinary people aected by these problems and ocials close to them
	%3. The deliberative development of solutions to these problems.	(ibid., all emphases added)

%this is exactly my point: deliberative democracy is about generating preferences. "Income tax" is a ready-made preference, that is then merely aggregated. To better understand, you need to deconstruct this stuff.

%for an example of empirical work on deliberation, cite Farrar et al 2003: single-peakedness and structuration increases after deliberation. Tax is exactly the example that this, in a pluralist debate, is not the case.

%---
%also, cite for empirical support for successful deliberation REykowski 2006.

%=======

%Old Offe class text (add footnote to all of this)
%Deliberative Democracy - Diffidence By Design?\\ Red Flags and Opportunities

%\section{The Problem: \\Democratic Institutions Are Underperforming}

%William H. Hastie (1904-1076), the first African American Justice on the U.S. Supreme Court warned us that ``Democracy is not being, it is becoming. It is easily lost, but never finally won''.

%And so, indeed, the global proliferation and sustaining of popular rule in the West notwithstanding, our democracies are today faced with new challenges \citep{Huntington-1991-aa}.

%Value research points to increasing disaffection of voters, not with democracy as a system of rule, but with the institutions and performance of real-existing, sustained democracies \citep{Dalton-2004-aa}
%\footnote{
	%Dalton (ibid.) has pointed out that voter disaffection may in fact be regarded as democratically desirable, if we hold it to be indicators of critical citizen oversight of governance. This assumption --- that changing evaluation criteria may be for the better of democracy --- has some merit, but does not warrant inaction. Except to a very cold-hearted, and epistemologically narrow mind of rational choice, dwindling voter turnout (ibid.) must be alarming everyone that, if this ``snake [does not] shed its skin, [it may well] perish.'' (Antoine de Saint-Exup\'ery, The Little Prince). In any case, the material deficiencies and inequalities of our world point to political systems that are not beyond improvement.}.

%Explanations for voter disaffection and perceived underperformance vary, invoking amongst others ``government overload'' \citep{HuntingtonInternational-Affairs.-1968-aa}, ``post-material'' or ``emancipative'' change and dimensions of values \citep{InglehartWelzel-2005-aa}, mind-boggling ``reflexive modernization'' \citep{BeckGiddens-1994-aa} or a democratic deficit inducing ``denationalization'' {Zurn-2000-aa}. 

%I employ here \citeauthor{PutnamPharr-2000-aa}'s \citeyearpar{PutnamPharr-2000-aa} model to explain decreasing citizen confidence in government and political institutions. As relevant ``boundary conditions'', they list increased \emph{information} about the flaws of the political process ``post-Watergate'', and heightened \emph{evaluation criteria} due to either, or both, rising and diverging expectations. Citizens then assess the performance of representative democratic institutions, which is in turn a product of the \emph{capacity and competence}, the \emph{social capital} and \emph{fidelity} of the governors. The capacity and competence refers to the \emph{ability} of governments to choose and to implement policies in the interest of their citizens, highlights a routinized decision-making process as such and points to a systemic level of analysis. Fidelity is an individual-level determinant of government performance, focusing on the incidental misconduct. Social capital affects performance as a ``civic lubricant'' as an infrastructure, or network of generalized trust between individuals.

%\section{The Proposal: Inclusive, Egalitarian, Rationality-based Deliberative Democracy}

%\paragraph{How Deliberative Democracy is a Solution} Deliberative designs are pitted to tackle underperforming political systems from all of the above-mentioned angles. On the performance side, turning away from aggregative, or even bargaining decision-making to democratic rule based on reason (rather than power), is held to advance the sophistication of decisions (capacity/competence) as well as to strengthen an orientation towards the common good \citep{Cohen-1989-aa} and egalitarian, \citeauthor{Rawls-1971}ian \citeyearpar{Rawls-1971} justice (social capital, fidelity). On the side of the citizens, deliberation, by virtue of its inclusivity further increases information while at the same time putatively improving their evaluation criteria through reasoned discussion.

%Deliberative democracy can be argued to \emph{respond} to a crisis of democracy, but at the same time \emph{transcends} the epistemology of its diagnosis. It entails two paradigmatic shifts: for once, preferences are no longer held to be essentially \emph{pre-social} dispositions of a utility-maximizing individual, but subject to and product of reasoned deliberation on the common good. An evaluation by citizens of government performance, thereby becomes somewhat superfluous. Secondly, deliberative democracy is a project in \emph{direct democracy}, dissolving the very representative divide between the governed and the governors, fundamental to the diagnoses of crisis.

%\section{The Game: \\Setting the Scene of the Political Process}

%Political decision-making is here understood always as a response to a collective action problem of some sort, that is, the otherwise under-provision of a public good, in the broadest possible sense\footnote{The (neo)liberal undertones of this conceptualization of politics of that which the market does not take care of are, if reluctantly so, acknowledged.}. Where uncoordinated cooperation through exchange (markets) fails and hierarchical coordination (guardianship, see further \citealt{Dahl-1989-aa}) is illegitimate, people have to agree on a process by which they will reach the necessary \emph{collectively binding} decisions.

%By constructing the political process from its outcomes, it is --- axiomatically, really --- implied here, not that the political process would in any way be overloaded \citep{CrozierHuntington-1975-aa}, but that it is dramatically under-resourced to address the challenges postindustrial society faces. This shift in focus on \emph{the ability} of collective decision making bodies is based on an idealist, utopian hypothetical of a potent polity which can choose and implement \emph{any} collective action, unless otherwise constrained by individual rights. This stands in stark contrast to the more widely, if hardly explicated, weary notion of a greatly constrained community that, if at all, can collectively improve only at the margin, and rather then addressing, outsources hopes for progress to the market, devolves them to local governance or rejects them as ``input overload'' (ibid.).

%\subparagraph{Capacity} By capacity, I understand the \emph{ability} of democratic rule \emph{to act} in the interest of the people, that is, in the last instance, the ability to enforce collectively binding decisions. This reflects \citeauthor{Zurn-2000-aa}'s (\citeyear{Zurn-2000-aa}: 188) notion of congruency between the social and the political spaces, where inputs and outputs of both realms must always occur at the same level. In simpler words: an input incongruence occurs (for example, disenfranchised illiterates), where people are affected by a political decision (or non-decision) of their own polity, in the making of which they had no say and an output incongruence (for example, international tax competition) is given when a polity is affected by a dynamic, over which it has no jurisdiction.

%\subparagraph{Competence} By competence, I understand the ability to \emph{reason} on and \emph{identify} \emph{policies} that are in the interest of the people, and, conversely, those, that are not --- and why.

%The distinction between capacity and competence --- in reality as in the following discussion --- is not as clear-cut as here stipulated. Where capacities are constrained, even a competent discussion is likely to anticipate which decisions are impossible to collectively enforce, and thereby retract on other ``second-best'' solutions. Conversely, a capacity to organize a public good hardly makes sense abstracted from, or in the absence of a competent understanding of its very possibility. The distinction therefore serves rather to structure the argument from different, albeit not entirely orthogonal angles.

%\section{The Pitfalls: \\Deliberative Democracy's Red Flags}

%\paragraph{Framing the Question}In the following I provide a discussion of the capacity and competence inputs to the performance of deliberative democratic institutions, leaving aside the empowering (information and evaluation criteria of the citizens) and civic (social capital, fidelity) dynamics, which I hypothesize to be much more unambiguously positively. I expand on \citeauthor{PutnamPharr-2000-aa}'s \citeyearpar{PutnamPharr-2000-aa} understanding of capacity and competence. 

%\subsection{Deliberative Capacity: Are we Missing the Elephant in the Room?}

%\begin{quote}
%	\emph{If they can get you asking the wrong questions, they don't have to worry about the answers.}\\
%	Thomas Pynchon
%\end{quote}

%Surveying the existing proposals for and empirical accounts of deliberative designs, one cannot help but notice that the overwhelming majority of them are small in scale, both in terms of geography as well as issues discussed. This may simply lie in the incremental progressing of its innovation. 

%In the following, I argue that the small scale of currently existing deliberations is not merely epiphenomenal, but essentially grounded in a flawed discourse, that it is consequential for deliberative prospects, and can make for a dysfunctional, if hegemonic, political process.

%In their proposal for ``Empowered Deliberative Democracy'' \citeauthor{FungWright-2001-aa} (\citeyear{FungWright-2001-aa}: 17) explicitly suggest what underlies much of the current designs:
%\begin{quote}
%\begin{enumerate}
%\item A focus on \emph{specific, tangible} problems
%\item Involvement of \emph{ordinary people} affected by these problems and officials \emph{close} to them
%\item The deliberative development of solutions to \emph{these} problems.
%\end{enumerate}
%(ibid., all emphases added)
%\end{quote}

%In a similar vein, even the ``bigger'' deliberative designs of Planning Cells and Consensus Conferences are believed to best suited for issues that are publicly significant and \emph{relevant to the lives of lay citizens} (\citealt{HendricksonTucker-2005-aa}: 84, emphasis added).

%What are consequences of this ``diffidence by design''?

%\subsubsection{Small is Inadequate}

%Before discussing concrete examples, a positivist ``outing'' of modernity may be in order: In the following analyses it is assumed that, whenever functional differentiation gives rise to a sophisticated division of labor and accumulation of capital (as is the case on all example societies), the antecedent complex, and ever far-reaching web of social and economic interdependencies will affect the provision of public goods, too.

%\paragraph{Chicago Devolution}The almost microscopic devolution in the Chicago Police Department (CPD) or the Chicago Public Schools (CPS) are insightful examples plagued by crass incongruences of political, social and economic spaces. While not explicitly deliberative formats, both innovations sought to improve inclusion, dialogue, and equality \citep{Fung-2003-aa}. 

% The revamped CPD sought to cooperate with residents \emph{on a neighborhood}-level to identify and to fight --- not prevent --- crime herds. By its very focusing on police ridings, it made larger-scale, macro determinants of crime invisible, and outside the scope of debate: income inequality, or zoning laws, to name just two of the factors otherwise so glaringly discernible in Chicago, itself almost the epitome of a decaying inner city\footnote{Both of these factors, incidentally, would not even need a truly macro perspective to become potentially visible, as they are already visible on a metropolitan level.}. 

%In Public Schooling, Chicago reinvigorated \emph{school-based} Parent-Teacher Associations, re-trained and coached staff and teachers and installed systematic performance management ``by comparing their methods with those of \emph{similarly situated but better-performing schools}'' (\citealt{Fung-2003-aa}: 124, emphasis added). No mentioning is made of comparisons with surrounding suburban Illinois or urban charter schools, which may have enabled an understanding of the structural factors influencing a school's success or may have exposed the gross differences in scholastic aptitude with students ``not similarly situated''. Or, one hypothesizes, such comprehensive analyses may even put the effect of school management in a humbling juxtaposition against other, more powerful macro-level factors.

%There is no reason to belittle the efforts of these brave \emph{Don Quixotes}, working and participating to improve their surroundings in the schools and police stations of Chicago. Many of them have demonstrated substantial results \citep{Fung-2003-aa}. But given the ``macro windmills'' grossly outmatching these deliberators --- windmills over which, under the current design they have no control or even mandated discourse --- they are destined to fail in their democratic decision-making. Congruency between inputs, that is, what deliberators can discuss and decide on, and the outputs, that is, the social and economic reality to which they respond, is violated. Even basic democratic criteria of accountability and responsibility for political decisions and outcomes (like the ``within-slum'' performance management) have no meaning in this context.

%\paragraph{Participatory Budgeting} Much lauded Participatory Budgeting (PB), pioneered in Porto Allegre, Brasil, is another example for the intricacies and dangers of deliberative devolution from a transition economy setting. Under PB, local community and organization representatives cooperate with city government in deliberative fora, both at the neighborhood level and in thematic groups. Using a sophisticated rating and decision procedure they allocate much of the city governments investments. Observer \citet{Sousa-Santos-1998-aa} notes how PB contributes to a ''counterhegemonic globalization'' by concentrating on special issues like land rights, urban infrastructure and drinking water in urban settings. He, like few other writers on deliberative designs and PB in particular, is aware of the resulting danger that PB projects remain ``Bean 'n Rice Works'', ``formulae for solving a few of the urgent problems affecting the popular classes, perhaps a less clientelist version than the traditionalist one, but no less imediatist and electoralist'' (\citealt{Sousa-Santos-1998-aa}: 479). Also, he points to the threats posed to PB by the 1998 financial crisis in Brazil, and wonders whether PB will still be possible ``when the material results turn out to be less anchored in the immediate needs of the regions'' (ibid.: 495) and how distributive justice will still be possible under increased functional differentiation

%\paragraph{Identity} Another way to look at the inadequacy of small-scale deliberation is to question the identity formation that undergirds it. As is often cited self-confidently, devolved deliberation is praised to be ``carved out of (...) the realm of neighborhood and locale''(\citealt{Boggs-1997-aa}: 751). Just some etymological intuition --- neighbor\emph{hood}, \emph{community} (living together), \emph{the local} people --- suggests a bias towards a smaller scale, identities of which, for some reason seem to be more valuable than others. 

%Of course, a ``neighborhood'', for the most part --- except actual acquaintances --- is no more or less of an ``Imagined Community'' \citep{Anderson-1983-aa}, no less constructed and contingent \citep{Gellner-1983-aa} than a Nation State, or ``The International Community'', for that matter. ``Mechanical Solidarity'' \citep{Durkheim-1893-aa}, the only kind that would justify the immediate, the local, has through multifaceted integration and functional differentiation long given way to ``Organic'' --- if any --- Solidarity, where interdependencies run far. And where modern technology has brought great mobility, locality, is, as Simmel \citeyear{Simmel-1917-aa} said (of boundaries) ``not a spatial fact with sociological effects, but a sociological fact which forms spatially". There is then no particular reason to focus on local policy making --- but highly problematic consequences, as is argued in the following.

%\paragraph{Macro-Level Abstractions} As the majority of theorists write, deliberation depends on a rational orientation towards the common good. To argue well, and with the public good in mind then, necessitates a good understanding of causal relationships ruling our social world. 

%By its very definition, under modernity, these causal relationships are complex and the interdependencies run far. Good deliberation thus \emph{has to} transcend the immediate, the tangible and the local and to render accessible all the complex connections that functional differentiation binds.

%The described, excessively immediatist micro-focus is inadequate under modernity. It disallows deliberators both to understand, and to change all but a tiny portion of what determines their circumstances. 

%An output incongruency arises.


%\subsubsection{Small is Uncritical}

%\paragraph{Freedom From vs. Freedom To} It was, and still is, the promise of deliberative democracy to transcend the old trade-off between freedom \emph{from} and freedom \emph{to}. Sadly, a deliberative democracy that does not render sufficiently accessible macro-level abstractions is likely to sustain the bias towards negative freedoms of liberal democracy, which \begin{quote} ``may make it very difficult to mobilize citizens to accomplish what they collectively value most positively, but these rights and powers do insure them from suffering from what they collectively value most negatively, i.e. tyranny --- whether exercised by a zealous minority or an intolerant majority'' (\citealt{Schmitter-2000-aa}: 28) \end{quote}. 
%Two things are important to note about deliberative democracy and a possibly sustained liberal bias. 

%First, The micro-macro dichotomy and and the rift between negative and positive freedoms are related: negative freedoms are protected by \emph{individual} rights, which by their very definition assume, and guarantee a degree of independence from broader, macro-level interrelations. The right to property, to ``pursue happiness'', or even --- less obviously liberal --- to form a political association (\emph{or not to}), all protect the individual from a possibly imposing collective good (or evil)\footnote{The anti-totalitarian, humanist and protective impetus of liberalism are not to be belittled here. Enshrined negative freedoms are a great achievement, particularly in a country like Germany, where a frenzied, putatively common good has at least once brought untold sorrow. The lessons of this experience notwithstanding, it is held here, that the needs have shifted in post-industrial society: negative freedoms, for the most part (possible exception of privacy) is largely uncontested, and antecedent values and institutions are strong and deeply rooted. As modernization progressed, however, positive freedoms have not grown equally. Much of the postindustrial world sees rising income inequalities, and even more dramatically, decreasing social mobility.}. Positive freedoms, on the other hand, require \emph{collective} organization to reach desirable outcomes. This collective organization needs to be ``congruent'' over the social, economic and political sphere. In micro-level deliberations, as I have tried to show, this is easily not the case, tilting the balance towards negative freedoms.

%Secondly, and more critically, the liberal trade-off between positive and negative freedoms has important equity implications. The balance between positive and negative freedoms is highly consequential, for they are not equally valuable to everyone. To put it blandly, rich people, who have much to protect and little to gain from affirmative and redistributive policy, value negative freedom more and are skeptical of positive freedom. While poor people certainly hold most negative freedoms equally dearly as the rich, some of them (pursuit of happiness) may not have much practical meaning under dire circumstances and others (right to property) may be opposed to their re-distributional needs.

%The above-mentioned devolution in the Chicago Police Department is a intriguing example of promoting negative freedom in at least two ways. First, it renders the distributional determinants of crime (decaying, impoverished inner cities) invisible. Secondly, it focuses our attention and decision-making on crime, more specifically, an law enforcement, not prevention --- quite the essence of promoting negative freedom.

%\paragraph{No Spectres of Historical Materialism, But Politicization} This appraisal of positive and negative freedoms need not, and should not, lead us to \citet{Marx-1867-aa}. Essentializing economic status to bring about any ``[groupist] class for itself, rather than in itself'' \citep{MarxEngels-1848-aa} is clearly opposed to the deliberative spirit. But truly fair deliberation has to open up \emph{all} the large-scale interrelations (Marxist vulgo: structure) of modern society for debate, including the omnipresent implications of late international consumer capitalism. It has to overcome power, as much as it has to \emph{politicize} matters. In \citeauthor{Bluhdorn-2007-aa} (\citeyear{Bluhdorn-2007-aa}: 313) words:
%\begin{quote}
%``Politicization is the realization that established social norms, social practices, and social relations are contingent rather than sacrosanct, that things could also be different and that citizens, individual and collectively, have political agency by means of which alternatives can be explored and implemented. This recognition that things could also be different has always been the igniting spark of the emancipatory-progressive movements, and politicization has always been their key strategy.''
%\end{quote}

%Wherever large-scale social relations are not understood, and opened up for deliberation, of course, no agency to change can be assumed. In a way, the still emancipatory-progressive movement of deliberation risks to neglect much of what \emph{could} be changed, over what \emph{can} be changed more easily, and locally.

%Where reason fails to include macro-level abstractions and enable comprehensive politization, it cedes decision-making ground back to power --- a game, in which in turn, the already powerful and rich will likely prevail. 

%To be serious about equality then requires us to be very ambitious about reason. Otherwise, an input-incongruency may arise.


%\subsubsection{Small is Anti-Statist}

%\paragraph{Do-It-\emph{Yourself} Democracy} As a corollary of its liberal, somewhat uncritical bias, small-scale deliberation also tends to be anti-statist. Taking up some of the anti-authoritarian, anti-bureaucratic, anti-functional-differentiation impetus of post-'68 left counter-culture, progressive and new social movements, current deliberative efforts tend to focus on the immediate, tangible and ``confined to the realm of neighborhood and locale'', not just out of necessity, but out of conviction (\citealt{Boggs-1997-aa}: 759). 

%The very essence of Weberian (\citeyear{Weber-1958-ab} rationalized authority to enable collective action in the face of highly integrated economic production, embodied --- at least for the time being --- mostly in the state, thus comes under attack. This cherishing of ``human-scale democracy''over all else, the donning of a large, necessarily anonymous and formalized bureaucracy, Boggs (ibid.) rightly fears, may move democracy ``in a defensive and insular direction, laying bare a process of conservative retreat beneath a facile rhetoric of grassroots activism''. 

%Perversely, the anti-authoritarian adage ``Keine Macht F\"ur Niemand'' ('no power for anyone', \emph{Ton Steine Scherben}, 1972), applied to deliberation in the modern world --- at least in  the absence of anarchist revolution --- acts to \emph{conserve}, to sustain the status quo.

%\paragraph{The Elephant in the Room} Again, the old focus on ``autonomy'' in small-scale deliberation becomes visibly consequential only when looked at through the prism of social inequality: who stands to gain, who to loose from this focus? Clearly, like under a liberal basis --- essentially the same paradigmatic dichotomy --- the rich will rely less on a strong Weberian (\citeyear{Weber-1918-aa}) state, and the poor, while they may not understand it or reject its anonymous and sometimes authoritarian interfaces, rely on the state's exclusive capacity for affirmative action and redistribution. 

%Boggs (\citeyear{Boggs-1997-aa}: 752) goes one step further and even conspires that this revolt against the state may have hegemonically ``strategic value in a period of global interdependence and worsening social crisis''.

%\paragraph{Deliberation --- Successfully Failing?} Eliasoph's \citeyear{Eliasoph-2001-aa} flamingly polemic account of how an exclusive concentration on the do-able, the small-scale, the consequentially less political, creates dynamics of an outright ``Culture of Political Avoidance'' and escapism is enlightening. 

%A more disinterested, and systematic explanation of this dynamic comes from the theory of \emph{Successfully Failing Organizations} where, when faced with excessively costly challenges, organizations may choose ineffective, but symbolic policies the failure of which cannot, and will not be systematically detected. Such examination and oversight of agents (governors), it is argued, is not desired even by principals (the governed), precisely because its likely revelation of failure would necessitate a greater allocation (vulgo: redistribution) of resources \citep{Seibel-1996-aa}. Failure then becomes successful in that it sustains the status quo, and those who have material stakes therein.


%\subsection{Deliberative Competence: Is it Doable?\\Words of Caution From Political Psychology}

%\begin{quote}
%	\emph{The best argument against democracy is a five-minute conversation with the average voter.}\\
%	Winston Churchill
%\end{quote}

%Churchill's quip is surely a conservative, non-egalitarian exemplification of representative democracy, and thereby not surprisingly at odds with deliberative projects. Still, proponents of deliberation haven reason to worry about citizens' competence. Recent research in political and cognitive psychology suggests, that it may not be so unproblematic to assume Churchill's skepticism away, particularly from an egalitarian perspective.

%That people likely and often do not think perfectly rational is nothing new: people have been found to frequently use ``quick and dirty'' heuristics, rather than complete cost-benefit analyses (\citealp{CohenMarch-1972-aa, Simon-1999-aa}, to name just a few). 

%\paragraph{The Not-So-Common Sense} Recent research in political psychology suggests, that --- contrary to the bounded rationality assumption --- imperfect human reasoning may not only stem from remediable cognitive scarcity, but may be developmentally determined. Rosenberg (\citeyear{Rosenberg-2007-aa, Rosenberg-2002-aa, RosenbergWinterstein-2008-aa, Rosenberg-2003-aa}) has suggested a threefold developmental sequence and typology of \emph{sequential}, \emph{linear} and \emph{systematic} reasoning. His empirical accounts suggest that if any, only systematic thinkers will be able to meet the cognitive demands for reasoned arguments, and egalitarian free speech of deliberative democracy. Moreover, this cognitive competence was found not to be domain specific, and while people may regress to lower levels of competence under high loads or appropriate cues, they are unlikely to easily, if ever, achieve higher than developed levels. ``Structurally (more and) less developed reasoning adults'' make then ``not only the adequacy of citizens' reasoning, but also their equality'' a problem for deliberative democracy (\citealt{Rosenberg-2007-aa}: 12).

%This empirical finding resonates well with some anecdotal frustrations in deliberative settings, for example in participatory health administration:
%\begin{quote}
%``The tendency of citizens to construct their arguments in a way that is regarded as unstructured, combined with their focus on highly localized issues, makes their speeches appear unclear, emotional, disruptive or irrelevant to most representatives of the other sectors. Moreover, this style of speech tends to be associated with poorer and less educated people, and it is regarded as not only ineffective, but also virtually unintelligible''.
%\end{quote}

%The conclusion Rosenberg draws from this is as skeptical, as it is ambitious. He suggests that ``deliberations must be more than remedial, they must be sites for political education and development'' (\citeyear{Rosenberg-2007-aa}: 13). If they fail, so will quality of reasoning in deliberation and equality.

%\section{The Critical, Feminist Response: \\The World Ain't a Positivist Machine}

%\begin{quote}
%	\emph{Das Ganze ist das Unwahre}\\
%	\emph{(The Whole is Falsehood)}\\
%	Theodor W. \citet{Adorno-1974-aa}, Minima Moralia
%\end{quote}

%Deliberative proposals in general, by virtue of their focus on reasoned arguments, have been charged to embody and essentially impoverished, positivist and thereby ``male'' outlook on the world, particularly from feminist critics \citep{Young-1996-aa}. The above discussion, expanding demands for a universal rationality and outright promoting macro-level abstractions is subject to the same criticism.

%A full-blown discussion of the epistemological justification of feminist theory, or relativism, is beyond the scope of this essay and unfortunately beyond my command of the relevant literature. A response to this criticism thus has to remain regrettably sketchy.

%\paragraph{The Blinders of Rationality} Its uneasily essentializing, groupist outlook notwithstanding (see further \citealt{Brubaker-2002-aa}), the feminist critique rightfully points to a number of blind spots of deliberation, and, by extension, blind spots of modern rationality. To widen the horizon again, Young (\citeyear{Young-1996-aa}: 130) suggests to relativize rationality, and to allow for rhetoric, storytelling and ``greetings'' by which she means a ``care-taking, deferential, polite acknowledgement of the Otherness of other''. 

%Also, she fears, universal rationality sustains hegemonic discourses, and limits our social imagination, as for example the nearly unchallenged postindustrial necessity of individual car mobility (\citealt{Young-2001-aa}: 687).

%\paragraph{Enlightening the Blind Spots} These blind spots, a modern rationality that takes due regard of the human condition, \emph{can} illuminate, \emph{enlighten}. There is nothing quintessentially disregarding of individual experience in modern rationality. Correctly understood, rationality points us to the universalized abstractions (for example, the distribution of means of production, productivity, market efficiency) which always condition individual experience in the modern, integrated world. 

%Modern rationality does not and should not claim that these constitute exclusive, comprehensively exhaustive or sufficient determinations of individual experience, but they are certainly a \emph{necessary} condition. Sometimes, the feminist insistence on individual, ``other'' experiences may actually serve well to perfect a universal rationality, for example, when it questions otherwise hegemonic discourses (`a car for everyone').

%Feminist critics, as I believe all social theoreticians, have to somehow further thinking about the world that enables people, and that enables progress. In this context, that means, they have to respond to the certain consequences of modern integration for the individual experience. They, and more positivist ventures no less, are obligated to \emph{integrate} the individual experience, its virtues and incomparability, wherever possible. But nothing fruitful follows, when the two perspectives are uncompromisingly opposed as mutually exclusive. No one is better off, and no one is empowered.

%\paragraph{No Reason Over Power Without Reason} In their discussion of deliberative democracy, feminist critics should pay due regard to its quintessential innovation: \emph{reason over power}(\citealt{Young-1996-aa}: 122). This not only describes a historical narrative, but also a functional necessity. To overcome ``power'' of plurally organized, but self-interested, and thereby always \emph{particular} agents as justification for aggregated decisions, \emph{some universal} criterion to negate particular claims is required. 

%Feminists, it appears, are opposed to this trade-off and believe we can leapfrog to an egalitarian \emph{and} particular utopia, which I think, for once is not only unlikely optimistic, but also, as of yet, inconclusive\footnote{Somewhat surprisingly, \citet{Young-2001-aa} speaks in favor of confrontational political activism of the subaltern elsewhere. This stands in an uneasy contrast with her, and the deliberative project's ambition to rid political decision-making of power. She argues, forcefully that `to have the subaltern speak \citep{Spivak-1988-aa}, they may have to shout first', before they deliberate. This familiar ``overrun'' vs. ``overcome''dichotomy is not new, well known from the Civil Rights struggle in the U.S.. The risk is that its main proposition to ``overrun'' old power structures before they can be ``overcome'' may counteract the fundamental, and universal delegitimization of power claims on which deliberation rests.}.

%For now, what remains --- or prevails --- when power is delegitimized, and reason relativized, as the Feminist critique suggests? My hunch would be: power wins. And that, from an enlightened positivist perspective, as well as from emancipative or critical theory viewpoints, cannot be desirable.

%The solution, in \citet{Rawls-1971} operative metaphor, must be to perfect the ``veil of ignorance'' towards power --- not towards individual experience, of necessity --- and to replace it with a human-faced rationality.

%\section{Conclusion: \\In Need of New Schools for Democracy}

%This discussion of potential pitfalls of deliberative designs as they are suggested today, and its underlying discourse, must not be misconstrued as an argument against small-level, or any, deliberation. My goal was to draw attention to the possibility of inevitable failure of such deliberation, in spite of its desirability because of a frequent incongruence between inputs and outputs, the political and the social world, under newly suggested formats as well as a somewhat questionable optimism about the cognitive competence of deliberating citizens and their equality (!).

%\paragraph{Capacity: Keep Going, Go Big} Small-scale deliberations, already, have yielded great results. In particular, as \citeauthor{FungWright-2001-aa} (\citeyear{FungWright-2001-aa}: 29) hope, it has educational qualities ``beyond the proximate scope and effect of participations, these experiments also encourage the development of political wisdom in ordinary citizens by grounding competency upon everyday, situated, experiences rather than simply data mediated through popular press, television or `book learning'.'' 

%This is a hope that may well be justified, in part. Wherever functional differentiation, antecend division of labor and far-flung interdependencies progress to a certain degree, macro abstractions rule the human existence. And thus, good deliberators will need their `book learning', too. Ordinary citizens (a conspicuous distinction) are affected everyday by things beyond their immediate, grounded experiences, in fact, affected by things they never experience at all. 

%It follows a call to reconsider the design of low-level deliberation, of deliberation (exclusively) under devolution in the absence of higher-level decision-making of the same, essentially desirable quality. 

%What must follow, are suggestions on how to integrate low-level deliberations at higher levels, and how to bring macro-abstractions to bear on their decision making. The recently suggested ``Deliberation Days'' may be one promising avenue to bring higher-level decision-making into the participatory picture, and, by the same token, to reconcile deliberative demands with representative democracy \citep{AckermanFishkin-2002-aa}.

%\paragraph{Competence: Keep Going, Careful} Likewise, the skeptical findings about the cognitive competence of deliberators, should not be construed to question the egalitarian or inclusive ambition of the project. Rather, they necessitate designs that take due account of this likely fact, in particular, of the unequal cognitive abilities of deliberators.

%As Rosenberg writes:

%\begin{quote}
%``Deliberative fora must not be regarded simply as empty stages that provide a venue for the realization of citizenship; nor must the design of these deliberative stages focus simply on the removal of obstructions that may inhibit freedom or give unequal scope for maneuver. Instead, deliberation must be understood as a site for the construction and transformation of citizenship. In deliberation, citizens are made as well as realized. The operative metaphor here is that of a school, but of particular kind. The educational goal is not the transmission of specific beliefs and values, although these are by no means irrelevant. Rather the central aim must be to foster the requisite cognitive development for a fuller autonomy, a greater communicative competence and a better ability to engage in a collaborative effort to make good and just public policy.''
%(\citealt{Rosenberg-2007-aa}: 20)
%\end{quote}


%copy/paste from europe piece, merely to inspire the discussion of democracy. These issues also show up in democracy, not just EU.
\subsection[Difference]{Difference} \label{sec:ID-Difference} 

\begin{quote}
	\emph{``The boundary is not a spatial fact with sociological consequences, but a sociological fact that forms itself spatially.''} \\
	--- Georg \citeauthor{Simmel1903} (\citeyear{Simmel1903}: 142)
\end{quote}

To this cosmopolitan prescription, people often reply with a positive and normative critique of difference, or --- the dreaded I-word --- identity. 

Positive critics of would-be United States of Europe point out that ``empirical conditions'' for democracy and statehood are not given: there is, supposedly, no european people, no european public sphere and no european party system\footnote{
	This according to former Hans-J\"{u}rgen Papier, former president of the German Constitutional Court at a panel discussion in Berlin on October 17, 2013.}.
Similarly, \citeauthor{Scharpf1997} more carefully 	speaks of ``a pre-existing sense of community [\ldots], of common history or common destiny, and of common identity'' on which democracy depends, but ``which cannot be created by mere fiat'' (\citeyear{Scharpf1997}: 20).
Such skepticism about european democracy seems to be founded in the assumption of all hitherto, modern statehood as ``socioculturally homogenous'' (for example, \citealt{BeckGrande-2007-aa}: 93) --- a notion that \citeauthor{Scharpf1997}, somewhat illogically, rejects (\citeyear{Scharpf1997}: 20). Historically and conceptually, that is not true: (democratic) states made nations at least as much as nations made (democratic) states (\citealt{Gellner-1983-aa}, recently \citealt{Schmitter1999}: 934). But even aside the history, methodological nationalism is also epistemologically fallacious. Again, \citeauthor{Brubaker-2002-aa}'s dictum applies: if, indeed, we have no European democracy because people feel as citizens of its \gls{MS}, than that is ``what we want to explain, not what we want to explain things \emph{with} (ibid.: 165, emphasis in original), or as \citeauthor{Simmel1903} said, a merely spatially formed, but sociological fact.

Normative critics of would-be United States of Europe --- maybe more honestly --- argue for smaller polities to preserve diversity. Surely diversity of identities and lifestyles is something worth protecting, but this argument perilously confuses what Jonas \cite{Marx2012} has recently identified as \emph{political} and \emph{ontological} identity. Ontological identity, for \citeauthor{Marx2012}, arises only as actual people get to know each other, their character, and preferences. Ontological identity, therefore, exclusively resides within small groups or institutions, such as a circle of friends, family or a small firm: it cannot be scaled up\footnote{
	\cite{Marx2012}'s ontological identity is similar to \cite{Brubaker-2002-aa}'s definition of groups (as opposed to groupisms): they, too, are mutually interacting, as friends and family are, and not merely based on categories, such as ``Greek'' or ``German''.}.
Political identity, by contrast, can be scaled to arbitrary dimensions, as it arises wherever people form a democratic polity, as their lives become more intertwined by, for example, economic integration. Maybe inspired by the proto-deliberative \cite{Arendt1958}, \citeauthor{Marx2012} takes pains to remind readers that political identity is \emph{not} merely an assortment of particular, material interest\footnote{
	\citeauthor{Marx2012} points out that etymologically, ``interest'' refers to that between and beyond individuals, and their needs.}, 
but, by contrast, that one takes on a political identity purely as a choice, if and to the extent that particular, material \emph{needs} are fulfilled. 

Both the traditional conflation of ``nation (=) state'' and the difference critique of European integration confuse ontological with political identity. The nation state assumes \emph{ontological} homogeneity, a dangerous fiction that has gone horribly awry more than once in the 19th and 20th century. By this rather sorry standard --- political integration through homogenization --- European integration really does look quite worrying. If not only all Germans must be punctual, Sauerkraut-eating and Goethe-reading, but all Europeans must share the same ``judeo-christian values'', the ``Occident'' really is in trouble.

Pluralist democracy and its supposed home, the nation state, harbor another ideological fiction: that polities are, or ought to be, merely households, or even families writ-large. The first-order, material reality of Europe, or any other modern society, is of course truly that of a household, \emph{oikos}, or economy, ruled by interests, or even needs. But democracy is, and ought to be, more: it \emph{is} the one, purposeful (not aimless!) process by which societies come up with their second-order answers, and, thus, must ideally be \emph{isolated} from material interest, or need. Pluralist democracy, however, extends the incentive logic to government: here, too, as in a household, people have pre-social interests, including material needs, that are simply aggregated into decisions. Akin to markets, pluralist democracy knows only procedural rules of ``level playing fields'', but allows no substantive standards. Preferences are a black box, beyond reproach or qualification. The nation state, likewise --- already etymologically (\emph{natio}, that which has been born) --- presumes (incorrectly) common descent, and therefore suggests that nationals share some common, material interest. By this standard, too --- politics as interest aggregation, or agglomeration --- further European integration seems doomed, if not dangerous. If not only in the Bundestag, MPs are supposed to vote their constituents' (or, more malignly, their lobbyists') pocket books, but in the European Parliament, too, MPs vote their countries --- much different --- welfare, the union may fray.

There are more hopeful alternatives to nation state democracy, and therefore, hope for European democracy. 

First, there is no reason why further European political integration would require any ontological homogeneity: just as people drive much the same cars in Portugal as in Sweden, they can be regulated by the same emission's trading and taxed by the same code, without anyone having to give up what they eat, how they dress or to which (or none) god they pray. In truth, this homogeneity has not existed in the \gls{MS}, except as nationalist ideology or xenophobia-baiting. A rural Bavarian may not --- other than her passport and other institutional vestiges --- have much more in common with a Hamburger, than with a Parisian, or an Athenian. They do not even really speak the same language, as is so readily assumed \citep{Kymlicka-2001-aa}: talking about the economic abstractions of the Euro crisis, a \emph{Financial Times Deutschland}-reading German will have almost as much trouble understanding a \emph{BILD}-reading compatriot as a \emph{Daily Mail}-subscriber from the UK. Surely, such absence of \emph{logos} is dramatic for democracy, if it is ever to allow \emph{action} \citep{Arendt1958}, but it is not much more of a challenge to establish it between, than within \gls{MS}.

Second, democracy must, and can be more than ``two wolves and a sheep voting on what to have for dinner'', as Benjamin Franklin remarked. The economic abstractions and material view of the mixed economy I have presented here may be misunderstood to suggest that a European intact mixed economy, too, is just another, more efficient way to organize a polity as a large oikos, as \citeauthor{Arendt1958} might have deplored. That is not so: european democracy is not instrumental to resolving the \glspl{PD} and other inefficiencies and inequities of currently defunct mixed economies, but the other way around, an intact mixed economy is merely instrumental to restore the freedom from need to \emph{act} democratically as citizens in a newly potent, European polity.

%Arendt ist die Hinwendung der Philosophie zur Welt.

%Arendt sagt das Subjekt muss neu verstanden werden: total proto-deliverativ

%Arendt sagt: Freiheit meint leider oft: ich will Freiheit von Politik lass mich damit in Ruhe...

%deliberative democracy, as a term, was first used by \cite{Besette1981}

%note this wicked link between deliberative democracy and the violent state: before the powerful proto-state, decisions often had to be by consensus (as in hunter-gather bands, apparently), and only once there was a Leviathan could we do without consensus and instead go for 1) parasitic rule 2) parasitic pluralism 3) minimalistic, electoral democracy

%maybe, deliberative democracy is becoming, not being (cite the justice on this)