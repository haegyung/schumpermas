%!TEX root=../tax-democracy-held.tex

\begin{landscape}
\footnotesize
%\begin{longtabu}[]{p{1cm}p{2.7cm}p{3cm}p{3cm}p{3cm}p{3cm}p{3cm}}
\begin{longtabu}[]{X[2]X[2]X[2]X[2]X[2]}
\caption[Tax Validity Claims]{Hypothetical Examples of More and Less Valid Claims in Taxation\label{tab:validity-claims-tax}}\\
%\small

\toprule

\emph{}
&\emph{Comprehensibility}
&\emph{Truth}
&\emph{Thruthfulness}
&\emph{Rightness}
\\

\midrule

%\nameref{chap:wanted}: \nameref{sec:epistemology}
%&
%&
%&
%&
%\\
%
%\nameref{chap:wanted}: \nameref{sec:axiology}
%&
%&
%&
%&
%\\
%
%\nameref{chap:wanted}: \nameref{sec:ontology}
%&
%&
%&
%&
%\\

%\nameref{chap:mixed-economy}: \nameref{sec:ends}
%&
%&
%&
%&
%\\
%
%\nameref{chap:mixed-economy}: \nameref{sec:means}
%&
%&
%&
%&
%\\

\nameref{chap:desirable-tax}:\ \nameref{sec:tax-optimality} (p.~\pageref{sec:tax-optimality})
&
	\st{``A deadweight loss occurs if otherwise pareto-improving exchanges do not occur''.}

	``Taxes can sometimes prevent people from trading things on the market, which they otherwise would have exchanged to at least mutual benefit.''
&
	\st{``Taxes are anti-growth''.}
	``Under some circumstances, taxes can depress market activity without raising revenue''.
&
	\st{``Lower taxes benefit everyone, because growth trickles down''.}

	``If and to the extent that people react to incentives, and that we can really compare their utility, perfect markets make some people better off without hurting others, compared to how things were before.''
&
	\st{``No one would get any work done, if all their rewards get taxed away!
	''}

	``It seems that markets do a good job in solving some problems, and we can, to that extent and in those domains accept that people will also be influenced by threats and rewards''.
\\

%\nameref{chap:desirable-tax}: \nameref{sec:tax-justice} (p.~\pageref{sec:tax-justice})
%&
%&
%&
%&
%\\

%\nameref{chap:desirable-tax}: \nameref{sec:tax-sustainability}
%&
%&
%&
%&
%\\

\nameref{chap:doable-tax}: \nameref{sec:tax-incidence} (p.~\pageref{sec:tax-incidence})
&
	\st{``The tax burden is distributed according to the relative price elasticities of sellers and buyers in any given taxed transaction.''}
	The ultimate burden of a tax may differ from who nominally pays it.
	Taxes are almost always levied on some trade, and their ultimate burden is distributed among the traders according to how sensitive they are to price changes.
	The less you can react to a price change, the more of a tax you bear.
&
	\st{``With a Financial Transaction Tax (FTT), we can take back the money from those who caused the financial crisis.''} %gls for FTT does not work

	``To the extent that a FTT discourages short-term trades, it does not raise any revenue.
	To the extent that a FTT raises revenue, behavior does not change.
	The burden of a FTT will be complexly determined by relative elasticities.''
&
	\st{``Employers already pay half of social insurance''.}

	``Social insurance is a payroll tax and falls entirely on labor, no matter the nominal distribution.
	The burdens are shared between employers and employees according to their relative price elasticities.''
&
	\st{``Big corporations should pay their fair share''.}

	``Corporations are not moral subjects;
	they should not be liable for redistributive or general revenue taxes.
	However, the rich people who hold a lot of stock in corporations should pay taxes on such income, or wealth.''
\\

%\nameref{chap:doable-tax}: \nameref{sec:tax-elasticity}
%&
%&
%&
%&
%\\
%
%\nameref{chap:doable-tax}: \nameref{sec:tax-schedule}
%&
%&
%&
%&
%\\
%
%\nameref{chap:doable-tax}: \nameref{sec:tax-timing}
%&
%&
%&
%&
%\\

\bottomrule
\end{longtabu}
\end{landscape}