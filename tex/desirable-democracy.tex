%!TEX root=../tax-democracy-held.tex
\section{Participation}

\section{Enlightened Understanding}

\section{Justice or Equality}

%about stories
	%``When a story gets inside us, we get less crazy.'' (Ira Glass)

%cite somewhere first sentence of Adam Smith, moral sentiments
	%effectively says:
%no one is an island

%here comes the legitimacy essay (jachtenfuchs)
	%Legitimacy started to become a concern for large-scale government in the 18th century, as relatively young nation states consolidated their monopolies on force and the rule of law.
%The advent of democratic rule can be understood both in terms of respective normative and philosophical progress as well as against the background of specific historical and functional constellations.
	%American revolutionary Thomas Paine (1776) first promoted representation (not explicitly democracy) only as corollary of independence.
%Criticizing the outcomes of British colonial rule, such as extraction of resources and involvement in European wars, Paine rested his argument on an idealist notion of autonomy, that is, the right of the governed to have a say in their government and cited American ability to ensure internal and external security as well as to maintain the rule of law in support of independence.
%While upholding a substantive notion of self-determination through representation, Paine’s, as much of the American Revolution’s democratic ambitions, are then best understood as means to minimize otherwise necessarily evil government, ensuring security and the rule of law but otherwise serving and leaving pre-political society to its state of natural liberty.
	%Samuel Huntington (1991) surveys what he has identified as three waves of democratization, of which the American, French and British revolutions only jumpstart the first (1826-1926), later partly reversed by pervasive forms of nationalist totalitarianism (1922-1942).
%A second wave occurs after World War II and in the course of decolonization (1943-1962), which in turn is partly reversed by the shifts towards bureaucratic authoritarianism.
%The third, and ongoing wave, begins with the dismantling of European and Latin American authoritarian regimes, gaining further momentum with the liberalization of ex-communist countries.
%Huntington employs a merely procedural (input-focused) definition of democratic rule as essentially effective participation in elite contestation (c.f.
%Dahl 1989, Schumpeter as cited in ibid.), also ignoring the question of democratic consolidation and scale of measurement (dichotomous vs.\ continuous) for his purposes.
%Huntington sees a two-step-forward, one-step-backward pattern in the expansion of democratic rule, associated, he suggests with greater individual liberty and reduced international and political violence.
%Wondering whether democracy has “won”, he identifies factors why it may or may not last, continue or cease to expand.
%With regard to the sustainability of existing democracies, Huntington highlights the importance of democratic values, points to a need for effective socio-economic reconciliation and sustained growth to prevent popular polarization, warns of “democratic overload” (c.f.
%King 1975, Birch 1984), external and internal violent disruptions and argues for regional snowballing effects in either direction.
%The continuing expansion of democracy, he argues, depends on future external support, particularly American appeal and role-modeling as well as potentially crucial internal characteristics of democracies-to-be, such as their cultural fit with democratic order and economic progress.
%He provocatively suggests that Confucianism and Islam may not be easily compatible with democratic order and fears new, effective and popularly appealing forms of economically advanced non-democratic rule such as religious fundamentalism (Iran) or oligarchic authoritarianism (Russia).
	%As democracy spreads in the 19th and 20th century, normative democratic theory evolved and differentiated.
%Robert Dahl (1989) provides one synthesized overview of five elements of an idealized democratic process, which real-existing polyarchies only approximate.
% First, democratic decision-making on collectively-binding decisions (political decisions) rests on effective participation (1) by which people can communicate, promote and insert their preferences into the process, requiring elections, and essential freedoms of speech, press and association.
%Effective and equal participation is challenged, when interest groups become overly powerful or are disregarded.
% To this he adds, voting equality at the decisive stage (3), that is, at a stage when concrete choices are available, but all previous decisions are still subject to change.
%Dahl, in contrast to Huntington, does not limit free and fair elections to effective elite contestation, but remains agnostic towards the institutional alternatives such as parliamentary representation or referenda (see below on Lijphart 1999b).
%Voting equality, on Dahl’s yardstick, is then much easier challenged by low turnouts, multilevel politics/systems (c.f.
%Zürn 2000) and increasing influence of experts, the senior executive and extra-parliamentary committees.
%Falling somewhere between procedural and substantive definitions of democracy, he also stresses enlightened understanding (2) as an element of the democratic process, whereby citizens discover, formulate and evaluate available policy options, rather than to act on ad-hoc, inconsistent, potentially dysfunctional preferences (c.f.
%Converse 1970).
%Risks for this understanding can arise from a monopolistic media landscape and generally deficient or stratified education.
%On the procedural meta-level, Dahl also sees a need for people to effectively control the agenda (4) of the democratic political process, which can happen, again, through a variety of institutions and may, like voting equality, be threatened by multilevel politics/systems.
%Lastly, Dahl’s democratic ideal mandates a categorically, pre-politically defined inclusive demos (5), by which he means all adult citizens, a practice fundamentally challenged by migration.
	%As democracy spread, also evolved different institutional variations in terms of seat allocation and preference aggregation (proportional representation vs.\ majority voting), systems of government (presidential vs.\ parliamentary) and  horizontal and vertical separation of powers (checks & balances, federal vs.\ unitary organization).
% Arend Lijphart provides an empirically based typology to systematize the observed institutional diversity.
%He suggests that democracies fall into two categories, majoritarian/Westminster and consensus/consociational and can differ on two dimensions, federal-unitary and executives-party .
%Majoritarian democracy typically rests on simple majority or plurality votes (in single-member districts), creating strong, one-party governments with or without popular majorities, where political opposition concentrates on voting the ruling party out of office.
%In consensus democracies, seat allocation and/or preference aggregation is done through proportional representation (PR), leading often to multi-party systems and potentially less stable coalition governments also featuring greater minority rights and more compromise.
% The federal-unitary dimension of political systems concern the degree of power a party commands once it has secured government.
%Majoritarian, on this dimension, means to have more power “at the center”, including a more unitary (as opposed to federal) (1) vertical and likely unicameral (as opposed to strong bicameral) (2) horizontal separation of powers as well as (3) more constitutional flexibility and (4) more lenient, if any, judicial review and (5) a more dependent central bank, with consensus designs occupying the polar opposite of each of the above.
%Systems differ on the executive-parties dimension by the likelihood of single parties to dominate government.
%Majoritarian designs are more likely to do so by virtue of (1) more likely single-party majorities, (2) a stronger executive or president, (3) more likely two-party systems, (4) majority or plurality (first-past-the-post) electoral rules and (5) highly pluralist and specific (as opposed to corporatist and integrated) interest groups.
%Lijphart contradicts the conventional wisdom that Westminster democracies provide more efficient government and provids evidence that consensus designs yield comparable if not better outcomes (less inflation, better welfare, better environment).
%He also praises consensus systems for being more democratic using, amongst others, an operationalization of Dahl’s above mentioned criteria as well as for – classically – being more representative (higher turnout, more women MPs).
%Lastly, Lijphart sees consensus democracies as generally “gentler and kinder” (smaller income disparities, less incarceration and capital punishment, more foreign aid), as well as specifically more suited to divided societies with enduring, rather than cross-cutting cleavages.
