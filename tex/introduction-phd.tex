%!TEX root=../tax-democracy-held.tex

%quote missing! or do I not need one here?

%this needs to be re-arranged when the other chapters are done.

This dissertation is fairly pedestrian.
I tread in some of the mundane minutiae of modernity;
initially the twists of tax, and later the details of democratic rule.
And yet, in that small print of the social contract, I have found a veritable crime story.

The story begins with \hyperref[part:puzzle]{a puzzle} (\autoref{part:puzzle}, p.~\pageref{part:puzzle}):
why, in the richest of countries, in the most enlightened of times (current day \gls{oecd}-world) do we still find ourselves amidst \hyperref[chap:3-crises]{three intertwined crises} (\autoref{chap:3-crises}, p.~\pageref{chap:3-crises}):

\begin{enumerate}
	\item of welfare states, too constrained to elegantly improve upon the equity, efficiency and sustainability of markets,
	\item of democracies, too paralyzed to meaningfully rule their economies and
	\item consequently, of political equality and economic opportunity?
\end{enumerate}

This lament alone does mean there was crime:
maybe, these are no crises, but just facts of life, and a thorough investigation is unnecessary.
Likewise, we do not call in murder police when some centenarian does not wake up one morning.
However, at least, we task a doctor to establish a cause of death.
So it is with these three crises:
before any criminal investigation can start, I must establish a causal theory for constrained welfare, paralyzed democracy and rampant inequality.
I find that causal theory in the complex interactions of markets and plans, as they coexist in a \hyperref[chap:mixed-economy]{mixed economy} (\autoref{chap:mixed-economy}, p.~\pageref{chap:mixed-economy}).
Specifically, I find that of the political institutions that make up an intact mixed economy, \hyperref[chap:tax-matters]{tax matters} most to welfare, democracy and equality (\autoref{chap:tax-matters}, p.~\pageref{chap:tax-matters}).
%wanted is missing.
%also, the order is wrong, but I am not sure thatÄs so bad.

This dissertation is staunchly reformist.
Just because a system may be in crisis, it does not require a revolutionary overhaul, nor does it warrant dialectical glee about its supposed inevitability.
I stick to the market economy \emph{and} the welfare state, and merely tinker with tax to make them coexist better, and to mutual advantage.
There, I have found a scandal.
%some chapters are missing here

Our choice of tax can only be scandalous, let alone criminal, if there is, in fact, a better tax.
And so, in \autoref{part:tax} (p.~\pageref{part:tax}) I ask:
what makes a \hyperref[part:tax]{tax} \hyperref[chap:desirable-tax]{desirable} (\autoref{chap:desirable-tax}, p.~\pageref{chap:desirable-tax}) and what makes it \hyperref[chap:doable-tax]{doable} (\autoref{chap:doable-tax}, p.~\pageref{chap:doable-tax})?
I evaluate all real and hypothetical taxes on these criteria, and find:
(progressive) taxes on the (unimproved) value of land and (postpaid) consumption are much \hyperref[chap:better-tax]{better} than the ones we have (\autoref{chap:better-tax}, p.~\pageref{chap:better-tax}).
By comparison, our existing taxes on income and (prepaid) consumption appear staggeringly inefficient, inequitable and unsustainable.
%but also unfair, reference all of these adjectives.
Because the tax foundations of our mixed economies are thus sabotaged, our democracies must needlessly trade off efficiency, equity, and sustainability.

This dissertation is sociological, not just economic.
I cannot merely posit a suboptimal choice in tax, but I must theorize and test some social process to account for the supposedly suboptimal choice in tax.

\hyperref[part:democracy]{Democracy} is the social process that ought to rule collective choice (\autoref{part:democracy}, p.~\pageref{part:democracy}).
It, too, needs specification:
what \hyperref[chap:desirable-democracy]{should} it strive for (\autoref{chap:desirable-democracy}, p.~\pageref{chap:desirable-democracy}), and what \hyperref[chap:doable-democracy]{can} it deliver (\autoref{chap:doable-democracy}, p.~\pageref{chap:doable-democracy})?
In the balance of these questions I find:
liberal, but deliberative democracy may be much \hyperref[chap:better-democracy]{better} than the representative institutions we have (\autoref{chap:better-democracy}, p.~\pageref{chap:better-democracy}).
By comparison, the status quo of pluralism appears to be a minimal formulation of liberal democracy, as skeptical about what people \emph{can} do, as it is restrictive about what democracy \emph{should} do.
Once a historical achievement, it today appears hopelessly outmatched by the vast complexity and tightly concentrated special interest of late capitalist society.

So, ``who dunnit"?
Because this dissertation is, at bottom, positivist, I cannot show what --- let alone \emph{who} --- prevented a better tax, or a better democracy.
Non-events such as as land or consumption taxation and deliberative democracy are always causally underdetermined, just as history is always overdetermined.
For a detective, that is a bit of a let-down, but there is still work to be done.
I might not find the perpetrator, but at least, I can absolve wrongly accused democracy, and, during the hearing, spread word of  the crime.

\autoref{part:tax-democracy} (p.~\pageref{part:tax-democracy}) investigates the link between \hyperref[part:tax-democracy]{tax and democracy}.
I start by rounding up \hyperref[chap:no-better-tax]{some suspects} for suboptimal taxation (\autoref{chap:no-better-tax}, p.~\pageref{chap:no-better-tax}) and proceed --- as Holmes advises --- by the method of elimination.
First up:
democracy itself.
Maybe, ``suboptimal'' tax is, in fact, the \emph{popular} choice.
Only if it is not, is there a failure of the political process to be puzzled over, and accounted for.
I suggest that tax choice depends on the \emph{kind} of democracy.
Tax choice may suffer \hyperref[chap:tax-under-pluralism]{under pluralism}, because it is very complicated and offers tightly concentrated special interest (\autoref{chap:tax-under-pluralism}, p.~\pageref{chap:tax-under-pluralism}).
More broadly, the two crimes of suboptimal taxation and limited democracy are intimately related.
People might not just widely misunderstand taxation, but they might err systematically about the trade-offs of ruling a mixed economy.
These systematic misunderstandings under the dysfunctions of pluralism might interact with the unattractive alternatives of a dysfunctional tax regime to dramatically constrain any popular choice of tax, and divert the polity away from the equity, efficiency and sustainability they might otherwise prefer.
Democracy and taxation might be closely intertwined in their supposed mutual crises, but they also constitute two sides of the same coin that is the liberal-democratic, capitalist social contract.
Democracy concerns the making of collectively binding decisions, taxation is the chief means to implement these agreed-upon plans within the market exchanges of free agents.
Conversely, democracy legitimates tax and calls the trade-offs of the mixed economy, and taxation also shapes the material conditions under which people decide, and collects the resources to power policy.
Not surprisingly, democracy and taxation share \hyperref[chap:common-grounds]{common theoretical grounds}, and their crises and reform hinge on the same questions of equality, justice, cooperation and human nature (\autoref{chap:common-grounds}, p.~\pageref{chap:common-grounds}).
%cumbersome, improve wording

This dissertation is also empirical.
To stay clear of hermetic ideology, and self-referential critique, I must get up from the normative armchair and show that really, a better democracy and better tax \emph{are} related and \emph{can} be had.

A good crime story cannot rely on conjecture alone, it needs supporting evidence.
I cannot provide an instrument of crime because history does not leave material what-ifs in its wake.
But I can try to capture these hypotheticals in an \hyperref[part:test]{experimental trap}:
if ordinary %better wording
people, under an \emph{alternative}, democratic process, choose an \emph{alternative}, preferable tax, \emph{that} would be as close to a smoking gun (\autoref{part:test}, p.~\pageref{part:test}) as it gets.
If, moreover, that experimental democracy can be shown to alleviate some of the dysfunctions of pluralism, and misunderstandings of the mixed economy, and if, consequently, through that democratic fora, people prefer an alternative tax, we know for sure that \emph{some} crime \emph{has} happened.

This dissertation is, lastly, \hyperref[part:hope]{hopeful}, despite it all (\autoref{part:hope}, p.~\pageref{part:hope}):
the crime remains unsolved, and I can point to no single societal culprit.
Maybe, that is for the better, as few good things ever come from such structuralist exorcisms.
There may be many reasons why we have no better tax, and the \emph{kind} of democracy we have might plausibly be one of them.
That alone ought to worry us.

If, in fact, our pluralist democracies becoming illegitimate and inefficient, their increasing failures will plague many arenas of collective choice other than taxation.
The death knells of legitimate and efficient pluralism --- complexity and tightly concentrated special interest --- \emph{are} the conditions of late capitalist society.
%they are also the good news of late capitalist society, because zero-sum transactions are bad
I \hyperref[chap:conclusion-phd]{conclude} that taxation may just be one these policy fields in which the limits of pluralism show early, and clearly (\autoref{chap:conclusion-phd}, p.~\pageref{chap:conclusion-phd}).
%broken href
Indeed, the entire story could be told the other way around, with popular rule as the corpse that found its master in the intricacies of tax.

This dissertation is fairly long and not very original.
Others already have, in different places and with different foci, written almost all that is contained in the next pages.
%add current total number
And yet, so far, no one has bothered to connect the dots.
That is what I offer here.

Raymond \citeauthor{Williams1992} has argued that the detective story appears at roughly the same time as sociology, and for the same reason:
to penetrate the modern world, as Conan Doyle's \emph{Sherlock Holmes} does, ``by an isolated rational intelligence'' \citep[88]{Williams1992}, otherwise covered by London fogs, or compartmentalized by functional differentiation \citep{Durkheim-1893-aa}.
And so it is with this crime story of taxation and democracy:
shrouded in nebulous complexity, pigeonholed into disjunct institutions and areas of expertise, it needs a lot of detective work to connect these dots.
To hear this case, I urge the reader to consider all of the evidence reviewed in this exposition, even if some of the detail might, at first, seem trivial.
It is not.
As Sherlock Holmes reminds us, ``there is nothing so important as trifles'' \citep[238]{Doyle1891}.
From the incidence of the \gls{cit}, the liquidity effects of taxing property to the macro- and micropolitics of democracy and the sociobiology of fairness, cooperation and equality
%add hrefs
--- these ``trifles'' are all important exhibits to make the case that democracy, taxation, and with it, all our modern lives, \emph{could}, and therefore, \emph{should} be better.

Some stories just cannot be told in the short form, and the story I have stumbled upon and here report may be one of those.
%include reference to the effect that books are better from Caplan
Luckily, detective stories can also be fun to read, and so, I hope, is this one.