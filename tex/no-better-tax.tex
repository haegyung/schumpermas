%!TEX root=../tax-democracy-held.tex
	%tax history
		%still need to look into why it didn't happen (Newspaper archives et al).
%Genschel:
%"An idea whose time never came"

%add comment, footnote:
%this was first submitted as \ldots

%How Do We Get The Perfect Tax?

\section{Because It is Not Better}
%or:
%how to go further?

%First Order Questions of Social Change
%The first order questions are about the supposed superiority of the PCT.
%They include:
% Is the PCT really a more efficient, more equitable tax (quantitative ‘proof’)?
% This requires an extensive model of the economy, difficult empirical specification of key inputs (for example, elasticity in demand for luxury goods, saving), agreement on a number of contentious axioms (diminishing returns, positional consumption, optimal savings rate).
%This task is certainly beyond my training, arguably beyond the scope of a PhD thesis and possibly beyond the modeling powers of economics.


%consider the "Garbage Can model"/Kingdom for my tax.
%Maybe the time of my tax didn't come yet.
%But maybe, it's here now (sovereign default in Spain?!?).
%Read Kingdon.
%This is an idea that was developed in Kohler-Koch's idea seminar.

\subsection{Modeling}
%write a section on why modeling is futile.

% There is still considerable work to be done in specifying (cardinally) the schedule and base of a PCT regime, in its implementation and transition.
%This work depends, again, on a comprehensive model of the economy and bureaucratic detail.
%This grunt work is unlikely to yield analytically productive insights, and may best be confined to think tanks and tax administrations.
%Implementing and specifying the PCT, absent a (good) model of the entire economy will probably be a matter of trial and error.

\subsection{Specification}
%do i need this here?
% How would a PCT be (cardinally) specified and implemented?


\section{Tax Competition}
%maybe this should happen somewhere earlier?
%the below section is a duplicate from the dual crisis
%Background:
%Taxation Trends in the European Union

	%The Directorate General for Taxation and Customs Union issues an annual report on taxation in the European Union.
%Results from the 2009 edition include:
	%Tax ratios in the EU-27 remain relatively high (39.8% of GDP), but differ greatly between old and NMS (Romania 29.4%, Denmark 48.7%).
	%NMS raise relatively more revenue by indirect, non-redistributive taxes in immobile bases.
	%Almost all member states increase the (indirect) burden in (relatively immobile) consumption through VAT and excise duties.
	%Top average (not marginal) Personal Income Tax (PIT) rates (37.8%) are in decline across almost all EU member states, but continue to vary dramatically between old and NMS (Bulgaria 10%, Denmark 59%).
	%Corporate Income Tax (CIT) rates are in rapid decline, from 35.3% in 1995 to 23.5% now.
%Again, NMS tend to have lower tax rates than older member states.
	%Implicit Corporate Income Tax rates (CIT-ITR) are however, stable if diverging, possibly due to cyclical effects, base broadening or cannibalizing on the PIT.

	%Is there European (Corporate Income) Tax Competition?

	%In a liberalized Single Market, where capital and, to a lesser extent, labor through trade, investment and migration flow to their most profitable use, it appears reasonable to assume that these factors of production will also respond to taxes.
%Investors and workers will, as much as they can, flock to locations where tax rates are lower than the respective costs of relocation.
%To sustain output and growth, states would then have to compete for factors of production with low tax rates.

	%Before turning to the central question of whether this would be a desirable or undesirable dynamic, first ask whether it is real, and if so, how it came about.

	%Genschel, Kemmerling and Seils (2009) suggest (corporate income) tax competition is shaped by four interrelated institutional mechanisms of the EU:
	%Market integration (↑) reduces transaction costs of cross-border arbitrage (think:\ exchange rate volatility, tariffs) and thereby facilitates tax competition.
	%Enlargement (↑) adds new, attractive markets:
%new members are diverse in size (!) and economic development (GDP/Capita).
%Smaller and poorer countries have greater incentives and possibilities to lower taxes.
	%Tax coordination (↓) makes taxation more similar, and therefore reduces the ways in which governments can compete for capital and labor.
	%Supranational judicial review (↑) enforces the non-discriminatory liberalization (think:\ Cassis de Dijon) and limits the ability of governments to unilaterally defend against tax competition (think:\ Tobin Tax).

	%Genschel et al.
%find that tax competition is different and greater within the European Union than outside of it, and that it accelerates with time and enlargement.

	%They also suggest that larger countries suffer disproportionately from tax competition as smaller countries can boost their revenues (and growth) with the low rates on the massive inflowing capital from larger neighbors.
%Conversely, poorer countries have greater incentive and ease to compete for scarce capital.
	%The competition is less harsh amongst (now delegitimized) preferential tax regimes (PTR) (think:\ Hong Kong).
%When countries target only the most mobile of factors with tax breaks, the overall revenue effect tends to be smaller and more symmetric.

	%(A Prisoner's Dilemma Game of Tax Competition.
%Payoffs are tax revenues#.)

	%If these empirical results are correct, EU tax competition can be modeled as a Prisoner's Dilemma game, where states (strictly) dominantly prefer low taxes over high taxes and (Nash) equilibriate in suboptimal, mutual low taxation.

	%The Case against European Tax Competition:
%It's a Race to the Bottom

	%The argument against European tax competition builds on the assumed Prisoner's Dilemma dynamic, and suggests that EU states may not only be incapable to maximize public revenue under competition, but that thereby depressed tax revenue will lead to debt crises, the underprovision of public or common goods and impaired redistributive ability or a retrenchment of the Welfare State.
%It could also be argued that a harmonized European Tax regime better equips member states to withstand shocks with countercyclical tax and spend policies.

	%A related argument can be made about a presumed inability to respond to structural misalignments caused by liberalization.
%When trade, migration and investment allow countries to specialize even more according to their factor endowments (think:\ Romanian Nokia, German Management Consulting), remaining, relatively scarce factors (think:\ unskilled laborer in Germany) may find their market wages# fall even further below the respective socially acceptable minimum income#.
%Rich states may then be forced to redistribute income to these individuals, but find themselves unable to raise the necessary revenues (progressively) without further reducing their competitiveness.
%Tax competition could then exacerbate vicious dynamics of structural unemployment.

	%Tax competition is harshest on tax bases (capital, labor, firms) that are highly mobile (think:\ private equity).
%EU member state tax codes may be forced to converge on certain kinds of taxes.
%To appreciate this possibility, consider the qualities of different redistributive-/general-revenue taxes#.

	%(reproduced from Genschel 2007:
%11) (whatever this was?)

%Second Order Questions of Social Change
%The second order question of social change is simply:
%Why don’t we have it?
%Possible answers (hypotheses) are:
% Progressive Taxation is subject to an international cooperation problem, akin to a Prisoner’s Dilemma.
%Because the PCT is progressive, it is individually rational for states not to implement it, even if it were collectively rational.
% This is a very interesting question that could potentially greatly enhance our understanding of the international political economy of taxation.
%It however, also requires a comprehensive model of the economy, and even international trade and finance to anticipate the hypothetical payoff of the PCT.
% Tax regimes are subject to great path dependency, they cannot be easily reformed.
% While a historical and tentative analysis of path dependency in tax regimes is possible and probably productive, the costs of reform can ultimately only assessed when a cardinal specification and implementation has been undertaken.
%Assessing the costs of the reform ex ante may be very hard, unreliable or impossible.
%Both a possible international cooperation problem and path dependency in tax regime choice are ultimately endogenous to broader questions of the political economy.
%The cooperation problem depends on the level at which the international political economy is governed.
%Path dependency depends on the domestic and international incidence of reform costs, in turn depending on the states ability to hand out side-payments.

% Ultimately, the PCT could be absent because people do not want it.
%The second order question of social change in tax regimes is:
%will people reject the PCT irrespective of the democratic process that governs their polity?
%If not:
%how do different democratic processes (pluralist, deliberative) constrain or even determine the politics and choice of tax?

%likewise:
%assumed theoretical link:
%if you want to hide injustice, than you'd best do it in a really complex system, like tax.

%what's the second order theory?

%I don't explain the elite-based (potential) reason for the non-reform.
%That, the second-order theory, would be someone elses job.
%discourse theory here?

Political science offers at least three rival perspectives to explain why we do not have the perfect tax, and common sense adds a third \hyperref[sec:Conspiracy]{suspicion} (page \pageref{sec:Conspiracy}).
I present these rival explanations here in a preliminary form.
Empirical testing and more specification must await further research.

\subsection[Path Dependence]{Path Dependence:\ We Have Always Done This} The path dependence perspective suggests that social systems generally resist change.
As specific institutions and solutions are adopted, their (re-)production enjoys economies of scale, and it becomes ever harder to alter or replace past arrangements \citep{Mahoney-2000-aa,Pierson-2000-aa}.
Once the QWERTY keyboard was introduced in 1878 and people had learned to use it, it was carved not only in plastic, but also in stone.
Other keyboard designs have never attracted a significant following given the strong network effects (using other keyboards) and large economies of scale (being really good at QWERTY) of the QWERTY design.
Incidentally, QWERTY was never designed to be particularly efficient, but to minimize the clashing of two neighboring typebars, a problem which is now long obsolete.

Much the same story can be told for the personal income tax:
it was originally implemented not on superior principle, but because of its intuitive appeal and practicality.
In a time before widespread retail, let alone electronic banking, the PIT was attractive because it could be withheld directly at the source of few, well-organized firms in the newly industrializing world.
Once adopted, the PIT was strengthened by network effects (other countries)
\footnote{
	The problems arising from incompatible tax regimes (for example PCT vs.\ PIT) between trading countries are discussed in greater detail in \autoref{sec:IntlPCT}.
}
and economies of scale (taxmen and taxpayers know it well).
Fundamental reform, again and again, failed.
As the \hyperref[sec:ScorePIT]{principal problems of the PIT} (page \pageref{sec:ScorePIT}) became ever more apparent, it was supplemented or replaced by other problematic (regressive), but \emph{compatible} taxes:
\hyperref[sec:ScoreVAT]{the VAT} (page \pageref{sec:ScoreVAT}) and increasing payroll contributions to social insurance.

\subsection[Deficient Political Process]{Deficient Political Process:\ They Don't Get it}
An alternative, more cynical perspective suggests that while PCT reform is possible, it is never undertaken because of a confused and deficient political process.

On cursory inspection of the political debate in tax, this perspective is intuitive if appalling.
The public debate on tax, hardly, if at all corresponds to the abstractions governing tax.
When people and politicians demand more corporate taxation or fight for employer and employee parity in the financing of social insurance, at least one of them does not know what they are talking about\footnote{Why this is wrong:
sections \ref{sec:ScoreCIT} and \href{sec:SICAreTaxes}.}.

More systematic evidence for the inaptitude of voters come from psychology \citep{Converse-1970-aa}, opinion research \citep{Delli-CarpiniKeeter-1996-aa} and recently, political psychology \citep{Rosenberg-2002-aa}.

A look in the cookbook of a professional campaign strategist reveals a more fundamental malaise:
``the task of the campaign strategist is to find the easiest path to victory'' \citep[9]{Malchow2003}.
If politicians and parties go for the low road, campaigning on topics and positions that are easy to explain\footnote{For a preliminary formulation of such a neo-Downsian theory of dysfunctional political campaigning, see Held (2010b).}, tax will necessarily fall by the wayside.

\subsection[Global Cooperation Problem]{Global Cooperation Problem:\ Race to the Bottom}
The international political economy perspective suggests that the PCT does not exist because of a widespread cooperation problem of high taxation.
Whether globalization, or, more accurately denationalization retrenches the welfare state, is a question widely discussed in political science.

%here used to be %!TEX root=../tax-democracy-held.tex

\begin{table}
	\caption{International Tax Competition, Stylized as a Prisoner's Dilemma}
	\label{tab:Tax-PD} %change label?
	\begin{center}
	\begin{tabular}{m{1cm}m{2,3cm}m{2,3cm}m{2,3cm}m{2,3cm}}
		& & \multicolumn{2}{c}{\emph{Home}} \\
		& &Low Tax& High Tax\\ 
		\cline{3-4}
		\multicolumn{1}{c}{\multirow{4}{*}{\emph{Rest of World}}} & \multirow{2}{2,3cm}{Low Tax} & 		\multicolumn{1}{|r|}{3} & \multicolumn{1}{r|}{0}\\ 
		\multicolumn{1}{c}{} & \multicolumn{1}{c}{}& \multicolumn{1}{|l|}{3} & \multicolumn{1}{l|}{10}\\ 
		\cline{3-4}
		\multicolumn{1}{c}{} & \multirow{2}{2,3cm}{High Tax} & \multicolumn{1}{|r|}{10} & \multicolumn{1}{r|}{7}\\ 
		\multicolumn{1}{c}{} & \multicolumn{1}{c}{}& \multicolumn{1}{|l|}{0} & \multicolumn{1}{l|}{7}\\ 
		\cline{3-4}
	\end{tabular}
	\end{center}
	\scriptsize{\emph{Home} and \emph{Rest of World} are the only two countries. They set their tax rates either high, or low. Capital and other mobile factors flow to whichever country has the lower tax rate. Payoffs are state revenues.}
\end{table}, now only in tax-matters

\paragraph{The Missing \href{http://maxheld.de/2009/10/13/setting-goalposts/}{Normative Counterfactual}.}
	%kill the url?
To know whether, and how much denationalization retrenches welfare states, political science would have to assess the welfare regime in a sufficiently large, prosperous and closed economy.
This of course, does not exist in the OECD world, and may not exist at all in reality.
I argue that the appropriate response, if only for methodological reasons, is not to ignore this problem, but to hypothesize a best case, normative hypothetical, where \emph{cooperation is given}.

The PCT \emph{is} that desirable and doable hypothetical that would be expected if cooperation were given.

\subsection[Conspiracy Theory]{\ldots And a Little Conspiracy Theory}
	\label{sec:Conspiracy}
The practical and ``analytic muddle'' that tax is \citep[862]{McCaffery2005} raises suspicions beyond path dependence, political process and tax competition.
Current tax systems are not just poorly designed and poorly understood.
They are mis-configured and misrepresented in a very \emph{specific} way.

Progression in the current fiscal configuration rests on the increasingly feeble and dysfunctional income taxation.
The PIT is under pressure from two sides.
On the one hand, its ugly backstops, the gift, estate and Corporate Income Tax are ever open to evasion while stifling economic activity.
On the other hand, income taxation, particularly of capital incomes is extremely vulnerable to tax competition.
As capital became increasingly mobile over the past decades, progressive PIT and high CIT rates became untenable.

As the PIT and its backstops crumbled as a source of revenue, states turned to the (seemingly) only alternative:
flat VAT-style, prepaid consumption taxes and proportional wage taxes or equivalent Dual Income Taxes.
These taxes were roughly compatible with the remnants of income taxation, did not require dysfunctional backstops and were relatively immune to tax competition.
VAT, Payroll and Dual Income Taxes all fall on relative immobile bases:
it is much more costly to move abroad for a worker with her family than for a mutual fund to change its portfolio.

For the past three to four decades now, governments have shifted their tax mix from PIT to VAT and payroll taxation (data for Germany and the UK in \citealt[11]{Kemmerling2009}.
This shift comes at a hefty price:
as VAT and payroll taxation are proportional, the tax burden on low-productivity workers increases.
Present some socially acceptable minimum living standard, structural unemployment ensures, which in turn requires transfers out of general revenue.

The late developed welfare state is in a pretty bad fix.
Given costly unemployment, real dissavings and increasing public debt it is \emph{structurally underfunded}.
If it raises the PIT, capital drains and growth falls.
If it raises the VAT and payroll taxation, \emph{structural unemployment} rises or working poverty ensues \citep{Kato2003}.

\begin{quote}
	\emph{``This traditional view has generated an impoverished choice set for tax, consisting of a badly flawed status quo on the one hand and a flat consumption tax of some sort on the other.
	Under the guiding light of the traditional view, we are heading ever closer towards a flat wage tax.''}
	\\*
	Edward J.\ \citet[812]{McCaffery2005}
\end{quote}

The status of quo of tax is not just arbitrarily bungled.
It is misconfigured to lead us to ever more proportional, if not regressive\footnote{As argued \autoref{sec:savings-norms}, whether a wage tax is considered regressive or proportional depends on the chosen \hyperref[sec:savings-norms]{savings norm}.}, taxation of immobile sources.
It is misconfigured to lead us to a polity paralyzed by underfunding, to an economy of debilitating structural unemployment or heart-wrenching working poverty.
It is misrepresented to make us belief that our only choice in tax is between anti-growth PIT and proportional VAT or regressive payroll.

\begin{quote}
	\emph{``The real and pressingly practical question for tax is not whether to have an income or a consumption tax, but what form of consumption tax to have.
	The stakes in this battle are clear and dramatic:
	the fate of progressivity in tax lies in the balance.''}
	\\*
	--- Edward J.\ \citet[817]{McCaffery2005}
\end{quote}

The status quo of tax is bungled in a way that serves the interest of the beneficiaries of the status quo.
If you are rich, you are content with how the pie is sliced.
If you are rich, you do not depend as direly on underfunded public goods and risk pools:
you have your exit options (gated community, private health ``insurance'', boarding school).
If you are rich, and if your capital is mobile your luck does not depend even on growth in your home country:
you get to send your capital abroad, to whichever booming economy generates the greatest income.

This is not to suggest a historical materialism of tax.
Structuralist scapegoating is often analytically hermetic, sometimes rhetorically dangerous and always politically impractical.

But when a policy failure has such conspicuous distributive consequences, one has to wonder whether agency and interest make regressive taxation of immobile bases a \emph{successful} failure\footnote{Policy and organizations persistently and \emph{successfully} fail, argues Wolfgang \citeauthor{Seibel-1996-aa}, as principals have little interest and less ability to effectively monitor agents' implementation of stated goals (\citeyear{Seibel-1996-aa}).

Tax legislators are the principals, tax policy and administration are the agents.
Given their impoverished choice set, legislators have little ability to effectively command and monitor agents implementation of an \hyperref[sec:tax-justice]{equitable}, \hyperref[sec:tax-optimality]{efficient} and \hyperref[sec:tax-sustainability]{sustainable} tax system (pages \pageref{sec:tax-justice}, \pageref{sec:tax-optimality}, \pageref{sec:tax-sustainability}).
Given international tax competition, and the contradictions of current taxation, tax policy and administration has no way of ever reconciling these stated goals.
Legislative principals may be content with the persistent failure of taxation.
Knowing that an ideally equitable, efficient and sustainable tax system would require much greater redistribution, which they wish to avoid, persistent failure allows them to pay popular, but inconsequential lip service to defunct (income tax-based) progression.}.
For legislators captured by, or serving rich special interest window dressing minimally redistributive and defunct tax policy may just be the professional thing to do.

\section[Prospects of the PCT]{How Could we Get the Perfect Tax?}
	\label{sec:HowToGetIt}
Given all these obstacles, getting the PCT will not be easy.
There is, however, one effect of the PCT which may act as a catalyst for its introduction and proliferation.

Recall that the PCT will replace all other revenue-generating taxes, including those on capital and corporate incomes.
In the short to medium run, a solvent government could even lift income taxation on foreign residents and foreign-owned corporations.
As a result, capital would rush into the PCT early adopter country, causing substantial harm to other income-taxing countries \citep[12]{Dalsgaard2005}.
\emph{Every} capital owner abroad will be better off for \emph{every} unit of capital they invest in the PCT early adopter.

Without international harmonization and cooperation, \emph{home} residents will have an incentive to evade the PCT.
As argued in the above, they may either migrate altogether, or invest abroad and consume abroad out of their offshored capital incomes.
Either way, they will be subjected to income taxation abroad, but get to avoid the PCT at home.
Because capital owners still pay CIT, PIT or a withholding tax abroad, they are better off only if, and to the extent that they wish to consume their offshored capital incomes.
For many capitalists, this will be a (small) component of their overall net worth.

In the balance of these two investment flows, it appears that the PCT early adopter will gain significant capital inflows.
This, of course, is classic \emph{beggar-thy-neighbor} policy.
In the context of the PCT, threatening other countries with zero income taxation may however serve as a catalyst for the introduction and proliferation of the PCT\footnote{This scenario is better understood as a backdrop to the strategic, multilevel games of introducing the PCT than as an actual policy prescription.
Threatening, or actually implementing a beggar-thy-neighbor strategy of this magnitude may result in severe diplomatic and economic disruptions, or even violent conflict.}.
If a critical mass of countries adopts the PCT and abolishes all income taxes, other developed trading partners may be forced to do the same.

This hypothesized dynamic requires much greater specification and econometric as well as game theoretical modeling.
If the trick works, the PCT carries within it not only the seeds to its own beggar-thy-neighbor proliferation, but also a promising recipe to overcome the international cooperation problem of progressive taxation.

\section{Loose Ends}
	\label{sec:LooseEnds}
This thesis generates pressing questions and promising avenues for further research.

\paragraph{Quantifying the PCT:\ Experiment or Model.}
So far, I have \emph{deduced} the PCT as the \emph{ordinally} superior tax.
I have argued that \emph{given} desirable and doable desiderata, the PCT emerges as the perfect tax.
I have not shown \emph{inductively} that the PCT produces results in the real world that are normatively preferable.
I have argued that the PCT is better (\emph{ordinally}), but not \emph{by how much} (\emph{cardinally}), both in terms of efficiency (welfare), equity (distribution) and sustainability (savings rate).

	The PCT should be tested \emph{inductively} and the superiority of the PCT should be \emph{cardinally} quantified.

\subparagraph{Inductive Test.}
A straightforward \emph{inductive} test of the PCT is not possible:
no large, rich country will implement the PCT just to ``give it a shot''.
Existing econometric research on the effects of taxation will also be inadequate:
market reactions to such a discrete, encompassing reform cannot be gauged from observations of incremental, continuous changes.
To name just one example, the price elasticity of excessive (positional) consumption is unknown at very high rates of taxation.
Instead of econometric research, a carefully crafted experiment in behavioral economics may serve better to verify the PCT inductively.

\subparagraph{Cardinal Quantification.}
Showing exactly \emph{how much better} the PCT will make us in the real world hinges on the same problems as an inductive test.
Absent a \emph{test run}, any attempt at (general equilibrium) modeling of the tax will suffer from underspecification.
Again, the price elasticity of excessive (positional) consumption is one of the key unknowns:
just how much revenue the PCT will raise at any given schedule, or how much positional waste it will curtail is not readily observable.

An economic model of rich economies with a PCT would still serve to at least roughly quantify the drivers of welfare and distribution.

Again, a behavioral experiment may be a more elegant, and powerful design to cardinally estimate the benefits of the PCT.

\paragraph{Specifying the PCT.}
As is argued in the above, the PCT still requires a lot of grunt work detail.
In particular, monetary dynamics of the PCT must be addressed, its formulae and schedule must be specified, an elegant design and administrative process must be drafted and a its implementation must be planned.
Specification of the PCT will depend on cardinal quantification of its effects and require extensive economic modeling as well as econometric data.

\paragraph{Informing the International Political Economy of Progressive Taxation.}
If the PCT is indeed superior, cardinally and inductively, this verifiably desirable and doable hypothetical will put new questions to the literature on welfare retrenchment.
Given that a better configuration is possible, political science must answer whether denationalization and tax competition are to blame for the present misery.

Two formal approaches apply here.

\subparagraph{Multi-Level Game Theory.}
One is the multi-level game theory of international tax harmonization \citep{Scharpf-1997-aa}.
Here, considerable effort should be devoted to understanding the strategic imperatives of governance both at the national and international level to adopt or avoid the PCT.
An econometrically informed modeling of the distributive effects \emph{within} and \emph{between} countries will be required to gauge this multi-level game\footnote{Behind the political economy of progressive taxation may loom another, even bigger question:
just how fast do we want developing, low-tax countries to play catch-up?

Some component of international tax competition may be adequately described as a pure cooperation problem.
When countries of same size and same GDP per capita compete for capital inflows, tax levels are a positive sum game:
if only everyone else would raise taxes, too, everyone would be better off (or so I hypothesize, anyway).

But there is also a \emph{between-country} redistributive effect to tax harmonization.
To achieve full tax harmonization in open markets, as argued in the above, all countries would have to have the same tax rates and the same tax bases.
In poorer countries with a typically smaller public expenditure quota, Scandinavian tax rates would be untenable.
Conversely, when low-cost tax-dumping countries attract foreign direct investments (FDI) from capital-deepened rich countries, they put their economic development on afterburner.

The choice of \emph{global} tax rates and bases becomes a zero-sum game.
If they are high, poor countries will grow slowly, if at all, but rich countries get to keep their capital and redistributive welfare states.
If global taxes are low, poor countries catch up rapidly, but rich countries loose capital and suffer from overburdened, underfunded government.
Sidepayments for the developing countries may be a welfare (Kaldor-Hicks, by defintion) improvement to recompensate poor countries, and to get them on board.}.

\subparagraph{Veto Playing.}
Closely related, international tax harmonization should be modeled as a veto-player problem \citep{Tsebelis-2002-aa}.
Given the anarchic nature of the international realm (taking on a realist view), veto playing appears to be a reasonable approximation of international tax harmonization.
In particular, a modeled comparison between tax harmonization at the international and European level with their different veto rights may be instructive to estimate the overall significance of the model.

\paragraph{Modeling the Introduction of the PCT.}
Lastly, the chain reaction dynamic hypothesized in the above will need deductive, as well as inductive verification and econometric quantification:
just how much costs will the PCT early adopter incur to the non-adopters?
And how quickly will these costs materialize?
How large would a critical mass of countries have to be to introduce and proliferate the PCT?
Which countries would be suitable to ignite the chain reaction?
\clearpage

%the following is from more from Europe piece:
%second order theory
\section{Who Dunnit?:\ Second-Order Theory of Negative Integration}
	\label{sec:who-dunnit}

So, a reader may ask, what does any of this have to do with who, or what  ``built welfare states in post-communist \gls{ceec}'', or alters them elsewhere in the \gls{eu}?

The defunct mixed economy of the \gls{eu} has everything to do with the continents welfare states, not just because in the East, the \gls{eu} (and \glspl{ifi}) might have --- and still do --- dictate economic policy to any country who might ever wish to join the union through convergence criteria, accession negotiations and conditionality \citep[55]{Bonker2006}, but because, much, much more fundamentally, any economic or social policy adopted in \glspl{ceec} after 1990, or elsewhere, is greatly constrained by the European economic regime in place.
\gls{eu} regional economic integration, to be sure, only dictates what cannot be done:
no progressive or high taxation, no independent monetary policy, and, sometimes, no effective regulation.
%add hrefs
European integration, in \citeauthor{Scharpf1997}'s influential formulation, is \emph{negative integration}, but nonetheless consequential \citep{Scharpf1997}.
If there is anything to the otherwise muddled empire thesis \citep{BeckGrande-2007-aa} it is this:
cripple your mixed economy and settle for the promises of neoliberalism and consumerism --- or perish in autarky.

I do not know, nor claim to explain here, who committed this ``perfect crime'' \citep{Galbraith2002a} --- but murder of the mixed economy, it was, and that is the deed that any second-order theory of social change has to explain.
	%Crouch 158 also notes absence of wealth distribution data.

\subsection[Suspects]{Suspects:\ Hunching Second-Order Theory}

\begin{quote}
	\emph{``Only from the margins can you see well.''}
	\\*
	--- Michel \cite{Foucault-1972-aa}
\end{quote}

I have argued that the existing literature on the second-order theory of the welfare state is pretty bad, at least because it does not engage the relevant first-order alternatives.

But what do I suggest instead?
Not much, mostly, because I have no data to build, let alone test or support any theory.
Bearing witness, here, as always, entails positive questions.
This second-order question of \emph{why}, by the possible and preferable hypothetical of an intact mixed economy, European regional integration is so tilted, and its welfare states in such disarray, is an entirely positive question.

The \gls{ceec} periphery of the \gls{eu} is probably a good place to gauge the second-order theory of european integration.
It is, according to \cite{Foucault-1972-aa}, at those (continental) margins where you can see clearest the hegemonic discourse, and probably other non-structuralist dynamics, too.

I can only present some hunches.

\subsubsection[No Enlightened Understanding]{No Enlightened Understanding}

\begin{quote}
	%\emph{``Europa ist ein schÖner Kontinent.
	%Es ist freundlich und sehr sauber.
	%Europa hat mehrere schÖne Mädchen.''}\\
	\emph{``Europa is a nice continent.
	It is friendly and very clean.
	Europe has a couple of pretty girls.''}
	\\*
	--- Andrei, 17, Romania (in \href{http://eurolektionen.de}{Eurolektionen} \citeyear{DeRuffray2010})
\end{quote}

\begin{quote}
	%\emph{`Mit leerem Kopf nickt es sich leichter.'} (Sprichwort aus Deutschland)\\
	\emph{``An empty head makes it easy to nod.''}
	\\*
	--- From Germany
\end{quote}

The first, immediate hunch at a second-order theory is that we --- including our social sciences --- have so thoroughly, misunderstood the mixed economy, that the european polities are at the mercy of the few, interested and educated enough to make informed, but self-interested choices.
European democracy would then violate the criterion of ``enlightened understanding'' \citep{Dahl-1989-aa}.
The alternatives presented to the sovereign, and the choices which she assumes have so far detached from the actual abstractions of the mixed economy that any remaining political choice is, in the best case, spurious and, in the worst case, a rigged game.

\subsubsection[False Consciousness]{False Consciousness}

\begin{quote}
	\emph{``We did this to ourselves.''}
	\\*
	--- Cailin, 17, Romania (in \href{http://eurolektionen.de}{Eurolektionen} \citeyear{DeRuffray2010})
\end{quote}

A variant of this first hunch assumes, that maybe, the children of the revolutions in \glspl{ceec} hold systematic misunderstandings of the mixed economy.
As \citeauthor{Bonker2006} reminds us, the two constituent systems were indistinguishable under state socialism with its ``tight coupling of the economic and political sphere'' \citeyearpar[35]{Bonker2006}.
It is, maybe, little surprise that the people who courageously threw out state socialism, but were still not used to ``clear demarcation lines'' between institutions as distinct as the ``state budget, state-owned enterprises and the central bank'' \citeyearpar[36]{Bonker2006}, did not turn into expert policy makers of mixed economies over night.

Vividly remembering the horrible inadequacies of the ancien regime (for example, \citealt{Szikra2009}, \citealt{Millard1992}), one might assume that people, moreover, held a specifically  ``anti state ethos and a rejection of many social functions of the allegedly 'over-protective state' '' \citeyearpar[130]{Millard1992}.
40 years of socialist mismanagement might well have given the state a bad name, and thereby jinxed any effort to build well-balanced mixed economies.
Along these lines, \citeauthor{Inglot2008} reports that ``in companson with continental Western Europe, East Central European countries have developed a spectrum of political forces that has clearly shifted in favor of market-liberal, nationalist, and conservative ideologies and policies of the political right'' \citeyearpar[212]{Inglot2008}, on labor \citealt{Crowley2002}, more broadly \citealt{OrenOuto2001}).

This provides a plausible, partial explanation for the defunct mixed economies of \gls{ceec}, but, to be clear, it is still history first as tragedy, then as farce \citep{Marx1852}.
That, plagued by decades of state socialism, their citizens would subsequently have to endure an impotent welfare state when they most needed it in times of transition must surely be another instance of a revolution devouring its own children.

%from PhD proposal
\subsection{Why we Aren't Doing Any Better ---\\The Multi-Level, Veto-Playing Game of International Progressive Tax Reform}

In the second part of the thesis, I then turn to the question of why states are not selecting the superior policies made plausible in the preceding section.
This is, returning to the game-theoretic sketch of the argument, where strategies and equilibria are estimated.

\paragraph{Modeling the Game}
Again lacking empirical evidence of what lies beyond the status quo, this will be a modeling exercise.
Two related bodies of formal theorizing apply here.

One is the multi-level game theory of international tax harmonization \citep{Scharpf-1997-aa}.
Here, considerable effort will be devoted to understanding the strategic imperatives of governance both at the national and international level to reform taxation.
An econometrically informed modeling of the distributive effects \emph{within} and \emph{between} countries will be required to gauge this multi-level game.

Closely related, international tax harmonization will be modeled as a veto-player problem \citep{Tsebelis-2002-aa}.
Given the anarchic nature of the international realm (taking on a realist view), veto playing appears to be a reasonable approximation of international tax harmonization.
In particular, a modeled comparison between tax harmonization at the international and European level with their different veto rights may be instructive to estimate the overall significance of the model.

\paragraph{Alternative Theories}
The selection of game theory as an explanatory framework of course means to subscribes to an approximately realist view of international relations and a rational choice perspective on policy making in general.
This is an appropriate theoretical response to the questioned claim of ``welfare retrenchment''.

Still, a careful consideration of alternative theoretical and epistemological approaches will serve well to elucidate possible biases and explicate limitations of the chosen baseline approach of a quantifiable international cooperation problem.

Alternative concepts and theoretical frameworks for explaining the non-emergence of the proposed alternative tax configuration include:
\begin{description}
	%why bf formatting?
	\item{\textbf{Ideology}}
	Lack of support for such a broadly and boldly redistributive scheme.

	\item{\textbf{Path Dependency or Historical Institutionalism}}
	The new tax system would be incompatible with the institutions and mindsets of established types, for example \cite{BeramendiRueda-2007-aa}.

	\item{\textbf{Deficient Political Process, Public Deliberation or Discursive Institutionalism}}
	Literatures on the political process, as well as, on discursive institutionalism apply here \citep{Cerami2009a}.
\end{description}

Lastly, a discussion of the merits of competing theories of international relations in the light of an hypothesized international tax harmonization will be undertaken.