\documentclass[]{article}

\begin{document}

	%The baby (welfare), the bathwater (current capitalism), and the bathtub (financial capitalism).

%Economic policy history
	%until the 1980
		%it's all about smoothing the business cycle
		%monetary policy + fiscal policy + wage setting

	%Reagonomics / Thatcher
		%supply side economics
		%make the labor market flexible
		%get the state out of the market
		%goal: more market

	%Financial crisis
		%return of keynesianism
		%some demand side management
		%also: financial market stability
		%also: external imbalances


%Note on Cassidy: there's systems with negative feedback (human hormonal system) and positive feedback (a-bomb, financial markets).

%Notes on Cassidy Ch 7: there's a calculator that had the black-sholes equation right in built in. That's a neat analogy to show how lazy people had gotten about the fundamentals analysis.
%The Efficient Market Hypothesis was a straight-on attack on fundamentals analysis. That's an interesting thought.

%What you're doing if you're not doing fundamentals analysis …

%There's an efficient market hypothesis that's like a central processing unit, that's what the metaphor is. What's happening in a market with an information commons problem is simply that you have a Zirkelschlsus in your excel spreadsheet, excel doesn't allow that, but financial markets, they do.



%the union only has one interest rate,

	%## Geld & Konjunkturpolitik
	%Die Eurogruppe teilt mit dem Euro nicht nur eine Währung, sondern damit notwendigerweise auch einen Leitzins (und alle anderen Bestandteile der Geldpolitik). In der kurzen (und mittleren Frist) ist es Aufgabe der Geldpolitik (etwa durch niedrige Leitzinsen, Open Market Operations und "quantitative easing") den Konjunkturzyklus zu glätten. Wenn die Wirtschaft sich "überhitzt", also mehr Projekte finanziert (fristentransformiert) werden als wahrscheinlich erfolgreich sind, sollte die Geldpolitik Fristentransformation (damit: Geldschöpfung) durch höhere Leitzinsen etc. erschweren. Kommt es zu einer rezessiven/depressiven Kreditklemme, also einem übermaßigen Pessimismus neue Projekte betreffend, sollte die Geldpolitik die Fristentransformation (damit: Geldschöpfung) etwa durch niedrige Leitzinsen vereinfachen.
	%### Kein Optimaler Währungsraum (Mundell, Lerner)
	%Eine Währungsunion reduziert, oder genauer gesagt, vereinheitlicht die (oben beschriebene) monetäre Konjunktursteuerung der beteiligten Volkswirtschaften. Ökonomische Räume sollten daher bestimmte Kriterien erfüllen, um eine erfolgreiche Währungsunion einzugehen:
	
	%on ECB governance
	%What is a central bank?	
	%controls the money supply
	%by purchasing assets (open market operations)
	%by other means

	%interest rate
	%reserve requirements

	%holds foreign exchange reserves (to control the exchange rate)
	%assets on the central bank create interest, which accrues to the member states, according to the ECB capital key

	%Appointing the ECB board

	%EU council in the composition of heads of states and government

	%Independence needs a very clear mandate, otherwise it easily conflicts with accountaility
	%Art. 27: price stability (undefined), particularly the time horizon is unclear

	%defined by the ECB as below, but close to 2% in the medium turn
	%What is the BElassos-Samuelson-effect?
		%Tradable goods will go up
		%untradable goods will also increase in price, hence there is more inflation after monetary integration (empirically unclear)


%- Volle Faktormobilität: Gegeben für Kapital, nur formal gegeben für Arbeit (Sprache, Kultur, Umzug etc.)
	%- Flexibilität bei den Faktorpreisen. Dies ist auf den Arbeitsmärkten (Löhne) nur eingeschränkt gegeben, etwa wegen Tarifverträgen.
	%- Synchrone Konjunkturzyklen und Symmetrische Reaktionen zu externen Schocks (Ölpreise). Beides ist in der Eurozone nur eingeschränkt gegegeben (vgl. Boom in Spanien, Irland vs. Rezession in Deutschland)
	%### Stimulus
	%Auch Mitgliedsländer sind und waren nicht immer optimale Währungsräume (etwa: Wiedervereinigung Deutschland). Die westeuropäische Mischökonomie hatte bisher immer noch ein alternatives Instrument, um die Ungleichgewichte eines geeinten Währungsraums zu beheben: Fiskalpolitik, oder genauer gesagt, Konjunkturpakete. Außerdem konnten die nationalen Ökonomien leichter die Lohnentwicklung koordinieren, moderate Tarifabschlüsse durch sozialpolitische Maßnahmen unterstützen (geringere Steuern, Abgaben etc.), oder Arbeitsmärkte durch Arbeitslosenversicherung und -Vermittlung flexibilisieren ("Flexicurity").
	%Der Eurogruppe stehen, trotz dramatischer Ungleichgewichte, diese Instrumente nicht zur Verfügung.

	%      * In einer Perfect Currency Area (korrellierte Risiken, ähnliche Struktur) kann der Staat über den Wechselkurs Ungleichgewichte korrigieren:
	%         * Ungleichgewichte können entstehen durch Überforderung der öffentlichen Hand
	%         * Überschuldung der privaten Hand
	%      * Wichtig: Risiko der Geldpolitik: Beggar-thy-neighbor, kompetitive Abwertung
	%      * Wenn sich der Zinssatz ändert, gibt es mehr Schuldscheine. (in der kurzen Frist wächst die Wirtschaft
	%Alternativen für die EU
		%   * Geldpolitik ist keine nachhaltige Antwort.

	%Begg 2008 points out that OCA principles were rejected in favor of nominal convergence principles. For Begg 2008 look again at 5.

		%Apparently de Grauwe (2006 as cited in Bordo et al 2011) has argued for a deeper political union.\
	%Bingo moment in Bordo et al 2011: 4, citing musgrave. Fiscal policy is: allocation/efficiency, distribution and stabilization. OCA is neatly summarized right in here.

%De Grauwel 2007
	%on OCA, optimal currency area
	%is all about the relationship between structural features and symmetric shocks

	%labor market institutions (wage flexibility)
	%labor mobility
	%legal frameworks

	%Solutions

	%Fiscal redistribution (from Ireland to Germany)
	%Labor mobility (Lukas moving from Germany to Dublin)
	%Domestic fiscal adjustment

	%Deficit in Germany
	%Surplus in Ireland
	%this is what the SGP (Stability and Growth Pact) rules

	%Real exchange rate

	%... takes a long time (in the US between states, 4-7 years!), particularly with regard to immobile factors
	%note: there is a real exchange rate between Berlin and Bavaria
	%the question is just, how fast is this adjustment - and the consensus appears to be: it is too slow, hence, the EU is not an OCA
		
%de Grauwe 2007, Chapter 10 (p. 222-249)
	%note that the original 1977 MacDougall Report recommended that long with monetary union, budgets were centralized to some degree.
	% why aren't public budgets documented like the budgets of private firms, including a full balance sheet? how is this ENRON accounting even possible?

%Hodson 2009
	%Problems with fiscal federalism (514)
	%too slow to react to short-term economic shocks (really?)
	%more distributive than stabilizing (empirical record) (really?)
	%problems of adverse selection and moral hazard remain


%aggregate supply is inelastic in the long run, only elastic in the short run, b/c of animal spirits (e.g sticky wages, sticky prices, misperceptions/relative prices). Note that these are all the flipsides of the costs of inflation (see above)

%some of this may be ok for smoothing, but you've gotta breathe in and breathe out. Don't only do one thing. This is key. Someone called this the ``assymmetric ignorance'' of Greenspan, who would always breathe out, but never in.


%aggregate supply is inelastic in the long run, only elastic in the short run, b/c of animal spirits (e.g sticky wages, sticky prices, misperceptions/relative prices). Note that these are all the flipsides of the costs of inflation (see above)

%some of this may be ok for smoothing, but you've gotta breathe in and breathe out. Don't only do one thing. This is key. Someone called this the ``assymmetric ignorance'' of Greenspan, who would always breathe out, but never in.


%### Keine Abwertung
	%Abschließend können die Euroländer die Exportpreise ihrer Volkswirtschaften nicht mehr abwerten, wenn schwere Ungleichgewichte (Handelsbilanzdefizite [GR] oder -Überschüsse [DE]) vorliegen. Derartige Ungleichgewichte entstehen etwa durch Faktorpreisrigidität (Löhne), niedrige Leitzinsen und große Produktivitätsunterschiede. 
	%Die nationalen Mischökonomien weisen solche Ungleichgewichte ebenfalls auf (etwa: Deutsche Einigung), können aber anstelle der Abwertung auf fiskale Instrumente (Soli-Zuschlag) zugreifen.

%discuss capital flight with regard to greece in the imbalances and crises section.

%add BOP table here.

%some quote?

%Krugman: there is a link between impossibility triangle (free movement of capital, etc.) and EU integration (optimal currency area? Optimal integration are?)

%mundell-fleming says: you can only have independent monetary policy (Or: currency union by another name!!!!) or fixed exchange rate or free capital movement (GATT!) bingo! This is the link between trade and monetary policy.

%Also cite (Feldstein 2005 as cited in Begg 2008: 13) the Common Pool Resource problem of fiscal discipline under one currency.

%old notes
	%Neske: "To ignore economics, as a society, as we do is irresponsible given the global environment in which we live"
	%- political education
	%- structural issue of globalization
	%- the bankers look into value - the value of the complex to see the fundamentals
	%- make products simpler, sp that there is a chance to SEE the fundamentals
	%= also: make incentives: would YOU put your own money into it?
	%- oringinate to distribute violates this, it can't work. baad incentives
	%Neske says there is a 50-50 chance for hyperinflation OR tax the shit out of people
	%"why inflation will not come" (DB Research)
	%- really, that's whistly in the forest
	%- my third idea: make it an asset tax
	%- let's look at the cyclical overprodcution in isolation - what can be done about that?
	%- the state is bad at preventing cyclical overproduction (in part because of the time inconsistency issue)
	%Neske is very good: impressive.
	%political: this should go to what
	%I got o be a bit critical about the state
	%it really ls: crisis as a chance
	%asset tax: same as progressive inflation only farer and better (no chaos, no disincentive to invest)
	%Neske: this is big-ass crises, as big as it gets, really
	%Mildner: there is market failure, and there is state failure, too.
	%Prof. Dr. Steffens
		%- Europe is very balance-sheet driven, not so much market driven (US, UK)
		%- traditional, archaig presentation of banking in Germany (savings produce savings for innovations which are made by engineers).
	%-- Geither March 8 2008: on Causes
		%"some were the product of market force,s some wer ethe product of market failures. some were the result of incentives created by policy r regulation"
	%Neske: securitization is a good thing, no one wants to or could afford to go back to village banking
	%Stormy: it's the job of banks to translate long-term into short-term fincancing: maturity transofrmaion is the business of banks
	%Its not right to only blame managers, you have to look at the system.
	%Check out Taleb on the Black Swan and fat tails
	%The fun that negative equity is.
	%Adam smith is back, a bit
	%If it's to good to be true, it"s too good to be true.
	%Legal arbitrage is dangerous
	%100ing and missing of debt frees up euity.
	%Neske dodged the question on redistribution.
	%Neske: do both a bit, Tax and inflation.
	%Redistribution is what we forget about.
	%Neske asked what would he recommend for bank management
		%- make sure there bank has no interest in private riches (!)
		%Neske: "The hunger for growth and chance" is so different in the world. it could dramatically shift to Asia.
	%The Guy's name was PRof. Dr. Udo Steffen from the Frankfurt School of Finance & Management
		%- there is good news in bubbles, too: easy cash
		%- problem with deleveraging: there will be fewer profits, then less credit, then less profit
		%- you have to move from value at risk to conditional value at risk (what if others fail)
		%- all risks are correlated in extreme events, that is the very definition of extreme events
		%- from a micro-perspective, when a bank holds a CDS and a bond on a company, those two will offset each other
		%- securitization is a good thing, but it must remain with the originator
		%- look at whateve rMisnky wrote
		%The key to understand it, says the frankfurt school guy, is not to look at the facts in isolation, but in combination - this is the old mistake that we keep making. We have to think about correlates.
		%Drop in housing values, sometimes up to 36% in 2 years (US S&P Case-Shiller Home Price Index, Phoenes, Arizona, based on Datastream data)
		%Fundamental upside risk bias: if investment bankers work without their own capital, the worst thing that can happen is they get fired.


%\section{In praise of bubbles}. As long as they're not building real stuff, they aren't very wasteful. They really just from the suckers to everybody else. Look at the REAL waste (building houses) and the REAL intertemporal redistribution (spending spree). Cite "how housing wealth isn't real wealth", get back to chapter 7 in the book on money

%financial intermediaries are: banks, funds, insurance. They do different jobs. For financial paper: these products and intermediaries are great, they do great things.

%EMH is a commons. It is a self-defeating hypothesis.

%derivatives: only when there is insurable risk, otherwise. and only when there are well-capitalized exchanges.

%Hayek is right about prices = information
        
        
%emergent properties:
        %   * self-reinforcing, positive feedback such as Herding, or a nuclear explosion
	%   * negative feedback, perfect market, human hormonal system Pareto is a good way to tell whether thing are going bad, but not a good way to know whether things are going well.
        
%Grossmann and Stiglitz: on the impossibility of the EMH, basic formulation of cooperation problem of information.

%note that, according to (\citealt{Mankiw-2004-aa}: 575), government debt eats up the lonable funds market, thereby crowding out other investment.

%The nurse (Irvine) was deep into negative equity.
%Krugman: link b/w panics + breakdown of Keyes, Ch 5
%also check back on the reasons of the financial crisis towards the end of Krugman (last chapter?)
%Krugman: there is a link between impossibility triangle (free movement of capital, etc.) and EU integration (optimal currency area? Optimal integration are?)
%Here's what you do: deconstruct trade, money, etc. and wonder: why are we so fucked up?
%Consider in greater detail: Galbraith on "conventional wisdom" (as cited in Cassidy Ch 2).
%Figure out the heterodox chicago economist of indian descent who disagreed with the deregulation mantra (as cited in Cassidy Ch 2).
%http://www.ted.com/talks/kevin_slavin_how_algorithms_shape_our_world.html
%check out this again: http://www.youtube.com/watch?v=qOP2V_np2c0
%Wenn ich es richtig sehe verletzt die Transaktionssteuer u.U. das Baby: Arbitrage ist gut, sorgt für intertemporale und globale Preisgleichheit. 
%Außerdem lässt sie das Badewasser drin: Die Margen von spekulativen Attacken (sic!) sind so groß, da kann keine relativ geringe Transaktionssteuer helfen.
%Zudem ist, wie so oft, unklar wer letztlich diese Steuer trägt (SMEs? Shareholders? Consumers? Debtors, Creditors?)


%Ideas for the comment:
	%you have to improve public accounting, think investment, demographic change, future obligations
	%You have to improve public accounting and "stress-test" it under different growth and shock scenarios (Spain, Ireland) (think Greece seeling CDSes on future incomes.
	%What about equity financing for states? (the problem with debt is, you only notice it, once shit happens) (financial markets have not as assumed in the past, punished markets, and particular when the interest rate risk of default is gone, it becomes a really bad measurement or incentive). This, I think, is a key idea. CDSs are not the problem. The problem is that they are the only financial incentive.
	%this is also key for Stormy's paper. Debt financing captures only extreme downside risks, not anything in between or above that.
	%and also, it's the normal herding behavior problem.
	%IMF is better to avoid moral hazard
	

%Here's what you can do to tackle the financial crisis:

		%extending holding periods
		%regulate derivatives (over the counter)
		%anticyclical policy
		%tax / tobin / transaction tax?
		%warwick commission: flexible capital requirements, rise/fall with countercyclical
		%was
		%ex ante for bubbles

		%capital requirements for risky bubbles
	%monetary policy (bad)

	%ex post for bubbles

	%resolution/bailing out/ chapter 7 / moral hazard
	%stress testing (ex ante, but covers only systemic risk)

	%macroprudential
	
%make use of the ludic fallacy
	%via wikipedia: According to Taleb, statistics works only in some domains like casinos in which the odds are visible and defined. Taleb's argument centers on the idea that predictive models are based on platonified forms, gravitating towards mathematical purity and failing to take some key ideas into account:

%check this out: http://www.fooledbyrandomness.com/tenprinciples.pdf

	%ex-post: resolution/bailing out/ chapter7 moral hazard
	%stress testing (ex ante, but only systemic risks)

	%microprudential

	%own capital
		%BASEL
		%off balance sheet stuff

		%these are all really, really blunt tools, they suck
	%what about the state inserting random noise?
	
	%M) Financial intermediaries such as commercial banks offer three services to broker deals between creditors and debtors that would not otherwise have happened. 
	%- Maturity transformation. They convert (usually) short-term liabilities (e.g. savings) into long-term assets (e.g. a factory). Banks bear the liquidity risk  of this transformation, that materializes if, for example, a sufficiently large, long-term debt defaults or sufficiently many short-term creditors withdraw at the same time. This liquidity risk of banks can give rise to a self-fulfilling prophecy in a bank-run. When an individual bank-run cascades through a financial system, it may cause a credit crunch or even a systemic banking crisis.
	%- Risk transformation. They convert risky investments into less risky investments by spreading among many lenders and taking on many borrowers (with different, ideally uncorrelated risks).
	%- Convenience denomination. They convert small deposits (one life insurance) into large loans (a new factory) and large deposits (a pension fund) into small loans (many mortgages). (This conversion may be least specific to financial intermediaries; capital markets can achieve much the same).
	%In this \emph{maturity transformation} generates (book) money.
	
	%more money stuff, no idea where this belongs
	%u)  When maturity transformation expands the money supply faster than economic growth warrants more promises are made than kept and inflation ensues (the \emph{monetarist} quantity theory of money, see first \citealt{Newcomb1885}, refined in \citealt{Fisher1911} revived in \citealt{Friedman1993}). Alternatively, inflation ensues when aggregate demand grows faster than aggregate supply (demand-pull inflation) or when external price shocks occur (cost-push inflation) (the \citeauthor{Keynes1936}ian \citeyear{Keynes1936} view). Similar dynamics hold for deflation, if with opposite signs. Both inflation and deflation are self-reinforcing (through price-wage spirals). 

	%States have a monopoly on issuing currency, heavily regulate banks and calibrate the money supply through their central bank (including setting the central bank interest rate, quantitative easing and open market operations).

%Ideas for the comment:
	%you have to improve public accounting, think investment, demographic change, future obligations
	%You have to improve public accounting and "stress-test" it under different growth and shock scenarios (Spain, Ireland) (think Greece seeling CDSes on future incomes.
	%What about equity financing for states? (the problem with debt is, you only notice it, once shit happens) (financial markets have not as assumed in the past, punished markets, and particular when the interest rate risk of default is gone, it becomes a really bad measurement or incentive). This, I think, is a key idea. CDSs are not the problem. The problem is that they are the only financial incentive.
	%this is also key for Stormy's paper. Debt financing captures only extreme downside risks, not anything in between or above that.
	%and also, it's the normal herding behavior problem.
	%IMF is better to avoid moral hazard

\end{document}